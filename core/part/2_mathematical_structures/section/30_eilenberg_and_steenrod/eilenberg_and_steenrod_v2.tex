\section{Eilenberg and Steenrod: From Control to Classification}

If Pontryagin taught us how to steer a system toward an optimal path,  
then Samuel Eilenberg and Norman Steenrod taught us how to classify the very space the system moves in.

Where Pontryagin’s principle guided trajectories through a dynamic landscape,  
Eilenberg and Steenrod stepped back and asked:

\begin{quote}
“What is the underlying shape of this landscape itself?  
What features remain unchanged, no matter how we stretch or bend it?”
\end{quote}

They weren’t optimizing a path through a space.  
They were discovering the **intrinsic structure of the space itself.**

\bigskip

\subsection*{The Leap from Pontryagin to Eilenberg and Steenrod}

Pontryagin’s optimal control relied on the geometry of trajectories:  
how to choose controls so that a system follows an extremal path.

But what happens when you don’t care about a particular trajectory,  
but about **what kinds of paths, holes, and surfaces are even possible in a space?**

✅ Pontryagin optimized within a known space.  
✅ Eilenberg and Steenrod **abstracted what it means to know a space at all.**

They pioneered **homology and cohomology theories**:  
algebraic tools for classifying spaces up to continuous deformation.

\[
\boxed{\text{Not the path, but the structure of possible paths.}}
\]

Where Pontryagin solved **control problems** by tracing curves,  
Eilenberg and Steenrod solved **topological problems** by counting the “holes” that constrain such curves.

\bigskip

\begin{tcolorbox}[colback=gray!5!white, colframe=black, title=\textbf{Sidebar: The Shift from Pontryagin to Eilenberg–Steenrod}, fonttitle=\bfseries, arc=1.5mm, boxrule=0.4pt]

\begin{tabular}{>{\raggedright}p{4cm} >{\raggedright}p{5.5cm} >{\raggedright\arraybackslash}p{5.5cm}}
 & \textbf{Pontryagin} & \textbf{Eilenberg–Steenrod} \\
\midrule
Key insight & Optimal path through a known dynamic system & Algebraic invariants classifying the topology of a space \\
Focus & Trajectories governed by differential equations & Homology and cohomology groups encoding connectivity, holes \\
Equation type & Pontryagin’s Maximum Principle; Hamiltonian systems & Functorial axioms; exact sequences; algebraic topology
\end{tabular}

\end{tcolorbox}

\bigskip

\subsection*{From Path to Pattern}

Pontryagin showed how to control a system’s state over time.

Eilenberg and Steenrod showed that certain properties of space don’t depend on coordinates, metrics, or even geometry—they depend only on **topological structure.**

✅ No matter how you stretch, twist, or deform a coffee cup,  
✅ Its fundamental topological features (like a single hole) remain unchanged.

Their insight was to encode these invariants algebraically:  
turning **spaces into algebraic objects**—homology groups, cohomology rings—that classify spaces up to homeomorphism.

They defined cohomology not by construction,  
but by a set of **axioms (the Eilenberg–Steenrod axioms)** that every “reasonable” cohomology theory must satisfy.

This axiomatic approach allowed for a powerful, unified language:  
just as Peano had axiomatized arithmetic,  
Eilenberg and Steenrod axiomatized the very algebra of topology.

\bigskip

\subsection*{A New Kind of Geometry}

Where Pontryagin’s trajectories traced **paths in state space**,  
Eilenberg and Steenrod’s homology traced **the “holes” that obstruct paths:**

✅ Loops that cannot be shrunk to a point.  
✅ Surfaces that cannot be filled in.  
✅ Higher-dimensional voids invisible to ordinary geometry.

Their work shifted the mathematical lens:

From:

\[
\boxed{\text{“What is the optimal way to move through a space?”}}
\]

To:

\[
\boxed{\text{“What is the structure of the space itself that makes such movement possible or impossible?”}}
\]

\bigskip

\begin{quote}
In Euler, we computed forces.  
In Lagrange, we minimized action.  
In Hamilton, we traced flows.  
In Jacobi, we found surfaces.  
In Cayley, we abstracted transformations.  
In Fourier, we decomposed vibrations.  
In Riemann, we curved the space.  
In Gibbs, we calculated fields.  
In Peano, we defined the space.  
In Christoffel, we corrected differentiation.  
In Ricci, we encoded curvature.  
In Levi-Civita, we transported vectors.  
In Einstein, we made curvature into gravity.  
In Noether, we discovered symmetry writes the laws.  
In Nash, we learned balance writes the game.  
In Pontryagin, we learned how to steer the game.  
In Eilenberg and Steenrod, we discovered that the space has its own algebra.
\end{quote}

\subsection*{From Optimization to Topological Invariants}

Eilenberg and Steenrod’s algebraic topology didn’t tell you how to move through a space;  
it told you **what kind of space you were moving through in the first place.**

Their work provided the algebraic scaffolding that made later advances in geometry, topology, physics, and category theory possible.

And they showed that beneath every geometry, every trajectory, every optimization problem,  
there lies an algebra of holes, connections, and cycles  
waiting to be counted, classified, and understood.

