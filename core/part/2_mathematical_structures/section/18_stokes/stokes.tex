\section{Stokes and the Circulation of Motion: From Orbits to Fields}

Where Cayley had reduced geometry to algebraic transformations, Sir George Stokes engaged a different, but equally fundamental, aspect of nature: the relation between motion and area, between the circulation of forces and the fields they produce.

Stokes, born in 1819 in Ireland, approached mathematics through the study of real physical phenomena. His early work in hydrodynamics and optics led him to explore how forces and motions distribute through continuous media. In these investigations, Stokes gradually uncovered a general principle, connecting the behavior of a quantity along the boundary of a region to its behavior within the region itself—a principle that would later bear his name.

\subsection{The Circulation of Celestial Motion}

Although Stokes did not directly work on celestial mechanics, his ideas resonate deeply with the structure of Kepler’s Second Law. Kepler had shown that the radius vector from the sun to a planet sweeps out equal areas in equal times. Expressed geometrically, the rate of area swept out is constant:

\[
\frac{dA}{dt} = \text{constant}.
\]

This reflects the conservation of angular momentum, a consequence of the central nature of the gravitational force: since the force always points along the line joining the planet and the sun, it produces no tendency to twist the motion about the center.

But one may also think of the planet’s path as tracing out a closed curve in space. Along this curve, the planet possesses a certain velocity at each point. Stokes was deeply concerned with the general question of how such quantities circulate along closed paths. He considered integrals of the form:

\[
\int_{C} \mathbf{V} \cdot d\mathbf{s},
\]

though in his time this would have been phrased as:

\[
\int_{C} V_x\, dx + V_y\, dy + V_z\, dz,
\]

where \(C\) denotes a closed curve, and \((V_x, V_y, V_z)\) are the components of the velocity at each point along the curve.

\subsection{From Circulation to Surface Flux}

Stokes discovered that such circulation integrals are intimately connected to what occurs on the surface bounded by the curve. He formulated the relation that would later be called his theorem:

\[
\int_{C} V_x\, dx + V_y\, dy + V_z\, dz = \iint_{S} \left( \frac{\partial V_z}{\partial y} - \frac{\partial V_y}{\partial z} \right) dy\, dz + \left( \frac{\partial V_x}{\partial z} - \frac{\partial V_z}{\partial x} \right) dz\, dx + \left( \frac{\partial V_y}{\partial x} - \frac{\partial V_x}{\partial y} \right) dx\, dy.
\]

In modern language, the right-hand side expresses the total "curl" of the vector field across the surface \(S\), but for Stokes, it was simply a sum of partial derivatives expressing how small rotational tendencies accumulated over the surface.

In the special case of planetary motion under central force, the components of these rotational derivatives vanish almost everywhere because the planet’s motion is always tangential to its orbital path, and the gravitational pull has no component orthogonal to the radius. Thus, while Stokes' theorem formally applies, the double integral often contributes nothing, leaving the circulation integral constant—consistent with Kepler’s discovery that area is swept out uniformly.

\subsection{From Physical Intuition to General Principle}

Although his original investigations focused on fluids and elastic media, Stokes' insights reached far beyond any particular domain. The idea that integrals over boundaries could be expressed as integrals over surfaces was profoundly general, applying to the motion of air, water, planets, and eventually to electric and magnetic fields.

In this way, Stokes introduced into mathematics a new kind of duality: the connection between local tendencies to rotate or twist, and global circulations around closed loops. This relationship would later find its full abstract form in the language of differential forms and manifolds developed by Riemann and others. But for Stokes, these principles emerged directly from physical intuition, grounded in the concrete study of motion and force.


\subsection{The Additivity of Regions: The Heart of Stokes' Theorem}

At its core, Stokes' theorem expresses a simple but powerful idea: large regions can be understood by breaking them into smaller pieces. The behavior of a quantity along the boundary of the whole is just the sum of its behavior along the boundaries of the parts.

Imagine a region divided into many small subregions, each with its own oriented boundary. Along each boundary, one may compute an integral of some quantity—what Stokes thought of as circulation, or later mathematicians would call the integral of a differential form \(\omega\).

If we sum these integrals over all the small pieces, something remarkable happens: the contributions along the internal shared edges cancel out, because each shared boundary is traversed once in one direction for one region, and once in the opposite direction for the neighboring region. Only the contributions from the outer boundary remain.

\begin{center}
PICTURE HERE
\end{center}

\textit{(Illustration of how boundary terms cancel when summing over subdivided regions)}

Thus, integrating \(\omega\) around the entire boundary is the same as summing the integrals over the boundaries of all the smaller pieces:

\[
\int_{\partial R} \omega = \sum_{i} \int_{\partial R_i} \omega.
\]

If we now imagine refining the subdivision further—breaking the region into smaller and smaller parts—this additive structure suggests that there exists some quantity, spread throughout the interior, whose integral over the whole region gives the same result as integrating \(\omega\) around the boundary. That interior quantity is, in modern language, what we call \(d\omega\), the "derivative" of \(\omega\).

But for Stokes, the conceptual leap was the additivity itself: that integration over regions behaves like summing up over parts, and that boundary contributions naturally cancel when interior edges are shared. This observation was rooted not in abstract algebra, but in the very physical problems Stokes was studying—fluid motion, circulation, and the distribution of forces through space.

It would only be later, in the hands of mathematicians like Riemann, Poincaré, and Cartan, that this idea of "breaking up regions into infinitesimal parts" would evolve into the modern language of integration on manifolds and exterior differentiation.
