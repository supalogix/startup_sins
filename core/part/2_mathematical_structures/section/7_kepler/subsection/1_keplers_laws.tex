\subsection{What Happens When a Theologian Does Too Much Math?}

If Galileo’s crime was mocking the Pope, Johannes Kepler’s was simply existing at the wrong time. A devout Lutheran, an obsessive mathematician, and a man who thought the universe was built on divine geometry, Kepler \textbf{just wanted to do horoscopes and prove that God was a geometer.} Instead, he destroyed Aristotle’s final hope of keeping planetary motion simple and got himself tangled in religious wars, witch trials, and his own mentor’s murder.

Before Kepler, everyone was still clinging to perfect circles. The idea that planets must move in circles was deeply ingrained in Aristotelian physics and medieval theology. Why? Because circles are perfect. And God wouldn’t make something as messy as an ellipse.

Kepler, however, wasn’t interested in what people wanted planetary motion to look like. He had actual data.

Kepler was hired by \textbf{Tycho Brahe}, an astronomer whose personality could be described as \textit{"drunk Viking who traded Thor for Jesus."} Brahe had a nose made of metal (lost in a duel), a pet moose (which died from drinking too much beer), lived a life of sexual immorality (essentially never officially married his wife because his high social status made it illegal to marry a commoner),  and had \textbf{the most accurate astronomical data in history} (which he refused to share).

\medskip

\begin{tcolorbox}[colback=blue!5!white, colframe=blue!50!black, title={Historical Sidebar: Tycho Brahe’s Uraniborg — The Hubble of the Renaissance}, breakable]

  Before satellites orbited Earth and billion-dollar telescopes peered into deep space, there was \textbf{Tycho Brahe}—the man who turned a Danish island into the astronomical epicenter of the 16\textsuperscript{th} century.

  \medskip
  
  In 1576, backed by royal funding (because even then, science needed patrons), Brahe built \textbf{Uraniborg}—Latin for "Castle of the Heavens"—on the island of Hven. This wasn’t just a fancy house with a telescope. It was the world’s first **purpose-built scientific observatory**.
  
  \medskip
  
  Brahe operated in the final age of \textbf{pre-telescopic precision}. Armed with massive quadrants, sextants, and armillary spheres—some stretching meters across—Brahe achieved observational accuracy unheard of for his time.

  \medskip
  
  How good was it?  His naked-eye measurements of planetary positions were accurate to within a single arcminute (\(1/60^\circ\))—a feat so extraordinary that his data remained unmatched until long after telescopes became standard.
  
  \medskip
  
  In many ways, \textbf{Uraniborg} was the **Hubble Space Telescope** of its era:

  \medskip
  
  \begin{itemize}
    \item It was a state-of-the-art facility dedicated entirely to mapping the cosmos.
    \item It pushed the limits of technology—not with lenses, but with sheer scale and craftsmanship.
    \item It generated the most precise astronomical datasets in human history up to that point.
  \end{itemize}

  \medskip
  
  But unlike Hubble, which floats serenely in orbit, Uraniborg came with Renaissance flair:

  \medskip
  
  \begin{itemize}
    \item A private printing press for publishing star charts.
    \item Alchemical laboratories—because why not mix astronomy with a little gold-making?
    \item A moat, gardens, and of course, quarters for entertaining nobles (and perhaps a tipsy moose or two).
  \end{itemize}
  
  \medskip
  
  Brahe’s observatory wasn’t just about science—it was a symbol of intellectual ambition, aristocratic pride, and the belief that with enough funding and metal instruments, you could quite literally chart the heavens.

  \medskip
  
  And when Brahe died, it wasn’t Uraniborg’s grandeur that survived—it was the mountain of data he left behind, waiting for a mathematician named Kepler to decode the true shape of planetary motion.
  
\end{tcolorbox}

\medskip

When Brahe died mysteriously in 1601 (possibly poisoned, but let’s not point fingers), Kepler inherited his data. And with years of obsessive calculations, he formulated three laws that finally shattered Aristotelian physics.

But it was \textbf{Kepler’s Second Law} that changed everything.

\begin{figure}[H]
\centering

% === First row ===
\begin{subfigure}[t]{0.45\textwidth}
\centering
\begin{tikzpicture}
  \comicpanel{0}{0}
    {Tycho's Father}
    {}
    {\footnotesize You can’t marry her. She’s a commoner. It’s an affront to the family name.}
    {(0,-0.6)}
\end{tikzpicture}
\caption*{Nobility: allergic to love.}
\end{subfigure}
\hfill
\begin{subfigure}[t]{0.45\textwidth}
  \centering
  \begin{tikzpicture}
    \comicpanel{0}{0}
      {Minister}
      {}
      {\footnotesize Wed to a commoner? That’s being unequally yoked, in every sense.}
      {(0,-0.6)}
  \end{tikzpicture}
  \caption*{Sanctity denied by class and scripture.}
  \end{subfigure}
  

\vspace{1em}

% === Second row ===
\begin{subfigure}[t]{0.45\textwidth}
\centering
\begin{tikzpicture}
  \comicpanel{0}{0}
    {Family Banker}
    {}
    {\footnotesize That marriage jeopardizes your inheritance. It’s an affront to generational wealth.}
    {(0,-0.6)}
\end{tikzpicture}
\caption*{Romance vs. compounding interest.}
\end{subfigure}
\hfill
\begin{subfigure}[t]{0.45\textwidth}
\centering
\begin{tikzpicture}
  \comicpanel{0}{0}
    {Tycho Brahe}
    {}
    {\footnotesize Hey babe, you can be my live-in girlfriend. Forever.}
    {(0,-0.6)}
\end{tikzpicture}
\caption*{Living in sin... with style.}
\end{subfigure}

\caption{Tycho Brahe: Losing his nose in a duel was somehow not the most dramatic part of his life.}
\end{figure}


