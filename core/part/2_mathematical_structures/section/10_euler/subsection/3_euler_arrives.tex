\subsection{Euler Arrives: The Absolute Chad of Notation}  

Euler, being the mathematical powerhouse that he was, looked at the battlefield and decided it was time to \textbf{take out the trash}. He introduced a \textbf{consistent, structured notation} that made calculus readable, logical, and actually useful outside of academic slap fights.

\[
f(x) = x^2 + 3x + 5
\]

Euler’s notation had several advantages:

\begin{itemize}
    \item \textbf{Systematic:} Instead of vague symbols and dots, every function was clearly defined with \( f(x) \).  
    \item \textbf{Extendable:} Higher derivatives were easy to write: \( f^{(n)}(x) \). No more playing \textbf{Guess the Meaning of the Extra Dot.}  
    \item \textbf{Algebra-Friendly:} Differentiation and integration were now \textbf{manipulable equations} instead of mystical processes.  
    \item \textbf{Universally Applicable:} Euler’s system wasn’t just for motion—it worked in \textbf{physics, economics, engineering, and beyond.}  
\end{itemize}

\begin{center}
\renewcommand{\arraystretch}{1.5}
\begin{tabular}{|c|c|c|}
\hline
\textbf{Mathematician} & \textbf{Function Notation} & \textbf{Derivative Notation} \\ \hline
Newton & $x$ & $\dot{x}, \ddot{x}$ (Fluxions) \\ \hline
Leibniz & $y$ & $\frac{dy}{dx}$ (Infinitesimals) \\ \hline
Euler & $f(x)$ & $f'(x)$ (The Chad of Notation) \\ \hline
\end{tabular}
\end{center}
