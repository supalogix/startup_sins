\section{Riemann Bends the World: Motion Finds a New Geometry (1854)}

Hamilton taught us to see motion as structure. He turned trajectories into flows and mechanics into geometry. The key player? The mathematical expression of energy—expressed through differentials and the relationships between position and motion. It told us how change occurred, and how quantities evolved in time. Motion became a matter of structure and transformation. Clean. Elegant. Flat.

But Hamilton’s world was still smooth and Euclidean. Space was a fixed, invisible stage. The structure of measurement was always the same, no matter where you were. 

\textbf{Then Riemann bent the world.}

In his 1854 habilitation lecture—one of the most quietly explosive talks in the history of mathematics—Bernhard Riemann asked a question that detonated the foundations of geometry:

\begin{quote}
\textit{What if the way we measure distance itself could vary from place to place?}
\end{quote}

In flat space, the infinitesimal distance between two points is given by:
\[
ds^2 = dx^2 + dy^2 + dz^2
\]
This is a special case of a more general form—a quadratic expression in the differentials.

But Riemann didn’t want “standard.” He wanted space to be free to twist, stretch, and curve. So he proposed something far more flexible:

\[
ds^2 = \sum_{i,j=1}^n g_{ij}(x) \, dx^i dx^j
\]

Here, the functions \( g_{ij}(x) \) vary smoothly from point to point, encoding how lengths and angles change across space. They are not constants—they are the very definition of geometry. The geometry of space wasn’t fixed anymore. It varied. Locally. Smoothly. Dynamically.

\textbf{Hamilton gave us the form of motion. Riemann gave us its texture.}

In Riemann’s world:
\begin{itemize}
  \item Direction is no longer universal—it depends on where you are.
  \item Distance is no longer absolute—it bends and twists with the shape of space.
  \item The quadratic expression becomes a local law: \textbf{“How is measurement defined here?”}
\end{itemize}

The real magic? This wasn’t just geometry. This was a way to guide motion.

The quadratic form tells us how steep a slope is. How far a step takes us. It defines the very \emph{feel} of change. Even without the language of gradients or optimization, Riemann laid the foundation for describing how motion responds to structure.

\textbf{In Riemann’s world, motion doesn’t just follow rules. It follows the shape of space itself.}

\begin{tcolorbox}[colback=blue!5!white, colframe=blue!50!black,
title={Sidebar: Hamilton’s Differentials Meet Riemann’s Forms}]
Hamilton introduced the idea of describing systems through differential relationships—time, change, and structure.

Riemann transformed these differentials into a \textbf{field}—a local rulebook for measuring distances and angles at every point.

Together, they gave us the geometric view of motion:
\begin{itemize}
  \item Hamilton: Motion as structure in time
  \item Riemann: Measurement as structure in space
  \item Unified: Motion unfolds within curved measurement itself
\end{itemize}
\end{tcolorbox}


\subsection{Descent Finds a Geometry: When Optimization Started Walking the Curve}

By bending the rules of space, Riemann didn’t just rewrite geometry—he reshaped how motion \textit{happens}.

With a local dot product at every point, you can now define what it means to move downhill—not just in flat terrain, but across landscapes of curvature. Whether you're walking across a hilly manifold, navigating a space of probability distributions, or optimizing a neural network, one truth remains:

\textbf{Motion listens to geometry.}

\begin{quote}
If Hamilton gave us the music of motion,  
Riemann handed us the score—drawn not on a grid,  
but on a surface that bends and curves beneath each step.
\end{quote}

This changed everything.

Gradient descent was no longer just about partial derivatives—it became a choreography. A direction, weighted and warped by the local metric. You weren’t just chasing the steepest slope. You were following the slope as measured by the ground you stood on.

\textbf{Descent became a dialogue between direction and space.}

And when we get to optimization in machine learning—on curved parameter spaces, in high-dimensional loss surfaces—Riemann’s influence will echo again:

\begin{quote}
To find the optimal path, you don’t just fall downhill.  
You fall \textit{with respect to the space you’re in}.
\end{quote}



\subsection{Descent Finds a Geometry: When Change Followed Form}

Once Riemann allowed geometry to vary across space, motion itself became something more than a path through coordinates. It became a response to the shape of space.

With a different expression for distance at every point—defined by a smoothly varying quadratic form—Riemann gave us the ability to describe how motion should unfold across uneven surfaces. Instead of following straight lines as in Euclidean space, motion would follow the paths that best reflect the local structure. These are the paths of least or stationary distance: the ones nature selects.

Whether a body moves along a curved surface, or an influence spreads through a non-uniform region, its trajectory is no longer dictated by global simplicity—it is shaped by local measurement. Even without the language of "optimization," the idea is clear: motion follows the form of space.

\begin{quote}
If Hamilton described how motion moves through time,  
Riemann described how it listens to the form of space.
\end{quote}

This idea—that motion follows geometry—reshaped physics. Once space could curve, the very notion of a "straight path" had to be redefined. And it was.

The calculus of variation, already used to describe physical principles of least action, now had a geometry that could adapt to curvature. The result: motion that flows not arbitrarily, but in agreement with the structure that Riemann laid out.

\textbf{From here, the story returns to the sky.}

\subsection{Kepler Reimagined: When Planets Obeyed Curved Geometry}

Let’s return to a law that once looked merely poetic:

\begin{quote}
\textit{A planet sweeps out equal areas in equal times as it orbits the Sun.}
\end{quote}

To Kepler, this was celestial harmony. To Newton, a byproduct of force.  
But to us—living in a post-Riemann, post-Hamilton world—it reveals something deeper:

\textbf{Motion that follows geometry. Flow that preserves structure.}

Kepler’s Second Law isn’t just about ellipses—it’s about conservation. In Hamilton’s mechanics, this law emerges from the conservation of angular momentum. But in the language of geometry, it emerges from a deeper source: symmetry.

When motion is governed by a structure that does not change in time or rotate with direction, then quantities must be preserved. In the case of planetary motion, this preservation is visible as the sweeping of equal areas—an echo of symmetry within curved space.

Now imagine that configuration space itself is curved, as Riemann allowed. Even there, a gravitational potential still defines a field of influence. But the response to that influence is no longer measured by fixed rulers—it is measured through the variable form of the space.

A planet moving in a gravitational field is not merely falling. It is following the path that, in the geometry of space, most efficiently expresses that fall.

\begin{itemize}
  \item The gravitational potential defines the influence.
  \item The quadratic form of space defines how distances—and thus motion—are measured.
  \item Together, they determine how a planet moves.
\end{itemize}

In this world, Kepler’s law remains true—not as a numerical coincidence, but as a geometric constraint. The preservation of area reflects the invariance of the structure. The curvature of space does not prevent the law—it explains it.

\begin{tcolorbox}[colback=blue!5!white, colframe=blue!50!black,
title={Kepler’s Law in Riemann’s Language}]
Kepler’s Law describes motion that obeys two principles:

\begin{itemize}
  \item A field that guides motion (gravitational influence)
  \item A structure that preserves form (geometric invariance)
\end{itemize}

From this interplay, the orbit arises—not as a simple curve, but as a geometric necessity.
\end{tcolorbox}

In the end, Kepler wasn’t just watching planets.  
He was witnessing the earliest glimpse of curved geometry in motion—  
A truth Riemann would give language to, and Einstein would one day turn into gravity itself.



\begin{tcolorbox}[colback=gray!5!white, colframe=black, title={Historical Sidebar: How Gaussian Curvature Paved the Road to Riemann}]
  Before Riemann dared to generalize geometry with a mutable dot product, his advisor had already whispered a quiet revolution into the ears of mathematicians.
  
  In 1827, Carl Friedrich Gauss published his \textit{Disquisitiones Generales circa Superficies Curvas}, where he introduced the concept of \textbf{Gaussian curvature}—a way to measure how curved a surface is at a point, using only information intrinsic to the surface itself. No need to peek into the surrounding 3D space.
  
  This was heretical for the time.
  
  Up to that point, curvature was something you defined by comparison to flatness in space. Gauss showed it could be computed purely from within. The result: a surface could know its own shape without referencing an external world.
  
  His masterpiece, the \textit{Theorema Egregium}, made this explicit:
  \begin{quote}
  \textit{The curvature of a surface is invariant under local isometries—it is an intrinsic property.}
  \end{quote}
  
  That idea—geometry from within—was the seed Riemann nurtured into a full-grown jungle. Where Gauss measured curvature at a point on a surface, Riemann asked: 
  \textbf{What if every point of space had its own geometry?}
  
  So while Gauss used curvature to understand shapes,
  Riemann used it to reimagine space itself.
  
  \textbf{From Gauss’s curved surfaces to Riemann’s curved worlds—differential geometry had found its playground.}
\end{tcolorbox}
