\section{Al-Battānī and the Rise of Trigonometric Functions: From Geometry to Algebra}

\subsection{From Chords to Sines: A New Language for the Sky}

By the 9\textsuperscript{th} century, Islamic astronomers had inherited the full technical apparatus of Greek astronomy: Ptolemy’s chord tables, geometric epicycles, and the firmament's nested spheres. But they didn’t just preserve it — they transformed it.

One of the most important of these reformers was \textbf{Al-Battānī} (c.~858–929), a Syrian astronomer whose work bridged classical Greek astronomy and the new Islamic mathematical tradition. In his magnum opus, the \textit{Kitāb az-Zīj}, he did something subtle but revolutionary:

\textbf{He replaced chords with sines.}

Where Ptolemy had computed the straight-line segment between arc endpoints, Al-Battānī instead used the vertical projection from a point on the arc down to the circle’s radius — what we now call the \textbf{sine function}.

\medskip

This was not just notational convenience. It was a conceptual shift:

\begin{itemize}
    \item \textbf{Chords} are geometric: defined by line segments across circles.
    \item \textbf{Sines} are algebraic: functions of angles, definable and tabulatable.
\end{itemize}



\subsection{The New Trigonometric Arsenal}

Al-Battānī introduced and systematically used the sine (\( \sin \)), cosine (\( \cos \)), and tangent (\( \tan \)) functions in computations involving:

\begin{itemize}
    \item Shadow lengths and solar altitudes,
    \item Angles of elevation for celestial bodies,
    \item Spherical triangles for planetary motion.
\end{itemize}

He also derived several trigonometric identities, including:
\[
\cos(x) = \sin(90^\circ - x)
\]
\[
\tan(x) = \frac{\sin(x)}{\cos(x)}
\]

These were the building blocks of what would eventually become modern trigonometry.

\medskip

\begin{HistoricalSidebar}{Sines as Divine Ratios}

In the intellectual climate of 9th-century Islam, mathematics was not merely a technical tool but a means to glimpse the unity of God’s creation. Al-Battānī’s adoption of the \emph{sine} (Arabic \emph{jiba}) over the Greek \emph{chord} carried resonances beyond convenience:

\medskip

\begin{itemize}
    \item \textbf{Tawḥīd (Divine Unity).}  
    While a chord partitions a circle into discrete segments, the sine—being a single-valued ratio—mirrors the Qur’anic emphasis on the oneness and indivisibility of the Divine essence.

    \item \textbf{Algebra as God-given Reason.}  
    Islamic scholars regarded algebra (from al-Khwarizmi’s \emph{al-jabr}) as a gift of prophetic wisdom. Framing celestial arcs in algebraic functions (sines) rather than purely geometric lengths aligned the heavens with the divine Logos.

    \item \textbf{Ritual Precision.}  
    Exact prayer times and Qibla determinations demanded interpolation. Sine-tables supported systematic arithmetic methods, echoing the belief that God’s laws are both precise and accessible through reason.

    \item \textbf{Metaphor of the “Bosom.”}  
    In Arabic manuscripts, the unpointed \emph{jb} was read as \emph{jaib} (“bosom”). Later Latin translators rendered this as \emph{sinus}. The image of the circle’s point drawn “into the bosom” of its radius subtly evoked themes of spiritual intimacy with the Divine.
\end{itemize}

\medskip

Through the sine function, Al-Battānī didn’t just streamline computations—he wove together mathematics, theology, and ritual into a unified vision of the cosmos.
\end{HistoricalSidebar}


\subsection{Trigonometry in the Service of Faith: Calculating Prayer Times}

While Al-Battānī’s trigonometric innovations were groundbreaking in astronomy, they also served a profoundly practical and religious purpose: the precise calculation of \textbf{Islamic prayer times}.


In Islam, the five daily prayers (\textit{salāh}) must be performed at specific times tied to the position of the Sun:

\begin{itemize}
    \item \textbf{Fajr} (dawn): Begins at astronomical twilight, just before sunrise.
    \item \textbf{Dhuhr} (noon): When the Sun crosses the meridian and casts the shortest shadow.
    \item \textbf{Asr} (afternoon): When the shadow of an object equals or exceeds its length.
    \item \textbf{Maghrib} (sunset): Immediately after the Sun sets below the horizon.
    \item \textbf{Ishā'} (night): When the red twilight disappears and darkness sets in.
\end{itemize}

These timings depend on the Sun’s \textbf{altitude} and \textbf{azimuth}, which change with the seasons and with one’s geographical latitude. Al-Battānī’s use of \textbf{spherical trigonometry}—particularly sine and tangent functions—made it possible to compute these solar angles with unprecedented precision.

\begin{quote}
\textit{“Establish regular prayers at the sun's decline till the darkness of the night and the morning prayer.”} \\
\hfill — \textbf{Qur’an, Surah Al-Isra (17:78)}
\end{quote}

\subsubsection{Why Accuracy Mattered}

As Islam expanded from the Arabian Peninsula into vastly different latitudes—from North Africa to Central Asia—the variation in solar behavior across geography made fixed timekeeping insufficient. Trigonometry became a vital tool for standardizing observance across the Islamic world.

\begin{itemize}
    \item \textbf{Muwaqqits} (timekeepers) were appointed at major mosques to use astronomical instruments—often astrolabes and quadrants—to calculate prayer times.
    \item These experts relied on the trigonometric methods developed by Al-Battānī and others to interpret solar altitudes and shadow lengths in different regions.
\end{itemize}


\subsubsection*{How Timekeepers Used Trigonometry to Determine Prayer Times}

Let’s illustrate how a \textbf{muwaqqit}—a mosque-based timekeeper—might have used Al-Battānī’s trigonometric principles to calculate specific prayer times using the tools available in the medieval Islamic world.

\medskip

\textbf{Instruments used:}
\begin{itemize}
    \item A \textbf{gnomon} (vertical stick or column) to measure shadow length,
    \item An \textbf{astrolabe} or \textbf{quadrant} to measure solar altitude,
    \item Astronomical tables derived from Al-Battānī’s \textit{Zij} (astronomical handbook).
\end{itemize}

\medskip

\textbf{Example: Calculating \textit{Dhuhr} and \textit{Asr} in Damascus (Latitude: \( \phi = 33.5^\circ \))}

\begin{enumerate}
    \item \textbf{Find the Sun’s declination \( \delta \)} for the given day.\\
    The muwaqqit would consult Al-Battānī’s tables to determine the Sun’s declination—its angular position north or south of the celestial equator.

    \item \textbf{Calculate solar noon altitude \( h_{\text{noon}} \)} using spherical trigonometry:
    \[
    h_{\text{noon}} = 90^\circ - |\phi - \delta|
    \]
    This gives the Sun’s altitude when it is at the meridian (local noon).

    \item \textbf{Measure when the Sun reaches this maximum altitude.}\\
    Using an astrolabe or quadrant, the muwaqqit tracks the Sun's rising angle. When it reaches \( h_{\text{noon}} \), it is time for \textit{Dhuhr}.

    \item \textbf{Determine the start of \textit{Asr}} by calculating when the shadow length equals the object’s height (Hanafi: twice the height).\\
    This uses the basic tangent relationship:
    \[
    \tan(\theta) = \frac{\text{height}}{\text{shadow length}}
    \]
    Rearranged:
    \[
    \theta = \tan^{-1}\left(\frac{1}{L}\right)
    \]
    where \( L = 1 \) (standard length) or \( L = 2 \) (Hanafi tradition). The angle \( \theta \) gives the solar altitude at \textit{Asr}.

    \item \textbf{From that solar altitude, calculate the time when the Sun will reach \( \theta \).}\\
    Using trigonometric time-angle relations from Al-Battānī’s tables, the muwaqqit determines when the Sun will descend to that altitude in the afternoon.
\end{enumerate}

\begin{HistoricalSidebar}{From Heaven to Hour: The Birth of Algorithmic Thinking}
    Long before the word “algorithm” became the language of machines, it was the logic of the cosmos.

    \medskip

    The methodical procedure used by medieval Islamic timekeepers—starting with the Sun’s declination and ending with exact prayer times—was more than just a checklist. It was a structured, repeatable \textbf{algorithm}, centuries before the term entered modern computation.

    \medskip

    The very word \textit{algorithm} comes from the Latinization of the name \textbf{al-Khwārizmī}, the 9th-century Persian scholar whose treatises on arithmetic and algebra laid the groundwork for symbolic and procedural thinking.

    \medskip

    Timekeepers like those who followed Al-Battānī were not just observers of the sky—they were \textbf{early algorithmists}, applying fixed inputs (like latitude and solar declination) to generate consistent outputs (prayer times), following a logic-driven sequence of steps.

    \medskip

    \begin{itemize}
        \item Input: Geographic latitude \( \phi \), date, and solar declination \( \delta \)
        \item Rule: Compute \( h_{\text{noon}} = 90^\circ - |\phi - \delta| \)
        \item Trigger: Observe maximum solar altitude → \textit{Dhuhr}
        \item Condition: Shadow = object height → \( \theta = \tan^{-1}(1) \)
        \item Output: Time when solar altitude equals \( \theta \) → \textit{Asr}
    \end{itemize}

    \medskip

    \textbf{Key insight:} Before computer science, before code, there were prayer tables and quadrants. In the Islamic Golden Age, algorithmic thinking was already alive—driven not by data science, but by devotion.
\end{HistoricalSidebar}


\medskip

This cycle was repeated daily, adjusted for seasonal solar motion. Thanks to Al-Battānī’s tables and trigonometric formulations, religious rituals across the expanding Islamic world could be synchronized to the heavens with mathematical precision.


\subsection{The Muwaqqit’s Instruments and Calculation Procedures}

Muwaqqits relied on a small suite of observational tools and Al-Battānī’s trigonometric tables to turn raw 
measurements into exact prayer times:

\begin{itemize}
  \item \textbf{Gnomon}: A vertical rod of known height \(H\), used to cast a shadow of length \(L\).
  \item \textbf{Astrolabe or Quadrant}: To measure the Sun’s altitude \(h\) directly when needed (especially near sunrise or sunset).
  \item \textbf{Al-Battānī’s Zij}: Tables of \(\sin\), \(\cos\), \(\tan\), \(\sec\), etc., plus daily solar declination \(\delta\).
\end{itemize}

\textbf{Step-by-step procedure:}

\begin{enumerate}
  \item \textbf{Record local latitude} \(\phi\) and date.  Consult the Zij for the Sun’s declination \(\delta\) on that date.
  \item \textbf{Measure shadow length} \(L\) of the gnomon.  Compute the Sun’s altitude above the horizon via
    \[
      \tan(\alpha) \;=\;\frac{H}{L}
      \quad\Longrightarrow\quad
      \alpha \;=\;\tan^{-1}\!\Bigl(\tfrac{H}{L}\Bigr).
    \]
    Use the tangent table to find \(\alpha\).
  \item \textbf{Determine \textit{Dhuhr} (noon)} by computing the meridian altitude
    \[
      h_{\mathrm{noon}}
      = 90^\circ - \bigl|\phi - \delta\bigr|.
    \]
    When the measured \(\alpha\) matches \(h_{\mathrm{noon}}\), it is time for Dhuhr.
  \item \textbf{Compute \textit{Asr} onset} by solving for the altitude \(\theta\) at which an object’s shadow equals its height (or twice its height in the Ḥanafī school):
    \[
      \theta = \tan^{-1}\!\Bigl(\tfrac{H}{L_{\mathrm{target}}}\Bigr)
      \quad\text{with}\quad
      L_{\mathrm{target}} = H \text{ or } 2H.
    \]
    The corresponding hour angle from local noon is then looked up in the Zij, giving the Asr time.
  \item \textbf{Mark the prayer lines} on the sundial or astrolabe plate: draw the Dhuhr line along the north–south axis, then the Asr line at angle \(q\), where
    \[
      \tan q \;=\;\frac{\sin\Delta\lambda}{\sin\Delta\phi}
    \]
    from Al-Battānī’s qibla formula, ensuring each prayer face and azimuth is precisely set.
\end{enumerate}

By following these fixed, table-driven steps—measuring shadows, consulting sines and tangents, and interpolating hour angles—muwaqqits enacted what was effectively an early algorithm, day after day, across the Islamic world.  
