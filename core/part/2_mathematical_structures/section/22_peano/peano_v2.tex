\section{Peano and the Abstraction of Objects: From Operations to Axioms}

If Riemann showed that space itself could bend,  
and Gibbs gave physicists a way to calculate fields and flows,  
then Giuseppe Peano took a different turn:

He stepped back and asked:

\begin{quote}
“What is a vector, really? What is a number, a set, a point—if not an object defined by operations?”
\end{quote}

Where Cayley, Fourier, and Gibbs developed tools for manipulating mathematical entities,  
Peano questioned the very nature of those entities.

He realized that before we could calculate,  
before we could transform,  
before we could integrate, differentiate, or decompose—  
we had to define.

\bigskip

\subsection*{The Leap from Gibbs to Peano}

Gibbs had given physicists vectors as concrete tools in \( \mathbb{R}^3 \):  
arrows with direction and magnitude, manipulable through dot and cross products.

But Peano stripped away the coordinates, the arrows, even the geometric intuition.  
He asked:

✅ What rules must a set of objects satisfy to be considered a vector space?  
✅ Can we define vectors without ever drawing them?

In Peano’s hands, vectors became not arrows, not geometric objects,  
but elements of an abstract set \( V \), defined purely by **axioms:**

\begin{itemize}
  \item Vectors form an abelian group under addition.
  \item Scalar multiplication distributes over addition.
  \item There is a zero vector acting as an identity.
  \item Scalars come from a field (like \( \mathbb{R} \) or \( \mathbb{C} \)).
\end{itemize}

\bigskip

\begin{tcolorbox}[colback=gray!5!white, colframe=black, title=\textbf{Sidebar: The Shift from Gibbs to Peano}, fonttitle=\bfseries, arc=1.5mm, boxrule=0.4pt]

\begin{tabular}{>{\raggedright}p{4cm} >{\raggedright}p{5.5cm} >{\raggedright\arraybackslash}p{5.5cm}}
 & \textbf{Gibbs} & \textbf{Peano} \\
\midrule
View of vectors & Arrows in \( \mathbb{R}^3 \) representing physical quantities & Elements of an abstract set obeying axioms of vector spaces \\
Operations & Dot product, cross product, divergence, curl & Axioms of addition, scalar multiplication; no coordinates required \\
Goal & Practical calculus for physics & Foundational formalism for mathematics itself
\end{tabular}

\end{tcolorbox}

\bigskip

\subsection*{From Calculation to Structure}

Where Gibbs’s vector calculus enabled practical computation,  
Peano provided the scaffolding that told us what it meant to compute at all.

His work was part of a broader movement toward **axiomatization**—  
a drive to ground mathematics not in pictures, physical analogies, or intuition,  
but in **formal rules governing symbols and operations.**

Peano’s approach echoed a deeper philosophical shift:

✅ Geometry wasn’t a picture.  
✅ Algebra wasn’t a manipulation.  
✅ Mathematics was a system of symbols and rules.

\bigskip

In defining vector spaces axiomatically, Peano opened the door to:

— vectors in infinite dimensions,  
— vectors over abstract fields,  
— vectors with no underlying geometric representation at all.

He untethered vectors from space.

\bigskip

\begin{quote}
In Euler, we computed forces.  
In Lagrange, we minimized action.  
In Hamilton, we traced flows.  
In Jacobi, we found surfaces.  
In Cayley, we abstracted transformations.  
In Fourier, we decomposed vibrations.  
In Riemann, we curved the space.  
In Gibbs, we calculated fields.  
In Peano, we defined the space itself.
\end{quote}

\subsection*{A Foundation for Abstraction}

Peano’s leap wasn’t a move toward new computations or new equations.  
It was a move toward **formal clarity.**

He didn’t give physicists a new tool;  
he gave mathematicians a new foundation.

His axiomatization of vector spaces foreshadowed the later abstraction of groups, rings, fields, and modules in algebra.

And his formal language inspired the development of symbolic logic, predicate calculus, and ultimately the search for consistency in mathematics itself.

In the grand arc from forces to flows, from surfaces to spectra,  
Peano planted a different kind of flag:

Mathematics wasn’t merely a description of physical reality—  
it was a language capable of describing **any structure**,  
whether geometric, algebraic, or purely formal.

And on that foundation, a new mathematics of structure would rise:  
a mathematics ready to face paradox, infinity, and undecidability in the century to come.

