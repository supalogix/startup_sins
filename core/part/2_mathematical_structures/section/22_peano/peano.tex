\section{From Calculus to Axioms: Gibbs Built with Vectors, Peano Built Vector Spaces}

\subsection{From Concrete Arrows to Abstract Spaces: Peano’s Vector Axioms}

Gibbs had shown how vectors in \(\mathbb R^3\) serve as arrows of force, flux, and rotation.  But soon “vector‐like” objects began appearing in very different guises:

\begin{itemize}
  \item Continuous functions \(f:[0,1]\to\mathbb R\),
  \item Infinite sequences \((x_1,x_2,\dots)\),
  \item Matrices and linear transformations,
  \item Polynomial and power‐series coefficients,
  \item Data points in high‐dimensional statistics, and more.
\end{itemize}

None of these look like a geometric arrow, yet each one admits addition and scaling just like \(\vec A+\vec B\) or \(a\,\vec v\).  To capture this shared behavior, Peano introduced a minimal, coordinate‐free framework: the \emph{vector space axioms}.

\paragraph{An Artist’s Palette Analogy}  
Imagine an artist’s palette loaded with tubes of paint.  Each color—crimson, cobalt, viridian, and so on—is like an element in our collection \(V\).  When the artist:

\begin{itemize}
  \item \textbf{Mixes two paints} \(A\) and \(B\) (e.g.\ red + yellow = orange), the order doesn’t matter (\(A+B=B+A\)), and mixing three paints at once is independent of grouping \(\bigl((A+B)+C = A+(B+C)\bigr)\).
  \item \textbf{Uses a neutral medium} (clear glazing medium) that leaves any paint unchanged (\(A+0=A\)), and can “undo” a mix by adding the complementary shade (\(A+(-A)=0\), a neutral gray).
  \item \textbf{Lightens or darkens} a color by adding white or black: scaling the paint by a factor \(a\) (\(a\cdot A\)) distributes over mixing \(\bigl(a\,(A+B)=aA+aB\bigr)\), and combining white + black acts like ordinary scalar arithmetic.
\end{itemize}

No matter whether the artist uses oils, acrylics, or watercolors, the rules of mixing and scaling are the same.  This universal “mix-and-scale” behavior is what Peano captured in his vector space axioms:

\begin{enumerate}
  \item \((V,+)\) is an \emph{abelian group}: mixing is commutative and associative, there is a neutral medium \(0\), and each paint has an “inverse” complementary shade.
  \item \emph{Scalar multiplication} (lighten/darken) satisfies
    \[
      a\,(A+B)=aA+aB,\quad
      (a+b)A = aA+bA,\quad
      (ab)A = a\,(bA),\quad
      1\cdot A = A.
    \]
  \item Scalars come from a \emph{field} (like \(\mathbb R\) for proportions), so additive and multiplicative rules hold as expected.
\end{enumerate}

With these axioms in place, “vector‐like” objects—whether they are paints, sounds, functions, or matrices—inherit the same reliable structure that Gibbs’s arrows obeyed in \(\mathbb R^3\).  


\paragraph{Why Axioms?}

\begin{itemize}
  \item \emph{Coordinate‐free:}  Vectors need not be triples \((x,y,z)\); no axes or dimensions are built in.
  \item \emph{Universality:}  Equations like \(\vec F=m\vec a\), \(\vec r\times\vec v=\vec L\), or \(\nabla\!\cdot\vec F\) hold verbatim in every new setting.
  \item \emph{New arenas:}  From solution spaces of differential equations to feature vectors in machine learning, the same rules apply.
\end{itemize}

\paragraph{Peano’s Axioms for a Real Vector Space \(V\):}

\begin{enumerate}
  \item \((V,+)\) is an \emph{abelian group}:
    \begin{itemize}
      \item Closure: \(u+v\in V\).
      \item Commutativity: \(u+v=v+u\).
      \item Associativity: \((u+v)+w = u+(v+w)\).
      \item Zero: \(\exists\,0\in V\) with \(v+0=v\).
      \item Inverses: \(\forall v,\;\exists -v\) with \(v+(-v)=0\).
    \end{itemize}
  \item \emph{Scalar multiplication} by \(a,b\in\mathbb R\) satisfies
    \[
      a(u+v)=au+av,\quad
      (a+b)v = av+bv,\quad
      (ab)v = a(bv),\quad
      1\cdot v = v.
    \]
  \item Scalars form a \emph{field} (e.g.\ \(\mathbb R\) or \(\mathbb C\)), so \(a+b\) and \(ab\) behave as usual.
\end{enumerate}

\paragraph{Illustrative Examples}

\begin{itemize}
  \item \textbf{Function space:}  
    \(V = C([0,1])\) with \((f+g)(x)=f(x)+g(x)\), \((a\,f)(x)=a\,f(x)\).  
    The zero vector is \(f(x)\equiv0\), and \(-f\) is its pointwise negative.
  \item \textbf{Sequence space:}  
    \(V=\ell^2\) of square‐summable sequences, added and scaled entrywise, with \(\mathbf0=(0,0,\dots)\).
  \item \textbf{Matrix space:}  
    \(V=M_{2\times2}(\mathbb R)\), where matrices add and scale entrywise.  
    This is exactly the habitat of Gibbs’s linear operators.
\end{itemize}

By distilling vectors down to these few rules, Peano lifted Gibbs’s arrows into the universal realm of \emph{linear} objects—unlocking infinite‐dimensional analysis, abstract algebra, and far–reaching applications across mathematics and physics.  



\bigskip

\begin{tcolorbox}[colback=gray!5!white, colframe=black, title=\textbf{Sidebar: Gibbs vs. Peano — Action vs. Abstraction}, fonttitle=\bfseries, arc=1.5mm, boxrule=0.4pt]

\textbf{Gibbs:}  
Vectors are arrows in \( \mathbb{R}^3 \). Use them to encode rotation, flow, and force.

\textbf{Peano:}  
Vectors are elements of an abstract space \( V \), defined only by axioms of addition and scalar multiplication.

\medskip

\textbf{Gibbs:}  
Cross products and dot products are geometric operations with physical meaning.

\textbf{Peano:}  
Linear structure comes first. Geometry is optional.

\medskip

\textbf{Gibbs:}  
Invented tools for physics.

\textbf{Peano:}  
Invented the space those tools live in.
\end{tcolorbox}











\subsection{Kepler’s Second Law in the Language of Vector Spaces}

Up to now we have seen vectors as arrows, functions, or abstract elements of \(V\).  Let’s revisit Kepler’s Second Law—“equal areas in equal times”—and show how it emerges naturally from the structure of a two‐dimensional vector space.

\medskip
\noindent\textbf{1. The Orbital Plane as a Vector Space.}  
Let \(V=\mathbb R^2\) be the vector space of position vectors in the orbital plane.  A planet’s state at time \(t\) is given by  
\[
\mathbf r(t)\in V,
\]
and its velocity is  
\[
\mathbf v(t) \;=\;\frac{d\mathbf r}{dt}\;\in V.
\]

\medskip
\noindent\textbf{2. Area via an Alternating Bilinear Form.}  
In any 2-dimensional real vector space, there is (up to scale) a unique alternating bilinear map
\[
\omega:V\times V\;\longrightarrow\;\mathbb R,
\]
defined in coordinates \((x,y)\) by
\[
\omega\bigl((r_x,r_y),(v_x,v_y)\bigr)
= r_x\,v_y \;-\; r_y\,v_x.
\]
Geometrically, \(\omega(\mathbf r,\mathbf v)\) is twice the signed area of the parallelogram spanned by \(\mathbf r\) and \(\mathbf v\).

\medskip
\noindent\textbf{3. Areal Velocity as a Vector‐Space Pairing.}  
The instantaneous areal speed is
\[
\frac{dA}{dt}
= \frac12\,\omega\bigl(\mathbf r(t),\,\mathbf v(t)\bigr).
\]
Thus “equal areas in equal times” is equivalent to the statement
\[
\omega\bigl(\mathbf r(t),\,\mathbf v(t)\bigr)
=\text{constant}.
\]

\medskip
\noindent\textbf{4. Conservation from Central Forces.}  
Under a central force (gravity from a fixed focus), one shows via Newton’s laws that
\[
\frac{d}{dt}\,\bigl[m\,\omega(\mathbf r,\mathbf v)\bigr]
=0,
\]
i.e.\ the scalar \(m\,\omega(\mathbf r,\mathbf v)\) is conserved.  Here \(m\) is the planet’s mass and \(m\,\mathbf v\) its momentum, so this conserved quantity is exactly the angular momentum \(L\).  

\medskip
\noindent\textbf{5. A Vector‐Space Interpretation.}  
Viewed purely in the language of vector spaces:
\begin{itemize}
  \item \(\mathbf r(t),\mathbf v(t)\in V\).  
  \item Their wedge (or 2-form) \(\mathbf r\wedge\mathbf v\) lives in \(\Lambda^2(V)\), a one‐dimensional vector space over \(\mathbb R\).  
  \item Conservation of \(\mathbf r\wedge\mathbf v\) means the system’s trajectory preserves a fixed element of \(\Lambda^2(V)\).  
  \item As \(\Lambda^2(V)\cong\mathbb R\), this single scalar encapsulates Kepler’s law: the “area form” is invariant under the motion.
\end{itemize}

\noindent In this way, Kepler’s geometric insight—equal areas in equal times—is reinterpreted as the preservation of an element in a one‐dimensional vector space of 2-forms.  No coordinates, no diagrams, just the axioms of vector spaces and bilinear maps.  
