\subsection{From Passive Laws to Purposeful Action: The Control-Theoretic Lagrangian}

This is where Pontryagin’s legacy collides with Marx’s. The mathematics of control—adjoint systems, maximum principles, feedback loops—can be reinterpreted as not merely technical tools, but as philosophical instruments of agency. They mark a shift from describing the world to intervening in it, from measuring reality to bending it toward purpose.

To understand this shift, we must begin with the classical Lagrangian.

In classical mechanics, the Lagrangian \( L(x, \dot{x}, t) \) expresses the tension between kinetic and potential energy. A system’s path is determined by extremizing the action:
\[
S = \int_{t_0}^{t_1} L(x(t), \dot{x}(t), t) \, dt
\]
This leads to the Euler–Lagrange equation:
\[
\frac{d}{dt} \left( \frac{\partial L}{\partial \dot{x}} \right) = \frac{\partial L}{\partial x}
\]
Here, the path \( x(t) \) is entirely determined by initial conditions. The observer is passive, and the system evolves according to laws that are impartial, fixed, and timeless. Mathematics, in this view, is a descriptive language—a mirror to nature.

But Pontryagin reframes this structure—not to reflect what is, but to determine what ought to be. In \textbf{optimal control theory}, we introduce a control input \( u(t) \), and the system’s dynamics are no longer autonomous:
\[
\dot{x}(t) = f(x(t), u(t), t)
\]
The goal becomes choosing the trajectory by selecting \( u(t) \) to minimize a cost:
\[
J[u] = \int_{t_0}^{t_1} L(x(t), u(t), t) \, dt
\]
Here lies the key contrast:

\begin{itemize}
  \item In the classical case, the system evolves along a natural path, governed by internal symmetries. The Lagrangian reflects a world in equilibrium—history as consequence.
  \item In the control-theoretic case, the system is open to intervention. The Lagrangian becomes a utility function, and the trajectory is shaped not just by laws, but by choices—history as design.
\end{itemize}

\begin{quote}
Pontryagin transformed the Lagrangian from a passive descriptor to an active cost: one that encodes goals, constraints, and intent. The system is no longer merely observed; it is optimized, directed, and shaped.
\end{quote}

In this light, Pontryagin’s contribution is not merely technical. It expresses a worldview where mathematics is not just the study of nature but the architecture of change—a tool for turning knowledge into action. The observer is no longer a spectator, but a participant. And the system is no longer inert, but steerable.


This blending of dynamics with agency reflects the intellectual atmosphere of the mid-20th century—especially within the Soviet Union—where mathematics was not just the study of the world, but its operating manual. The observer is no longer a spectator, but a planner. The system is not inert, but actionable.

The neural network, in this light, is not merely a data-fitting apparatus. It is a dynamic system, optimized not just to reflect the world, but to act upon it. And perhaps, in some deeper sense, it reflects the evolution of theory itself—from passive witness to intentional agent.

To navigate this constrained optimization, Pontryagin introduced a dual vector \( p(t) \), the \emph{costate}, akin to canonical momentum. With it, he defined the \textbf{Pontryagin Hamiltonian}:
\[
H(x, u, p, t) = p^\top f(x, u, t) + L(x, u, t)
\]
This mirrors the classical transform \( H = p \dot{x} - L \), but with a critical difference: the velocity \( \dot{x} \) is no longer given—it is mediated by the control \( u(t) \), embedded within the dynamics.

The resulting structure, known as the \textbf{Pontryagin Maximum Principle}, elevates the variational logic of mechanics into the realm of control. It dictates that optimal behavior satisfies:

\begin{itemize}
  \item The \textbf{state equation}: \( \dot{x}(t) = \frac{\partial H}{\partial p} \)
  \item The \textbf{adjoint (costate) equation}: \( \dot{p}(t) = -\frac{\partial H}{\partial x} \)
  \item The \textbf{maximum condition}: \( u^*(t) = \arg\max_u H(x, u, p, t) \)
\end{itemize}

Unlike in classical mechanics, here the present is navigated through a lens of future consequence. The control \( u(t) \) is not just a mechanical force—it is a \emph{choice}, made under constraint, to achieve a goal. This shift—from description to direction, from inevitability to intention—mirrored a broader transformation in Soviet scientific thought: mathematics in the service of design, prediction, and centralized control.

\begin{quote}
The Pontryagin Hamiltonian is a geometry of action with political intention. Where Kolmogorov’s abstractions aimed to transcend the state, Pontryagin’s equations made the state's logic executable.
\end{quote}


\begin{tcolorbox}[title=Sidebar: From Marx to Machines — The End of History Becomes the Reign of Systems, colback=gray!10, colframe=black, fonttitle=\bfseries, breakable]

  If Marx imagined the ``\textbf{end of history}''\footnote{The "End of History" is a concept originally rooted in Hegelian and Marxist thought, where history progresses through dialectical conflict toward a final state of resolution—whether that’s communism for Marx or absolute spirit for Hegel. In modern times, the phrase was famously repurposed by Francis Fukuyama, who argued that liberal democracy marked the endpoint of ideological evolution.} as the resolution of contradiction, then 20\textsuperscript{th}-century technocrats reimagined it as something far more mechanical: the triumph of \textbf{systematic administration} over political struggle.
  
  In the Soviet Union, Marxist dialectics morphed into \textbf{cybernetic planning}. Control theory, once a mathematical tool, became an ideological instrument—modeling the economy as a dynamic system that could be steered, corrected, and eventually perfected. History, it was believed, could be brought under regulation like a missile trajectory or industrial assembly line.
  
  But this shift wasn’t limited to the East.
  
  In the West, particularly after World War II, thinkers like Norbert Wiener, Herbert Simon, and later Fukuyama framed history not as dialectic but as optimization. Liberal democracy + market capitalism = global equilibrium. The messy human drama of revolution, class conflict, and ideology could now be replaced with algorithms, forecasts, and feedback loops.
  
  \textit{Politics, rebranded as process.}
  
  What emerged was a vision of the future in which bureaucratic systems, AI agents, and policy models manage the world “post-ideologically.” Struggle isn’t resolved—it’s absorbed. Every contradiction becomes a dashboard metric.
  
  \textbf{From dialectics to data. From revolution to runtime. From class struggle to quarterly targets.}
  
  \textit{It’s not that history ended. It just got outsourced to the servers.}
  
\end{tcolorbox}



\begin{figure}[H]
\centering

% === First row ===
\begin{subfigure}[t]{0.45\textwidth}
\centering
\begin{tikzpicture}
  \comicpanel{0}{0}
    {Marx}
    {}
    {\footnotesize History moves through contradiction. But it ends in freedom.}
    {(0,-0.6)}
\end{tikzpicture}
\caption*{The premise: history as struggle.}
\end{subfigure}
\hfill
\begin{subfigure}[t]{0.45\textwidth}
\centering
\begin{tikzpicture}
  \comicpanel{0}{0}
    {Pontryagin}
    {}
    {\footnotesize I gave the system a goal. Now it converges.}
    {(0,-0.6)}
\end{tikzpicture}
\caption*{The reply: convergence through control.}
\end{subfigure}

\vspace{1em}

% === Second row ===
\begin{subfigure}[t]{0.45\textwidth}
\centering
\begin{tikzpicture}
  \comicpanel{0}{0}
    {Marx}
    {}
    {\footnotesize But whose goal? Whose equilibrium? A cost function is a class position.}
    {(0,-0.6)}
\end{tikzpicture}
\caption*{The critique: optimization isn’t neutrality.}
\end{subfigure}
\hfill
\begin{subfigure}[t]{0.45\textwidth}
\centering
\begin{tikzpicture}
  \comicpanel{0}{0}
    {Pontryagin}
    {}
    {\footnotesize Then let’s encode the future. Let the system optimize liberation.}
    {(0,-0.6)}
\end{tikzpicture}
\caption*{The proposal: control as historical closure.}
\end{subfigure}

\caption{Marx and Pontryagin debate whether optimal control theory marks the end of history—or just another form of it.}
\end{figure}


\begin{tcolorbox}[colback=gray!10, colframe=black, title={Sidebar: Why a Marxist Cosmology Needed a Cost Function}, fonttitle=\bfseries, breakable]

  For Newton, the universe unfolded under divine ordinance; for Lagrange, under rational harmony; for Einstein, under a subtle, impersonal order. Each cosmological vision embedded a metaphysics: the sky was not just geometry—it was theology, philosophy, and politics mapped onto the stars.

  \medskip
  
  For \textbf{Pontryagin}, committed Marxist-Leninist and mathematical revolutionary, the cosmos could not be a passive, predetermined machine. Dialectical materialism taught that history was driven by contradictions, struggle, and human agency—not by immutable, preordained laws. A universe without intervention would have been metaphysically incomplete.

  \medskip
  
  In this worldview, the existence of a \emph{cost function}—a formal representation of goals, priorities, and valuations—was not merely a technical convenience. It was a philosophical necessity.

  \medskip
  
  \begin{itemize}
    \item In classical mechanics, trajectories unfold by extremizing an action functional imposed by nature.
    \item In Pontryagin’s control theory, trajectories are shaped by an \emph{explicitly chosen cost}—reflecting human intention within material constraints.
    \item To a Marxist, this mirrors history: not a spontaneous equilibrium, but a process steered by class struggle, revolutionary vanguard, and political purpose.
  \end{itemize}

  \medskip
  
  A cosmology without a cost function would imply a universe without a steering mechanism—without agency, without the possibility of transformation. For Pontryagin, this would have been ideologically unacceptable. The \emph{cost function} was the mathematical embodiment of Leninism: a mechanism for embedding goals into the structure of evolution itself.

  \medskip
  
  In this light, Pontryagin’s Maximum Principle wasn’t merely a solution to an optimization problem—it was a dialectical inscription of agency into the very fabric of dynamical systems. The universe didn’t just obey laws. It could be guided toward a purpose.
  
\end{tcolorbox}
  