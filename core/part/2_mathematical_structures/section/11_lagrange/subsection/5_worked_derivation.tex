\subsection{Worked Derivation: Lagrange’s Method and Kepler’s Law}

Consider a dynamical system described by two generalized coordinates:
\[
q_1 = \text{radial coordinate}, \quad q_2 = \text{angular coordinate}
\]

Assume the kinetic energy has the form:
\[
T = \frac{1}{2} m \left( \left( \frac{dq_1}{dt} \right)^2 + a(q_1)^2 \left( \frac{dq_2}{dt} \right)^2 \right)
\]
and let the potential energy depend only on \( q_1 \): \( V = V(q_1) \). Then the Lagrangian is:
\[
L(q_1, q_2, \dot{q}_1, \dot{q}_2) = T - V = \frac{1}{2} m \left( \dot{q}_1^2 + a(q_1)^2 \dot{q}_2^2 \right) - V(q_1)
\]

Now apply Lagrange’s equation for each generalized coordinate:
\[
\frac{d}{dt} \left( \frac{dL}{d\dot{q}_i} \right) - \frac{dL}{dq_i} = 0
\]

For \( q_2 \):

Since \( q_2 \) does not appear in \( L \), we have:
\[
\frac{dL}{dq_2} = 0 \quad \Rightarrow \quad \frac{d}{dt} \left( \frac{dL}{d\dot{q}_2} \right) = 0
\]

Compute:
\[
\frac{dL}{d\dot{q}_2} = m a(q_1)^2 \dot{q}_2
\quad \Rightarrow \quad
\frac{d}{dt} \left( m a(q_1)^2 \dot{q}_2 \right) = 0
\]

Therefore:
\[
m a(q_1)^2 \dot{q}_2 = \text{constant}
\]

This expression is the conserved momentum conjugate to the angular coordinate \( q_2 \), and in planetary motion, it corresponds to **angular momentum**.

Now consider the infinitesimal area \( dA \) swept out in time \( dt \):

\[
dA \propto a(q_1)^2 \dot{q}_2 \, dt \quad \Rightarrow \quad \frac{dA}{dt} \propto a(q_1)^2 \dot{q}_2
\]

Thus:
\[
\frac{dA}{dt} = \text{constant}
\]

Which is precisely **Kepler’s Second Law**.

