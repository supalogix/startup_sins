\section{Al-Jazari: Mechanical Sequencing and Early Programming}

While Heron's inventions hinted at mechanical automation, it was \textbf{Al-Jazari} (1136--1206 AD) who advanced the idea of programmable mechanical sequences to a new level. Working in the Islamic Golden Age, Al-Jazari combined hydraulics, gears, and precise control mechanisms to create machines capable of performing complex, time-dependent operations.

\subsection{Programmable Automata and Dynamic Systems}

Al-Jazari's mechanical ingenuity was most evident in his construction of \textbf{automata}---self-operating machines designed to follow a preset sequence of actions. These included:

\begin{itemize}
    \item \textbf{Musical Automata}: Devices in which rotating drums with carefully placed pegs triggered mechanical musicians to play instruments in a timed sequence, an early form of mechanical ``programming.''
    \item \textbf{Cam-Driven Automata}: Machines that used camshafts to translate rotary motion into pre-designed movements, enabling more sophisticated sequences of actions.
    \item \textbf{Complex Water Clocks}: Elaborate timekeeping devices that dynamically adjusted displays according to the time of day and seasonal changes, integrating flow mechanics and geared control systems.
\end{itemize}

Unlike modern numerical computers, Al-Jazari's devices did not perform arithmetic calculations. Instead, they performed \textit{logical and time-based control} operations, executing predefined mechanical programs with precision. In this sense, his machines represent a true form of \textbf{mechanical computation over events and timing}.

\subsection{Legacy}

Al-Jazari demonstrated that mechanical systems could embody not only continuous motion but also discrete sequences of behavior. His automata, clocks, and hydraulic devices laid essential conceptual groundwork for later developments in mechanical computing, robotics, and control theory.

