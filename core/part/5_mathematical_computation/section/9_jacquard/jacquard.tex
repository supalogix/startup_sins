\section{Joseph-Marie Jacquard: The Programmable Loom}

Following the mechanical calculators of Pascal and Leibniz, a new idea began to take shape: machines could be \textit{programmed} to carry out complex sequences of operations automatically. This vision was realized by \textbf{Joseph-Marie Jacquard} (1752--1834), who applied it not to numbers, but to weaving.

\subsection{The Jacquard Loom: Mechanical Programming with Punch Cards}

In 1804, Jacquard introduced the \textbf{Jacquard Loom}, a revolutionary textile machine that could weave intricate patterns by following a sequence of instructions encoded on punch cards.

Key features included:

\begin{itemize}
    \item \textbf{Punch Cards}: Each card encoded one row of a weaving pattern, with holes corresponding to the threads to be raised.
    \item \textbf{Sequential Control}: The loom automatically advanced through the deck of cards, reading each row in sequence and adjusting the loom's operations accordingly.
    \item \textbf{Reprogrammability}: By changing the set of punch cards, entirely different patterns could be produced without modifying the loom's mechanical structure.
\end{itemize}

This was not just automation---it was \textit{programmable} automation: the behavior of the machine could be completely changed by altering the external program.

\subsection{Legacy}

The Jacquard Loom demonstrated for the first time that machines could execute complex operations based on externally supplied instructions. The use of punch cards to control machine behavior directly inspired later developments in mechanical computation, including the designs of Charles Babbage’s Analytical Engine and the punched card systems of early digital computers. Jacquard’s invention marked a decisive step toward the modern idea of a programmable machine.

