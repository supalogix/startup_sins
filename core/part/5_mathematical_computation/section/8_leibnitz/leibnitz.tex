\section{Gottfried Wilhelm Leibniz: Toward General Mechanical Computation}

Building on earlier advances, \textbf{Gottfried Wilhelm Leibniz} (1646--1716) expanded the vision of mechanical computation beyond basic arithmetic. He imagined machines that could perform a broad range of operations systematically---and began building devices to realize that vision.

\subsection{The Stepped Reckoner: Addition, Subtraction, Multiplication, and Division}

Leibniz designed and built the \textbf{Stepped Reckoner}, an ambitious mechanical calculator capable of:

\begin{itemize}
    \item \textbf{Addition and Subtraction}: Using rotating gears, much like Pascal’s Pascaline.
    \item \textbf{Multiplication and Division}: Implemented through repeated addition or subtraction, automated by mechanical steps rather than manual repetition.
\end{itemize}

The core innovation of the Stepped Reckoner was the \textbf{Leibniz Wheel}---a stepped cylindrical gear that could engage a variable number of teeth depending on its rotation, allowing different place values to be selected mechanically.

Although the Stepped Reckoner was prone to mechanical failures and difficult to manufacture precisely, it represented a conceptual leap toward general-purpose mechanical arithmetic.

\subsection{The Dream of a Universal Machine}

Beyond building specific devices, Leibniz articulated a broader vision: that reasoning itself could be mechanized. He imagined a future where symbolic logic and computation could be carried out by machines, describing a \textbf{``calculus ratiocinator''}---a universal logical calculus.

This philosophical leap foreshadowed later developments in formal logic, symbolic computation, and ultimately digital computers.

\subsection{Legacy}

Leibniz’s mechanical work bridged two worlds: practical devices for numerical calculation and theoretical foundations for symbolic reasoning. His inventions and his vision made him a critical ancestor of both the mechanical calculator and the modern computer.

