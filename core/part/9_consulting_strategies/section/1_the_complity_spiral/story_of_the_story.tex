\section{The Complicity Spiral: How to Make Everyone Dirty So No One Can Cleanly Leave}

\subsection{Horror Trope: Fake Relationship}

This story is similar to the Steven King's Carrie. There is something about the 
relationship that is not genuine. The power trope comes from knowing who has the
knowledge, what is the purpose of the lie, and how it will be revealed.


\subsubsection{Trope Synposis}

For some of us, starting our own business is hell; unfortunately, that is true for
David Morales, too. Business (\textbf{politics, workplace}) is one big, \textbf{forced proximity}
trope for David (\textbf{loner, tortured hero}) only gets more suffocating. David''s
shy and naive nature (\textbf{fish out of water}) makes him an easy target for 
Micheal and Serena (\textbf{antagonist, stalker}) when David get's his first big break.
Micheal and Serena (\textbf{suspects}) tormet his bewilerment (\textbf{victim}). 
David''s wife Emma (\textbf{protector}) tries to help David but inadvertently makes 
things worse.

Later, Emma''s desire to help her husband (\textbf{loner, fish out of water}) makes
her and easy target for Micheal and Serena (\textbf{antagonist, stalker}). Emma is 
at first suspicious of their help as she is a bit naive about the lifestyle 
(\textbf{secrets}).

Upon attending social gatherings, her \textbf{fake relationship} blossoms under 
Micheal and Serena's attention (\textbf{fish out of water}) and enjoys herself
(\textbf{red herring}). The ever present Serena (\textbf{mentor}) reassures Emma
about the world she wants to enter and new experiences she could enjoy.

After her first sexual encounter, she fully embrases her new identity 
(\textbf{fairy tale, ugly duckling}).

However, Micheal and Serena (\textbf{hidden identity})
are using her to manipulate David
(\textbf{the con}). When David (\textbf{man in peril}) get''s blamed for the
engineering failure (\textbf{stranded}), Micheal throws David under the bus
(\textbf{tortured hero, victim}).

In the aftermath, David has to deal with auditors and regulators (\textbf{road trip}).
He doesn't understand, that Micheal has rigged the situation (\textbf{the con}).
With David (\textbf{man in peril}) being the face of the system failure, everyone
involved (\textbf{red herring}) is incentivized to play along (\textbf{mistaken identity}).

In the end, Serena is with David but is no longer wants to be with him (\textbf{forced proximity}).
Micheal and Emma (\textbf{stalker}) have drawn Emma into their circle of influence 
(\textbf{victim}).

The extra fuel of the \textbf{fake relationship} is David's feeling of betrayal by Emma.




\subsection{Emotion Amplifiers}

\subsubsection{David Morales (Indecision)}

\paragraph{Description} A character can enter an uncomfortable state of indecision when they
must decide on a course of action, but they struggle to know which way to go.

\paragraph{Physical Signals and Behaviors}

\begin{itemize}
    \item Talking through with mentor
    \item Avoiding people who are waiting for the character's decision
    \item Writing down pros and cons
    \item Fact checking or researching options
\end{itemize}

\paragraph{Internal Sensations}

\begin{itemize}
    \item Being filled with nervous energy
    \item Signs of high blood pressure (i.e flushed skin, chest pains, shortness of breath)
    \item Having a panic attack (if the stakes are high and a choice seems impossible)
\end{itemize}

\paragraph{Mental Responses}

\begin{itemize}
    \item Confusion over what to do
    \item Mentally calculating the outcomes of specific choices
    \item Experiencing a flight response when the situation is broached
    \item Feeling threatend or pressured
    \item Being terrified of making te wrong decision
\end{itemize}

\paragraph{Efforts To Hide the Indecision}

\begin{itemize}
    \item Working hard to appear confident and self-assured so people won't lose faith
    \item Garnering sympathy in other areas
\end{itemize}


\paragraph{Associated Power Verbs}

\begin{itemize}
    \item Avert
    \item circumvent
    \item doubt
    \item dread
    \item elude
    \item fixate
    \item obsess
    \item overthink
    \item put off
    \item think
    \item second-guess
    \item promise
    \item regret
    \item wrestle
\end{itemize}

\paragraph{Emotions Generated By This Amplifier}

\begin{itemize}
    \item Anguish
    \item Anxiety
    \item Apprehension
    \item Conflicted
    \item Dread
    \item Insecurity
    \item Overwhelmed
    \item Worry
\end{itemize}

\paragraph{Duties Or Desires That May Be More Difficult To Fulfill}

\begin{itemize}
    \item Putting family first
    \item Trusting their gut in other situations
    \item Making other decisions
\end{itemize}

\paragraph{Scenarios For Building Conflict And Tension}

\begin{itemize}
    \item A hard deadline being set for the decision
    \item Suffering from a degenerative cognative condition that grows worse as time goes by
    \item Soliciting advice from an unreliable or untrustworthy person
    \item Knowing the right choice but facing temptation to do something else
\end{itemize}

\subsubsection{Emma Morales (Hypnotized)}

\paragraph{Description} Hypnosis is an altered state of consciousness that makes the subject
highly susceptible to suggestion.

\paragraph{Physical Signals and Behaviors}

\begin{itemize}
    \item Being compliant; agreeing with what the hypnotists says
    \item The character describing what they are seeing when they're asked to do so
    \item Calming down immediately when instructed or reassured by the hypnotist
    \item Changing behavior based on a pre-determined cue (a sound, word, sentence, or action)
    \item Reacting to hallucinatory sensory stimulation (behavior matching the emotional trigger)
\end{itemize}

\paragraph{Internal Sensations}

\begin{itemize}
    \item Foggy or tunnel vision
    \item A reduction of pain
    \item Feeling deeply relaxed
\end{itemize}


\paragraph{Mental Responses}

\begin{itemize}
    \item Resisting teh hypnosis (if the character is fearful)
    \item Trying to set aside anxiety or fear about the anxiety
    \item Feeling skeptical about it working
    \item Being open to suggestion (while retaining a level of awareness and control)
    \item Having intense focus
    \item Being unaware of the passage of time
    \item Being able to turn off or change emotions as instructed (i.e. the 
    character going from fearful to calm when the hypnotist reiterates they are safe)
\end{itemize}




\paragraph{Efforts to Resist The Hypnosis}



\begin{itemize}
    \item Not following instructions (to relax, listen to the speaker's voice, etc...)
    \item Focusing on things that will distract them from being pulled in
    \item Forcing the body to remain tense
    \item Using pain to stay alert (i.e. pinching themselves)
    \item Talking of being disruptive 
\end{itemize}




\paragraph{Emotions Generarted By This Amplifier}

\begin{itemize}
    \item Anticipation
    \item Doubt
    \item Skepticism
    \item Eagerness
\end{itemize}

\paragraph{Scenarios for Building Conflict and Tension}



\begin{itemize}
    \item Developing a confusing post-hypnotic reaction to something
    \item Realizing during the session that they are under hypnosis
    \item Seeing something untrustworthy
\end{itemize}