
\subsection{Relationship Triag}

In the days that followed, Emma felt... flatter. It was as if some part of her emotional 
range had been dimmed. It was still operational, but on low power mode. The panic was gone, but 
so was the 
spark. She no longer cried in the shower, but she didn’t laugh in the kitchen, either. And she 
wanted to.

The kitchen smelled like cinnamon and warm apples. David was wearing that old soft flannel she loved on 
him, sleeves rolled up, hair a little messier than usual. The kids were gathered around the island like 
an audience waiting for the next punchline.

``Okay,'' David said, flipping a pancake with theatrical flair, ``what do you call a dinosaur who farts 
in public?''

The older one groaned. ``Dad, no.''

David grinned. ``A blast from the past.''

The younger one exploded into giggles, slapping the counter and almost knocking over a mug of juice.
David caught it just in time, handed it back, and ruffled her hair with a gentle shake of his head.

``Pure chaos,'' he said. ``You two are pure chaos.''

Emma stood at the edge of the doorway, one hand resting on the frame. The coffee in her mug had gone cold. 
She hadn’t taken a sip.

The morning sun filtered through the window above the sink, casting soft gold across the floor. 
There was 
syrup on the counter, socks on the tiles, and a toy giraffe on the chair she usually sat in. The 
kind of mess 
she used to find charming. 

David flipped another pancake. The kids clapped. He winked at them and pretended to bow. It was 
all so familiar. 
It was so sweet. It was so full.

And she felt... nothing.

She felt like she was watching someone else’s home movie with the volume turned down. She watched 
the way David’s 
shoulders relaxed when he laughed, and the way the kids’ faces lit up with attention and sugar. 
The scene 
radiated spontaneous and unfiltered love.

She stood there, unmoved. She felt like a guest in her own life.

David caught her eye across the room, and offered her a soft smile. She smiled back, but it 
was the practiced 
kind of smile. It was the kind that doesn’t quite touch the eyes. David couldn’t see the 
hollowness from where 
he stood. She was getting good at that part.

The older one called out, ``Mom, you want one?''

Emma blinked. ``What?''

``A pancake!'' the little one repeated. ``We saved you the dinosaur-shaped one!''

David held it up with tongs. It vaguely resembled a stegosaurus with chocolate chip eyes.

Emma nodded. ``Thanks, sweetie.''

She walked over and sat down. The plate clinked gently as he placed it in front of her.

The syrup shimmered in the light.

Oliver beamed at her as she picked up her fork.

And still... nothing.

The parties continued.

The invitations still came. 

Emma still dressed up. 

Emma still smiled. 

Emma still let herself be passed gently from one curated room to another.

Emma still let herself be passed not so gently from one warm body to another. 

But inside her, something was missing.

It wasn’t guilt. That had unhooked.

It wasn’t shame. That had softened.

It was... intimacy.

She missed wanting David. 

She missed the flutter of early touches. 

She missed the stupid in-jokes.

She missed the way they used to get drunk on proximity instead of substances. 

She missed being kissed like she was a secret, and not a shared ritual.

And David, for all his quiet endurance, seemed to feel it too. His eyes lingered on her longer. 
His hands hesitated where they used to roam. They still had sex. But it felt like careful,
competent, and detached choreography. 

It was as if she was tired from too much pleasure.

\medskip

\begin{TechnicalSidebar}{Ketamine: Dissociation, Depression, and the Afterglow Gap}

  Ketamine, a dissociative anesthetic originally developed for medical use, has seen increasing 
  off-label application in the treatment of depression, PTSD, and chronic pain. It is also used 
  recreationally for its euphoric and hallucinogenic effects (Muetzelfeldt et al., 2008; Morgan et al., 2004), 
  particularly in the context of party culture and polysexual spaces (Santos et al., 2020).

  \medskip
  
  But the aftermath isn’t always so vivid.

  \medskip
  
  \textbf{Dissociative Aftermath.}  
  A key side effect of ketamine is depersonalization. Depersonalization is a sense of detachment from one’s self, body, 
  or surroundings. This can persist well after the psychoactive effects wear off, especially with 
  repeated use. Users often report feeling like a “ghost” in their own life, emotionally flattened, 
  or distanced from otherwise meaningful connections (Krystal et al., 1994; Pomarol-Clotet et al., 2006).

  \medskip
  
  \textbf{The Emotional Lag.}  
  While ketamine can provide a short-term spike in mood (and even temporary antidepressant effects 
  at clinical doses), some users experience a blunted affect in the days that follow. This "afterglow 
  gap" can feel like a muted emotional bandwidth where laughter, grief, and desire are all throttled 
  just below conscious access (Liebrenz et al., 2009; Wilkinson et al., 2017).
  
  \medskip

  \textbf{K-Cramps and Somatic Feedback.}  
  Chronic or high-dose ketamine use can result in abdominal pain and bladder dysfunction, colloquially 
  known as K-cramps or ketamine bladder syndrome. These physical symptoms often amplify the user's 
  awareness of disconnection between mind and body, reinforcing the very sense of alienation that 
  drew them to ketamine in the first place (Chu et al., 2008; Shahani et al., 2007).

  \medskip

  \textbf{Psychotic Episodes.}  
  Though rare, especially at therapeutic doses, ketamine can induce psychotic-like symptoms in some 
  individuals: paranoia, hallucinations, or distorted perceptions of time and agency. These effects 
  are more likely with high doses, frequent use, or a pre-existing vulnerability to mood or 
  psychotic disorders (Malhotra et al., 1997; Froese et al., 2020).
  
  \begin{quote}
  To lose guilt and shame is not always liberation. Sometimes it’s just signal loss.
  \end{quote}

\end{TechnicalSidebar}


\medskip

One night, at a midweek dinner with Mia, Emma said it out loud.

It was at a garden party 
with a low wall of rosemary and lavender hemmed the edges of the patio, and the air 
smelled faintly like citrus and heat.

Emma sat at the far end of the table with a half-finished glass of Pinot curled in her fingers. She 
wasn’t drunk, but the world had that softened filter. It was as if everything blurred just slightly 
at the edges. She’d been quiet most of the evening, letting the talk flow around her like background 
music. But then, as Mia reached for the serving dish, Emma spoke.

``Do you ever miss romance?'' she asked. ``Like... real romance?''

Mia paused, still holding the spoon. She turned her head slightly. The candlelight caught the 
angle of her cheekbone.

``Define real,'' she said, her tone even, almost amused.

Emma hesitated, tracing the rim of her glass with her fingertip. She didn’t want to sound naive. But 
the words came anyway.

``The kind that doesn’t need a theme,'' she said. ``Or a safe word. Just... two people. One look. 
Something that doesn't need a setup or a signal. Something simple.''

Mia, seated across from her and halfway through carving a slice of lamb, glanced up with a half-smile.

``You’re craving oxytocin,'' she said, sliding the blade through the meat. ``That’s not 
nostalgia. That’s neurochemistry.''

Emma blinked at him. ``I’m serious.''

She hadn’t meant for it to come out sharp, but there it was. It was a flicker of frustration 
breaking through 
the polish. She shifted in her chair and looked away, suddenly aware of the quiet that had 
fallen around her 
sentence. The others were pretending not to listen.

Mia didn’t laugh. Instead, she set the spoon down carefully and leaned back in her seat. 
Her expression had 
changed. It softened. However, it was also more focused. It was intent. She looked at Emma 
not with amusement now, 
but with something closer to recognition.

``I’m serious too,'' she said.

Mia reached into her clutch, as if producing a solution rather than a suggestion. The gesture 
was fluid, and  
practiced. From inside, she withdrew a tiny delicate and rose-tinted sachet. She held it between 
two fingers, 
not quite offering it yet. The candlelight gleamed off the gold-ink lettering on the edge.

``Have you ever done MDMA?'' she asked. ``Not at a party. I mean really done it. Just you and 
David. No noise. 
No games. Just closeness.''

Emma didn’t answer right away. Her gaze flicked to the sachet, then to Mia’s face. There was 
no pressure in 
her voice. There was no persuasion. It was just possibility. It laid out between the appetizers 
and the wine 
like the next course in a meal Emma hadn’t realized she’d already been eating.

Mia’s eyes didn’t leave hers.

``It won’t take you out of your body,'' she said. ``It’ll put you deeper into it.''

She placed it gently in Emma’s palm. It was heavier than expected.

``Don’t take them tonight,'' Mia said, voice low, almost parental. ``Wait until
you can hear yourself think.''

Emma looked down. The pouch was cinched with a silk cord, faintly scented like
citrus and vetiver. She didn’t open it. She didn’t need to.

Mia continued, her tone matter-of-fact, but with a reverence Emma hadn’t heard
before. ``No mirrors. They break the spell. Candles only. Use soft ones. None of
that hotel lighting.''

Emma nodded, still quiet.

``No phones. No playlist roulette. Pick something familiar. One album. Something
that feels like skin.''

David had stepped outside to call the valet. Through the glass, Emma could see
him checking his watch, then looking up, searching for her.

``Two glasses of water,'' Mia added, ``room temp. And just the two of you. No
friends dropping by. No surprises. Just... let it unfold.''

Emma closed her fingers around the pouch.

``They’re clean,'' Mia said, answering the question Emma hadn’t voiced. ``Tested.
Measured. Calibrated better than most marriages I know.''

She tucked a loose strand of hair behind her ear and added, almost offhand:

``I keep a stash for emergencies,'' she said. ``Usually for when one half of a
couple accidentally falls in love with me.''

Emma blinked. ``That happens often?''

Mia raised an eyebrow. ``More than you'd think. They invite me in for spice,
but forget that I'm not just the seasoning... I'm the catalyst.''

She smiled, but there was weight behind it. Her smile had a practiced detachment.

``And when that happens,'' Mia continued, ``I have them light a few candle, 
play Frank Ocean, and chemically reset their limbic system.''

She gently tapped the pouch in Emma’s hand.

``This isn’t drugs,'' she said. ``It’s relationship triage.''

Emma didn’t laugh, but her mouth softened. She looked down at the pouch again,
as if it now carried stories she hadn’t asked to inherit.

``Anyway,'' Mia said with a shrug. ``It's better than having awkward conversations 
at midnight. This way, everyone walks away calm, glowing, and pretending they were
never that attached to me.''

Just then a car cut a soft arc across the patio and parked. Mia turned toward it, and
then her expression shifted.
The edge in her shoulders softened. The mischief returned to her posture like it had 
been waiting offstage.

Emma saw it. It was the quiet way Mia slipped back into herself, or maybe out of herself
(depending on how you looked at it).

Mia stood, smoothed her dress with one hand, and reaching for her purse with the other.

Just as she pushed back her chair, she glanced down.

``Don’t fall in love with me,'' she said with a light voice. 

As she turned, something in her shifted from mischievous to unguarded. 
The playfulness gave way to something gentler --- almost sad --- like a truth she didn’t 
mean to admit out loud.

Her voice dropped. It was quieter, and not playful at all.
``I like you too much for that.''

Mia didn’t wait for a response. Her heels clicked against the patio stone as she walked 
toward the waiting car.

Emma didn’t say anything. She didn’t need to.
She understood exactly where Mia was going.



