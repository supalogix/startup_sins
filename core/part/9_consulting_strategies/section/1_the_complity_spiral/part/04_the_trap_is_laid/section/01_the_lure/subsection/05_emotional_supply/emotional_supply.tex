
\subsection{Emotional Supply}

Their relationship built slowly. It was like thread spun around a finger until the blood flow thinned.

At first, dinners were double dates. Then not. Serena started calling when she knew David 
was traveling.

``Just checking in,'' she'd say, as if friendship came with a calendar.

One night, over wine at a quiet tapas bar in Tribeca, Emma confessed:
``Sometimes I wonder if I’m disappearing. Like, piece by piece. And no one notices except the kids. 
And even they aren’t sure.''

Serena didn’t flinch. She reached across the table, took Emma’s hand gently, and said:
``That’s because no one trained you to want anything of your own. You’re still learning what 
shape you are. But I see you.''

It didn’t sound manipulative.
It sounded like grace.

And after that night, things shifted.

Not with declarations. Not with lines drawn or crossed.
But with soft permissions. The kind you don’t notice until you’ve already said yes.

Emma started answering Serena’s calls in the bathroom with the door locked.
Started picking outfits with Serena’s voice in her head.
Started saying “we” in sentences that had nothing to do with David.

And still, no one had kissed anyone.

But Emma was already disrobing in more vulnerable ways.

Emma stood near the balcony door, cradling her wine glass, the city lights blinking faintly 
through the sheer drapes.

Serena’s voice came from behind her, low and deliberate.

“Do you always flinch when someone looks too long?”

Emma turned. “I didn’t flinch.”

Serena stepped closer. “You shifted. It’s different.”

Emma looked down at her glass. “Maybe I’m just not used to attention.”

Serena reached gently, touched the necklace at Emma’s collarbone, and let her finger trace the 
chain — not the pendant.

“That’s not true,” she said. “You’re just not used to the kind that lingers.”

Emma didn’t pull away. She didn’t breathe.

Mia’s voice floated in from the couch, teasing. “Has she told you what she notices first? It’s 
never what you expect.”

Serena tilted her head. “With you? It’s the way you wait before answering. Like you’re still 
checking if your truth is allowed.”

Emma blinked.

“It is,” Serena added. “You’re allowed.”

Emma’s voice was barely audible. “No one ever said that out loud before.”

Mia sat up slightly, watching now. “Do you want someone to?”

Emma hesitated.

“I want someone to mean it.”

Serena smiled, then walked around behind her --- slow, deliberate --- and brushed Emma’s hair 
back, exposing the nape of her neck.

“I meant it the moment you let me touch your silence.”

Emma exhaled. She hadn’t realized she was holding her breath.

Mia whispered from across the room. “Careful, love. That’s the kind of sentence that gets under the skin.”

Serena leaned close, breath warm near Emma’s ear. “That’s the point.”

And Emma, who had always been careful about boundaries,
didn’t notice the thread
until it was already woven through her spine.

Emma started to crave her. 

Emma did not crave Serena sexually. At least, not at first. 

Emma craved her emotionally.

Emma craved her chemically.

It was like Serena was some controlled substance no one warned her about.

It was like there was something in the way she spoke that calmed the hum in Emma’s head.

It was like every conversation left a faint afterglow she kept trying to recreate.

It wasn’t friendship anymore.

It was dosage.

\medskip 

\begin{PhilosophicalSidebar}{Co-Dependency and the Chemistry of Attachment}

    In addiction recovery circles, there’s an old piece of advice:  
    \textbf{No new relationships for the first year.}

    \medskip
    
    The advice is not moral obligation. 

    And the advice is not religious dogma.  

    \medskip

    \begin{quote}
        \centering
        \textbf{It is chemical detox.}
    \end{quote}

    \medskip
    
    Because addicts don’t just get high on substances.  

    \medskip

    \begin{itemize}
        \item They get high on people.  
        \item They get high on the thrill of being needed.  
        \item They get high on the dopamine of being seen.  
        \item They get high on the illusion that someone else can complete them.
    \end{itemize}
    
    \medskip
    
    This is the foundation of \textbf{co-dependency}.  

    \medskip

    \begin{itemize}
        \item It is a pattern where the self becomes fused with the presence, approval, or emotions of another.  
        \item It is a pattern where boundaries dissolve in the name of closeness.  
        \item It is a pattern where affection becomes currency, and attention becomes the drug.
    \end{itemize}
    
    \medskip

    That why programs like 
    \textit{Co-Dependents Anonymous (CoDA)} and
    \textit{Sex and Love Addicts Anonymous (SLAA)}
    are abstinence-based.
    It is not out of piety, but because of data.  
    They treat emotional dependency the same way other AA type programs treat narcotics. 
    Emotional dependency for them is a compulsion with withdrawal symptoms, relapse cycles, 
    and triggers that hijack the brain.

    \medskip
    
    In CoDA, one of the core teachings is brutally simple:  
    \textbf{If taking someone away from you feels like withdrawal, it's not love. It's dependency.}

    \medskip
   
    In SLAA, the equivelent teaching cuts bone deep:
    \textbf{Love is something that demands you sacrifice for someone. 
    Co-dependency is something that demands someone sacrifice for you.}
    
    \medskip
    
    The line between the two can blur.  
    But the test is simple:  
    If their absence feels like detox, then the connection wasn’t clean.

    \medskip

    
\end{PhilosophicalSidebar}


\subsection*{Editor Questions for ``Emotional Supply''}

This section pivots into the psychological tension between emotional intimacy and dependency, dramatizing a slow escalation through gesture, suggestion, and neurochemical need. The following questions probe character realism, narrative pacing, and the ethics of influence.

\subsubsection*{Character Dynamics}

\begin{itemize}
  \item Does Emma’s shift from conversation to emotional craving feel gradual and believable?
  \item Is Serena’s behavior ambiguous enough to leave open the question of intention: nurturing, manipulative, or both?
  \item Does Mia’s inclusion deepen the triangulation or muddy the emotional focus?
  \item Is the relationship between Serena and Emma more compelling as tension, or does it risk tipping prematurely into overstatement?
\end{itemize}

\subsubsection*{Voice \& Tone}

\begin{itemize}
  \item Is the tone appropriately restrained given the emotional stakes, or does it risk melodrama in parts?
  \item Are lines like ``I meant it the moment you let me touch your silence'' effective in their poetic gravity, or too polished to feel authentic?
  \item Should Mia’s dialogue maintain its flirtatious detachment, or reflect more curiosity or unease?
\end{itemize}

\subsubsection*{Narrative Flow}

\begin{itemize}
  \item Does the pacing of the scene (from tapas bar confession to balcony intimacy) build tension effectively?
  \item Is there sufficient contrast between public and private settings to heighten vulnerability?
  \item Does the final paragraph (\textit{“It wasn’t friendship anymore. It was dosage.”}) land as an emotional climax or feel overly diagnostic?
\end{itemize}

\subsubsection*{Psychological Realism}

\begin{itemize}
  \item Are the markers of emotional dependency (e.g., calling with door locked, using “we,” craving afterglow) grounded and familiar enough to resonate?
  \item Do the characters’ gestures (touching the chain, tracing silence) align with realistic intimacy, or do they verge on symbolic excess?
  \item Is Emma’s emotional shift given enough internal justification to make her craving credible?
\end{itemize}

\subsubsection*{Philosophical Sidebar}

\begin{itemize}
  \item Does the sidebar on co-dependency reinforce the narrative or compete with it for emotional attention?
  \item Are recovery concepts like SLAA and CoDA effectively introduced for lay readers, or do they need context or clarification?
  \item Does the quote formatting within the sidebar aid clarity and rhythm, or distract from the message?
\end{itemize}

\subsubsection*{Thematic Framing}

\begin{itemize}
  \item Does the central metaphor of ``dosage'' work as a bridge between intimacy and addiction?
  \item Is the exploration of craving nuanced enough to invite reader empathy without moralizing?
  \item Is the line between love and dependency explored with complexity, or could it benefit from more ambiguity?
\end{itemize}

\subsubsection*{Structural Questions}

\begin{itemize}
  \item Should this section be split between the tapas scene and the balcony for greater dramatic impact?
  \item Would it help to give Emma more internal narration during the scene — or is the restraint intentional?
  \item Should Mia return later to triangulate or reinforce this moment of threshold crossing?
\end{itemize}

