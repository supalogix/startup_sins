\subsection{The Old College Try}

That night, David couldn’t sleep.

Not because of fear. Not exactly.

Fear would’ve been clean. Sharp. Something he could parse like a bug report, isolate like a stack overflow.

This was different.

This was the soft, rhythmic throb of too many truths colliding.

He lay in bed beside Emma, watching the soft rise and fall of her breath. She had curled toward the window, 
one hand tucked under her cheek. At peace.

He wasn’t.

Because the model wasn’t ready. Not really. And he knew it in the same way a pilot knows when the 
weather is wrong.  
No alarms. Just instinct. Muscle memory. A sense that something small had been missed... and that 
small was all it took.

He could still veto it.

That had been the agreement.

Hart had framed him as the conscience of the model. He framed him as the final brake on acceleration.  
David was the failsafe built into the social contract.

But that was the theory.  
In practice, the launch had already left the station. Hart had done what Hart did best:  
He’d built consensus before the conversation even reached the room.

And David could feel it.

The smiles. The nods. The way people said his name like it already belonged to their success story.

He wasn’t just the engineer anymore. He was the headline.

Kessler had called him ``mission critical'' during the last check-in.  
Michael had slipped a compliment into an investor dinner: ``David’s not just technical. He’s principled.''  
Paolo had thanked him for the architecture memo, saying it ``showed judgment.''

Even Serena had said it: ``He’s what makes this real.''

And Emma?

Emma had laughed the loudest at the terrace jokes. She had introduced him to Serena’s circle with a 
kind of pride he hadn’t seen in years.  
She had become someone else — not in a bad way. Just... more. More confident. More magnetic. More at 
home in a world that wasn’t his.

And that was the hard part.

It wasn’t that he didn’t belong.

It was that he did, but only through her.

The thought came uninvited:  
\textit{This isn’t her orbiting me anymore. I’m orbiting her.}

He wasn’t angry. He was proud. But also... displaced.

The more Emma flourished under Serena’s mentorship --- the dinners, the panels, and the late-night texts that 
ended in laughter  --- 
the less David could pretend he was the one pulling her forward. If anything, he was the one being pulled now.

And if he pulled the brake?

If he tanked the deal?

He wouldn’t just disappoint Hart.  
He wouldn’t just cost Kessler political capital.  
He wouldn’t just stall the launch.

He would embarrass Emma.  
Undermine her new status.  
Maybe even make her choose.

He stared at the ceiling and thought about the old ethics lectures at CalTech.  
The ones about risk tolerance and design responsibility.  
The “prime directive” of engineering: \textit{You are responsible for what you build.}

But what if you didn’t build it?

What if you were just... upstream of it?

What if your name was on the top line of a flowchart, but the logic had already been shaped in meetings 
you didn’t attend?

Was that still your responsibility?

He could still say no.  
He could still be the person who pulled the cord and took the hit.

But the question he couldn’t shake was:  
\textit{Who takes the hit with me?}

Because it wouldn’t be Hart. Hart had a dozen lifeboats and a publicist.  
It wouldn’t be Kessler. Kessler would make it look like prudence.  
And it wouldn’t be Emma. Emma had just found her footing. Her momentum. Her voice.

No.

The only one who’d look like he blinked?

Was David.

He exhaled, slowly.  
And thought about something his old advisor once said during office hours.

\begin{quote}
  Sometimes you don’t get the luxury of being pure.  
  Sometimes all you get... is the chance to be careful.
\end{quote}

David sat up, padded to the kitchen, and poured himself a finger of whiskey.

He didn’t toast.

He just took a breath.

And whispered to no one:

\textit{``Alright. Let’s give it the old college try.''}

\medskip

\begin{PsychologicalSidebar}{Compartmentalization --- The Ethics of Splitting the Self}

What David is experiencing is a classic case of \textbf{compartmentalization}: a defense mechanism 
where conflicting values, roles, or emotions are mentally isolated to reduce internal conflict. It 
allows a person to function under pressure without confronting contradictions all at once.

\medskip

Psychiatrist \textbf{George E. Vaillant}, in his work \textit{Adaptation to Life} (1977), categorized 
compartmentalization as a \textbf{neurotic-level} defense. They are more mature than denial or distortion, 
but still risky when it prevents ethical integration. It's common in 
high-performing individuals juggling incompatible demands: the principled engineer, the loyal partner, 
and the institutional actor.

\medskip

Empirical support comes from the study “The Reality of Recovered Memories: Corroborating Continuous, 
Recovered, and Fabricated Memories of Childhood Sexual Abuse”. The study showed that people could 
recall traumatic experiences in therapeutic or emotionally intense contexts while remaining unaware 
of them in everyday life. The brain, under stress, can firewall memory by preserving function at 
the cost of coherence.

\medskip

David isn't being dishonest.

\medskip

He's splitting — ethically, emotionally, and professionally.

\medskip

    \begin{itemize}
        \item He knows the model isn’t safe.
        \item He knows the launch is already politically in motion.
        \item He knows stopping it could hurt Emma’s standing — and cost him hers.
    \end{itemize}

\medskip

Rather than reconcile these, David lives each truth in isolation, in its proper setting. He’s the safety 
brake in theory, the team player in meetings, and the protector beside his wife. Each role intact. None 
fully integrated.

\medskip

As \textbf{Albert Bandura} later warned in his research on moral disengagement, compartmentalization 
in hierarchical systems enables people to cause harm without feeling culpable because responsibility has 
been quietly distributed across silos.

\medskip

The danger isn’t malice.

\medskip

It’s insulation.

\medskip

And in complex systems --- technical, financial, or personal --- insulation can be indistinguishable from 
complicity.

\end{PsychologicalSidebar}
