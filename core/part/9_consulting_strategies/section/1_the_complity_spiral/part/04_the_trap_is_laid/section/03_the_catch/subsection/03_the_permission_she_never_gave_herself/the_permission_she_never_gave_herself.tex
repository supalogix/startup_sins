
\subsection{The Permission She Never Gave Herself}

Serena told Emma, “You ever try glass?” when Emma confessed, in a quiet moment between 
drinks and glances,
“I feel like I’m supposed to be... more fun. I’m just not there, sometimes.”

Serena didn’t push. She never pushed. She simply offered—like she was handing over a 
missing part Emma 
didn’t know to ask for.  And Emma, already floating on the warm edges of the evening, 
said yes.

The crystal pipe had felt almost ornamental in her delicate, curated, and safe hand.

She inhaled... held... then released.

And suddenly, she was fluent in her own body.

The shift was subtle at first. 

There was a warmth behind her eyes. 

There was a loosening in her spine. 

But by the time the third song had shifted into something bass-heavy and tribal, she 
wasn’t just performing. 
She was present.

There were hands. 

There were lips. 

There were voices that moaned her name without needing to know who she was.

She was lying across the Moroccan rug Serena always said was vintage. 

Her thighs were pressed against someone’s hip. 

Her mouth was filled with someone's dick.

At one point, David was behind her. At another, beside her. Maybe 
both. 

Time didn’t matter. 

Identity didn’t matter. 

There were just limbs... and heat... and sensation... and bodies folding 
into bodies like chords resolving into harmony.

And she had led it.

She had been the one pulling Serena to the floor. 

She had been the one guiding Mia’s mouth.

She had been the one whispering to David when to watch and when to touch.

The next morning, she woke up wrapped in Egyptian cotton sheets and someone else’s 
perfume. 

Her thighs ached. But they ached in a strange satisfied way. 

There was glitter on her collarbone and a faint, unfamiliar 
bruise along her hip. 

She traced it with her fingers and waited for the shame to arrive.

But it didn’t.

Instead, what came was confusion. 

What came was a trembling dissonance she couldn’t quite name.

She sat up, pulled on David’s shirt, and wandered to the bathroom mirror.

``Who was that?'' she asked herself

``That’s not me.'' she told herself as she looked in the mirror.

But even as the thought surfaced, another pushed up behind it.

``Maybe it is you. Maybe that is you... untethered.'' a voice told her in her head.

``Maybe that’s you,'' the voice said again... quieter this time, but closer. 
``Not the version you pretend to be. The one you buried.''

Emma pressed her palms against her temples. 

The air felt thick. 

The air felt like it was pressing back.

``But I didn’t want that,'' she whispered aloud, but she wasn’t sure to who she 
was talking to.

Another part of her --- colder, and more composed —-- replied without speaking.
``You didn’t stop it either.''

She caught her reflection in the window. For a split second, it didn’t feel like 
her. The angle was 
wrong. The eyes were watching her --- not from within --- but across some invisible 
pane.

She touched her face. 

It was hers. 

And yet... not hers.

Inside her head, it no longer felt like a single voice narrating her choices. 

Her head felt like a committee. 

One part horrified. 

One part curious. 

One part proud.

One part... absent.

``Maybe it wasn’t me,'' she tried again.

And the voice --- or was it just her, echoing back? --- flatly answered.
``Then why did it feel so natural?''

She ran cold water over her wrists and whispered aloud:
``I just don’t know how to be her without the glass.''

There was no judgment in her voice. 

There was just observation. 

It was like a scientist logging a result she didn’t expect.

Because somewhere in that tangle of limbs and breath and music, she had found a 
version of herself that 
felt... real.

This version of her was not performative. 

This version of her was not subdued. 

This verion of her was just finally, and fully allowed.

It wasn’t about the sex.

It was about slipping the leash.

It was about meeting the woman buried under ten layers of apology and second-guessing.

Ten layers of permission she had never given herself.

\begin{TechnicalSidebar}{\textbf{State-Dependent Disinhibition — When the Filters Fall Away}}

  What Emma experienced wasn’t fabrication. It was amplification.

  \medskip
  
  In psychology, this is referred to as state-dependent disinhibition — a phenomenon in which 
  certain emotional or behavioral filters (typically regulated by the prefrontal cortex) are 
  dialed down under the influence of specific substances. These filters are normally responsible 
  for suppressing impulses related to fear, shame, inhibition, and social control.

  \medskip
  
  Methamphetamine (or “glass”) in particular doesn’t create new desires. It doesn’t implant alien 
  scripts. Instead, it increases the availability of dopamine and norepinephrine in the brain — 
  heightening arousal, confidence, and a sense of invincibility, while weakening the brain’s usual 
  checks and balances.
  
  \medskip
  
  \textbf{The result?}
  Desires that were already present — perhaps buried under cultural conditioning, self-judgment, 
  or emotional trauma — become actionable. Not because they’re new. But because the internal 
  brakes are lifted.
  
  \begin{quote}
  It’s not that you become someone else.
  It’s that you act without negotiating with yourself first.
  \end{quote}
  
  Clinical studies on methamphetamine and other stimulants confirm this pattern. For example, in 
  the Subjective Experience of Meth Sex (SEMS) study (Semple et al., 2004), the majority of 
  participants described the drug as removing inhibition, increasing confidence, and dissolving 
  shame. Behaviors they once viewed as off-limits became accessible — not because their values 
  changed, but because the emotional weight behind those values was temporarily anesthetized.
  
  \medskip
  
  This isn’t limited to meth.
  MDMA lowers fear and enhances empathy.
  Ketamine detaches thought from emotion.
  Alcohol narrows attention and blurs consequence.

  \medskip
  
  Each one tweaks the threshold for action in its own way.
  
  \medskip

  What matters — clinically and ethically — is understanding that what happens in those altered 
  states isn’t always a distortion. Often, it’s a disclosure. A glimpse of a self that lives beneath 
  the filters. One that may not always feel safe to acknowledge in sober daylight.

  \medskip
  
  That’s why post-experience confusion is so common. The actions were real. The feelings were real. 
  The filters, however, were temporarily offline.

  \medskip
  
  And when they return?

  \medskip
  
  So does the reckoning.
  
\end{TechnicalSidebar}
