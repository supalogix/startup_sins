
\subsection{The Threshold and the Fall}

The weeks that followed blurred.

Emma had never felt more alive.

She also had never felt more fragmented. 

The thrill that glass had given her --- the shocking 
boldness, and the sudden authority in her own skin --- didn’t last.  

It flickered. 

Then faded. 

And what remained was something quieter. 

What remained was something sharper.

What remained was... guilt.

It was not for what she’d done. At least, not specifically. 

The nights were hard to piece together. 
It was more ambient than that. 

The guilt was a knowing. 

The guilt was a knowing that she'd crossed some invisible threshold she didn’t 
remember consenting to. 

The guilt was a recognition that she had become, somehow, a stranger to herself.

The confidence was gone.

The afterglow curdled into anxiety. 

The afterglow turned into a tightening sense that she was 
out of phase with herself. 

Every moment took more effort. 

She second-guessed what she wore, what she said, and how people looked at her. 

She started skiping meals, then binging at night. 

She started crying in the shower. 

She started avoiding her reflection. 

She started to pick at her skin. 

\medskip

\begin{PsychologicalSidebar}{The Body as a Battleground --- 
  Dissociation, Skin, and the Female Experience of Trauma}
  
  For many women, trauma does not just fracture the mind.  
  It rewrites their relationship to the body.  
  The skin becomes both the stage and the script.  
  Control becomes the only form of safety.
  
  \medskip
  
  As Judith Herman (1992) noted in her foundational work \textit{Trauma and Recovery}, 
  female trauma often manifests as internalized shame.
  It is not just over what happened, but over how it changed them.  
  The self becomes foreign.  
  The body becomes suspect.
  
  \medskip
  
  \textbf{Dissociation}, a common response to chronic or early trauma, 
  allows the mind to survive what the body cannot escape (Putnam, 1997).  
  Memories go hazy. Emotions detach.  
  But the cost is high.
  In the absence of connection, women often turn to control.
  It is not control of circumstances, but of flesh.
  
  \medskip
  
  \textbf{Cutting} and \textbf{skin-picking}, while different in intensity,  
  serve a similar psychological purpose.  
  It is to make pain visible, local, and manageable.  
  They anchor. They focus.  
  They offer a perverse relief
  because in a world where boundaries have been violated,  
  self-inflicted pain is the only kind that obeys.
  
  \medskip
  
  Neurobiological studies support this.  
  Schmahl et al.\ (2006) found that women with trauma histories 
  who engaged in cutting showed increased pain thresholds and reduced 
  amygdala activity. Yet, they reported feeling emotionally calmed.  
  Similarly, Snorrason et al.\ (2012) documented how women with compulsive 
  skin-picking behaviors described it as a response to guilt, boredom, and self-disgust
  which is often rooted in trauma, perfectionism, or emotional overload.
  
  \medskip
  
  Both behaviors are reinforced not because they feel good,
  but because they make the pain \textbf{predictable}.
  
  \begin{quote}
  \itshape
  When I cut, the pain is clean.  
  When I pick, I don’t have to feel the rest of me unraveling.  
  It’s just this one thing. Just this one place.
  \end{quote}
  
  \medskip
  
  In many female-coded trauma responses, the logic is not destruction.  
  It’s containment.  
  It’s making a private pain legible in the only language the body remembers.  
  And for a woman trained to see her body as something sacred, watched, evaluated, or 
  trespassed,  sometimes the only way to reclaim it... is to hurt it herself.
  
\end{PsychologicalSidebar}

\medskip

David noticed how her behavior changed. He didn’t say anything at first. He wasn’t sure how. 
And maybe he didn’t want to know. 

But the weight never left her. Even when she smiled. Especially when she smiled.

Serena found her near the edge of the terrace wall, just beyond the reach of the patio lights. 

``You okay?'' she asked, the words gentle but weighted, like she already knew the answer.

Emma let out a breath that wasn’t quite a sigh. Her fingers twitched slightly, then stilled again on her 
shin. She didn’t meet Serena’s gaze.

``I don’t know how to carry all of this,'' she said, her voice thin and uneven. ``I feel like I’m 
unraveling.''

Serena didn’t respond immediately. She let the quiet stretch between them, let it settle like a shared 
blanket. Then she leaned forward, resting her forearms on her knees, her bracelets catching the light.

``I know that feeling,'' she said softly. ``When the shame and the desire fight for the same space inside 
you. When you’re both the experiment... and the evidence.''

Emma’s head turned slightly, just enough for Serena to see the tension in her jaw. 

Her eyes were glassy. 

Her eyes were not quite crying, but they were close. 

Here eyes were like tears that were backing up somewhere behind the words she couldn’t 
say. 

``It’s not even the sex,'' she said. Her voice cracked a little. ``It’s the thinking afterward. The 
who-was-that. The what-the-fucks. The...'' Her shoulders lifted in a helpless shrug. 
``The what-did-I-just-do?''

The words hung there --- fragile and suspended --- like a spider’s thread caught in the wind. Serena didn’t 
rush to answer. She reached out with one hand resting lightly on Emma’s back. Serena did not do it 
to pull her closer. Serena did it to remind her she was real.

And for a moment, neither of them moved. They just sat in the half-light of someone else’s garden, 
two women suspended between the selves they remembered and the ones they hadn’t yet named.

And here’s what Emma didn’t say.

Not to Serena.
Not to David.
Not even to herself in words she could fully hear.

Emma was afraid. 

Emma was not afraid of what she’d done.

Emma was afraid of who might one day know.

Emma still went to church.

Emma still brought the kids to Sunday school with clean fingernails and a neutral lipstick.

Emma still kept her hair neat, her smile soft, and her shoulders slightly angled when standing in group 
photos. She kept herself the way you do when you’ve been trained to present ``put together.''

However, under the presentation was something hollow and loud.

Emma wasn’t afraid of guilt. Guilt she could live with.

She was afraid of being a hypocrite.

She was afraid of the dissonance between the woman she performed and the woman she unleashed.

She was scared her children would find out.
That one day they’d hear too much. That they'd ask a question she wouldn't be ready 
for, and that she'd break trying to answer.

She was scared her parents would find out.
That her mother would look at her with that bone-deep disappointment that doesn’t need to be spoken to 
land like a verdict.
That her father would go quiet.

She was scared the people at church would find out.

She was scared that her name would end up on some list of people who start praying for her ``out of concern.''

But mostly --- and this was the shame-shaped knot at the center of her ---
she was scared that none of it could be undone.

That the woman she’d become in Serena’s world wasn’t a detour.

It was her.

Emma was afraid that her soul wasn’t clean anymore. 

She was not afraid because she had sinned. 

She was afraid because she still wanted to.

And she was afraid of the quiet, devastating question:

\begin{quote}
What if the mask was never the lie?
What if the real betrayal... was wanting it?
What if the real betrayal... was liking it?
\end{quote}

Serena nodded slowly, then reached into the folds of her clutch like she was retrieving a pen 
or a piece of gum. Her fingers emerged with a small vial, the kind that might hold perfume or a single, 
exquisite pill. It clicked softly as she opened it, revealing a lozenge nestled like something sacred.

She extended it toward Emma, her hand open, steady, the gesture more invitation than offer.

``Try this,'' she said, her voice low, almost coaxing. ``Just a little. Let it dissolve under your tongue.''

Emma stared at the lozenge but didn’t reach for it. Her lips parted, but no words came at first. Then, 
after a beat:

``What is it?''

Serena’s eyes didn’t leave hers. ``Just K,'' she said. ``Just for an hour. It won’t hype you up. It’ll 
quiet the static.''

The breeze caught the edge of Emma’s dress, and she pressed it down absently, her gaze still fixed 
on the vial. Serena’s tone softened further, slipping into something tender, almost maternal.

``You don’t have to fix anything tonight,'' she said. ``You just have to breathe.''

Emma looked down at Serena’s hand. Then she looked at 
her own hands, and curled it tightly in her lap like she was bracing for something. 

Slowly, she exhaled...  and reached.

The guilt didn’t vanish. 

The guilt just... unhooked. 

The guilt just detached. 

The guilt became a mist instead of a weight. 

Her body felt distant, but not numb. 

Her thoughts softened at the edges. 

The sharp loops of guilt and shame melted into slow clouds of color.

She didn’t feel high.

She felt weightless.

For the first time in weeks, the inside of her head wasn’t screaming.

She didn’t feel like she owed anyone an apology for simply wanting to feel good.

That night, Emma didn’t say much. But when she curled into bed beside David, she didn’t cry.
She didn’t need to pretend, anymore.

She just closed her eyes.

And floated.

\medskip

\begin{TechnicalSidebar}{Ketamine and the Neurochemistry of Disconnection}

  Ketamine is a dissociative anesthetic first developed in the 1960s and used widely in both 
  surgical and psychiatric settings. In recent years, 
  largely because of its unique ability to disrupt entrenched,
  thought loops and reduce emotional reactivity
  it has found new footing as a treatment for 
  depression, PTSD, and suicidality (Zarate et al., 2006; Feder et al., 2014).

  \medskip

  \textbf{Mechanism of Action:}  
  Ketamine is an NMDA receptor antagonist, meaning it blocks certain glutamate pathways in the brain.  
  Glutamate is the brain’s primary excitatory neurotransmitter, critical for learning, memory, and mood 
  regulation. By interfering with NMDA receptor activity, ketamine shifts neural signaling away from 
  familiar circuits and creates space for new patterns to form (Krystal et al., 2019).

  \medskip

  This neurochemical disruption has two primary psychological effects:

  \begin{itemize}
    \item \textbf{Dissociation:} a separation between self and body, or thought and emotion (Morgan et al., 2004).
    \item \textbf{Cognitive defusion:} the loosening of previously “sticky” thoughts, especially those tied 
    to shame, trauma, or identity (Dakwar et al., 2014).
  \end{itemize}

  \medskip

  In Emma’s case, what she experienced wasn’t just sedation.  
  It was disengagement from the shame circuitry.  
  The inner critic quieted. Emotional rumination dissolved.  
  Under the influence of ketamine, even entrenched guilt can feel like vapor instead of iron 
  (Wilkinson et al., 2017).

  \medskip

  \textbf{But what about sex?}

  \medskip

  While ketamine is not classified as an aphrodisiac, its dissociative properties can have profound effects 
  on sexual perception (Santos et al., 2020):

  \begin{itemize}
    \item It reduces self-referential thinking, which includes body image concerns and performance anxiety.
    \item It temporarily lowers limbic system reactivity, decreasing fear and shame responses.
    \item It enhances sensory distortion, which can make touch feel novel or surreal.
  \end{itemize}

  \medskip

  In controlled therapeutic contexts, these effects have been harnessed to help trauma survivors 
  re-engage with their bodies (Dames et al., 2022).  
  But outside clinical boundaries — especially in chemically complex environments like chemsex —  
  ketamine’s emotional detachment can also become a form of consensual disinhibition, allowing acts that 
  would normally feel emotionally overwhelming to be experienced in a muted or even euphoric register 
  (Bourne et al., 2015).

  \medskip

  This can be freeing.  
  It can also be dangerous.

  \medskip

  Because while ketamine quiets judgment, it doesn’t erase consequence.  
  What it offers isn’t clarity.  
  It’s relief.

  \medskip

  And when relief feels like intimacy, the line between care and control can begin to blur.

  \begin{quote}
    In the right hands, ketamine can create space to heal.  
    In the wrong hands, it creates silence that others can fill.
  \end{quote}

\end{TechnicalSidebar}

