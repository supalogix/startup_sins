\subsection{Filtered Light and Governance Fog}

``Where’s the board memo?'' asked the man in the dark suit — Special Counsel for the Congressional 
Subcommittee on Financial Accountability. He spoke plainly, but each word felt like it had been 
cleared with legal counsel.

David looked down at the folder in front of him. ``Which memo, exactly?''

``The one documenting leadership’s awareness of the leverage adjustment and cross-product exposure. 
The one that should’ve gone to the Risk and Audit Committee in Q2. We’ve reviewed the board packets. 
It’s not there.''

David cleared his throat. ``If it wasn’t escalated, that would’ve been Compliance’s responsibility.''

The counsel nodded once. ``So you didn’t draft a briefing note?''

``No formal memo, no. We discussed elements of it in working groups.''

``Any minutes from those meetings?''

``Possibly. Not all sessions were minuted.''

``Were any slides presented to executive leadership?''

``There were slides,'' David said. ``But they were high-level.''

``How high-level?''

``Portfolio allocation bands. General trends. Scenario ranges.''

``Any mention of the synthetic tranche correlation drift?''

David hesitated. ``Not explicitly, no.''

The counsel glanced down at a binder. ``Your team internally referred to that drift as 
`uncontained contagion velocity’ in a Slack thread dated April 17th. Would you say that rises 
to the level of board visibility?''

David blinked. ``That was informal language.''

``So the board received a sanitized version?''

``They received a \textit{strategic} summary,'' David said carefully.

``Without the risks.''

``Without the emerging anomalies,'' he corrected.

``And who decided those anomalies didn’t merit inclusion?''

``That would have been a judgment call across multiple leads.''

``But your name is listed as the document owner on the draft outline. Yes?''

David didn’t answer.

The counsel didn’t press — not directly.

``Mr. Morales, when boards are kept in the dark, we investigate whether it was by accident or 
by design. Right now, 
it looks like your team filtered the light. That’s not a modeling issue. That’s governance.''

He closed the folder.

``And the next question will be: who gave permission... and who gave cover.''

\medskip

\begin{HistoricalSidebar}{The Congressional Subcommittee on Financial Accountability}

  The \textbf{Congressional Subcommittee on Financial Accountability} is less a financial authority and more a political 
  lens — trained on moments when markets fail and someone, somewhere, must be made to answer.

  \medskip
  
  Historically activated after high-visibility collapses --- Enron (2001), Lehman Brothers (2008), Archegos (2021) --- the 
  Subcommittee is tasked with tracing breakdowns in oversight, disclosure, and board governance. Its focus isn’t technical 
  modeling or trading algorithms; it’s \textit{who knew what, when}, and why warnings were buried, softened, or ignored.

  \medskip
  
  Unlike regulatory bodies such as the SEC or FSOC, which prioritize structural risk, the Subcommittee pursues political 
  and ethical accountability. It doesn’t ask if the system failed. It asks whether people in positions of fiduciary trust 
  failed to act.

  \medskip
  
  In hearings, terms like ``strategic ambiguity,'' ``sanitized summaries,'' and ``decision path opacity'' become signals 
  of willful negligence. In this theater, plausible deniability often reads as intent.

  \medskip
  
  The result may not be criminal indictment. Howeverr, reputational collapse begins here.
  
\end{HistoricalSidebar}



