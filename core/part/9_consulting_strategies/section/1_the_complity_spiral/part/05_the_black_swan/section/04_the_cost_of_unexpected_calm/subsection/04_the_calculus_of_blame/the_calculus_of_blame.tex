
\subsection{The Calculus of Blame}

The scotch bottle was half-empty now.

Kessler sat still, one elbow on the table, hand resting against his temple like he was 
holding up the weight of the firm by leaning. Hart, as always, looked untouched—tie 
straight, voice calm, like he was briefing a client, not salvaging the wreckage of a 
billion-dollar implosion.

``They’re going to come for blood,'' Kessler said. ``Shareholders. The regulator. Maybe 
DOJ, if someone wants a headline.''

Hart nodded. ``They always do. But not all blood is equal.''

Kessler stared at the city outside the window. ``The board will want a name.''

``They’ll need more than a name,'' Hart said. ``They’ll need a narrative. Something 
that sounds like a discrete failure. Something human.''

Kessler turned slowly. ``You mean a scapegoat.''

``I mean a bridge,'' Hart corrected. ``Between what failed... and why it wasn’t the 
board’s fault.''

Silence settled between them.

Hart reached into his jacket pocket and pulled out a folded printout: internal 
exposure timelines, public statements, timestamped trade authorizations.

He slid it across the table.

Kessler didn’t touch it. He already knew what it said.

``David,'' he said flatly.

``He owned the portfolio logic,'' Hart said. ``He led the risk review committee. And 
he greenlit the shipping hedge three weeks before peace talks resumed.''

Kessler let out a bitter chuckle. ``We all greenlit that.''

``Yes,'' Hart agreed. ``But he documented it.''

They sat for a moment. Hart poured more scotch, unhurried.

``He’s not the cause,'' Kessler said, almost to himself.

``No,'' Hart replied. ``But he’s traceable. And right now, optics matter more than truth.''

Kessler tapped the rim of his glass, thinking.

``And if he doesn’t play ball?''

Hart met his gaze. ``He will. He’s loyal. He still thinks this was a system failure.''

``It was,'' Kessler muttered.

Hart didn’t disagree. But he also didn’t care.

``Give the board language they can use in the minutes,'' he said. ``Unanticipated mispricing. 
Model overfitting. Sector hedging fatigue.''

Kessler’s voice was low now. ``That’ll bury him.''

``It’ll protect the firm,'' Hart said. ``What’s left of it.''

A long pause.

Then Kessler leaned back and sighed. ``You’d have made a good priest, you know that?''

Hart smiled faintly.

``Priests have faith,'' he said. ``I just understand absolution.''

\medskip

\begin{TechnicalSidebar}{\textbf{Absolution --- A System for Sanitizing Guilt}}

    In Catholic theology, \textbf{absolution} is the formal release from guilt or 
    punishment granted by a priest after the confession of sins.
    
    \medskip
    
    It is not denial.

    \medskip
    
    
    It is not exoneration.

    \medskip
    
    It is the acknowledgment that sin occurred—followed by a ritualized mechanism for disarming 
    its consequences.
    
    \medskip
    
    The process has four canonical elements:

    \medskip
    
    \begin{enumerate}
        \item \textbf{Contrition}: sincere remorse.
        \item \textbf{Confession}: naming the sin aloud to a priest.
        \item \textbf{Absolution}: priestly declaration of forgiveness.
        \item \textbf{Penance}: prescribed actions to balance the moral ledger.
    \end{enumerate}
    
    \medskip
    
    But in practice, absolution functions less like grace and more like \textit{compliance}:

    \medskip
    
    
    A self-reporting protocol.  
    A containment ritual.  
    A system for reconciling institutional ideals with human failure—without collapsing the 
    institution.
    
    \medskip
    
    Hart’s comment about understanding absolution isn’t religious.

    \medskip
    
    
    It’s procedural.

    \medskip
    
    
    In his world, guilt is not erased.  
    It’s \textit{allocated}.  
    Then sanitized through structure, timing, and distance.

    \medskip
    
    Because sometimes, \textbf{preserving the institution} requires one man to kneel so the others 
    don’t fall.
    
\end{TechnicalSidebar}
