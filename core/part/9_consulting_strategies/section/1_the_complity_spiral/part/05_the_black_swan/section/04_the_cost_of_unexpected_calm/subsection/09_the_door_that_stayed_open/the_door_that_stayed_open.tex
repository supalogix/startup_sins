
\subsection{The Door That Stayed Closed}

The house was warm when he returned.

Music drifted --- low and meandering --- from the kitchen speaker.
The soft hum of domestic ritual filled the air.
The clink of cutlery. 
The buzz of a 
dishwasher mid-cycle. 
The faint scent of something sweet overcooked. 

Emma stood at the stove in an oversized sweatshirt. 
She was barefoot, had her shoulders loose, and one hand 
stirring a pan with no urgency. 
She was softly humming if not a bit off-key. It was a lullaby she used to sing the kids.

She turned when he entered, slower than usual.

``You’re home early,'' she said. 
Her smile appeared a half-second too late to be geniune, though. 
``Anything new?''

David crossed the room and kissed her cheek.

``Nothing urgent.''

Her hand hovered over the pan again.

Her eyes were wide but unfocused. 
Her breath was even. 
Her entire body carried a kind of softness that wasn’t from relaxation.
It was from distance. 

She was here.

But also not.

David saw the way her gaze moved through him instead of to him. 

He saw the way her limbs floated between tasks instead of 
moving with purpose.

He didn’t ask.

Because he already knew.

The first time she took ketamine, it had been the week after the Vermont house.
It was the weekend with Serena and Mia that she took ketamine. 

David remembered waking up and finding her sitting alone on the balcony at dawn, 
and wrapped in a towel. 

She didn't laugh. She didn’t cry. She didn’t scream. She just said, flatly:
``I feel like I left the room but my body stayed.''

He didn't known what to say.

Later that week, she started microdosing. 

She started microdosing... quietly. 

She started microdosing... without permission.

And she started microdosing... without discussion. 

It was not rebellion.

It was relief.

He watched the weight come off her shoulders, molecule by molecule.

The feelings didn’t vanish. 
They just got farther away.

The Ketamine didn’t fix her. 
It just made her float.

And when she drifted, she didn’t break.

Now, watching her stir the pan --- eyes unfocused, and a gentle but unmoored smile --- 
he knew exactly what it meant.

She was underwater again.

She was not not drowning.

She was just choosing not to surface.

He knew that look. 

He knew that tempo. 

He knew that softness.

It was the same softness she wore when the world had asked too much of her and she 
needed the volume turned down.

She was managing... the only way she knew how.

So he said nothing.

Because that was the pact.
Spoken or not.

``How bad is it really?'' she asked, still watching the spoon move in circles.

David leaned against the counter.

``Bad,'' he said. ``But not unmanageable. Everyone’s worried.''

She nodded, as if that was expected. ``They’ll find someone to blame.''

``They always do.''

She smiled again.

It was the kind of smile that meant I know.
But also... please don’t make me say it out loud.

Then came the thunder of footsteps overhead. The kids burst into the kitchen with drawings and crumbs 
and stories about monsters in the dryer.

David let them overwhelm him. Let them pull him into their orbit.

He joked. He praised. He listened like the world hadn’t ended eight hours ago.

But even as he laughed, his mind wandered.

He thought of the trust.
Of the signature.
Of the life he’d sealed for them in exchange for his own silence.

He looked at Emma again.

She was at the sink now, washing something already clean. Her hands moved with a gentle rhythm, 
but her face was soft and far away.

And he understood.

She didn’t want the truth.

She wanted space.

She did not want space because she was weak.

She wanted space because she’d carried her share of heavy things already. 

She was allowed to float now. 

He would carry the weight, so she could drift.

He was determined to make her reality livable.

He would make sure of it.

``I’m starving,'' he said.

She smiled, not fully back, but enough. ``There’s pasta.''

He sat down with the kids. He picked up a crayon. And he colored a house.

\medskip

\begin{PsychologicalSidebar}{\textbf{Choosing the Matrix: The Psychology of Reality Rejection}}

    In a pivotal scene from \textit{The Matrix}, Cypher sits across from Agent Smith, savoring 
    a virtual steak.  
    He knows the steak isn’t real — but he chooses it anyway.

    \medskip
    
    \textit{“Ignorance is bliss,”} he says.

    \medskip
    
    
    Cypher’s decision isn’t just treasonous. It’s psychological.  
    He opts for illusion over knowledge, comfort over clarity, sensation over suffering.

    \medskip
    
    
    This isn’t science fiction.  
    It’s cognitive dissonance in action — a term coined by psychologist \textbf{Leon Festinger} in 1957.
    
    \medskip
    
    \textbf{Cognitive Dissonance:}  
    The psychological discomfort that arises when reality contradicts deeply held beliefs, values, 
    or desires.
    
    \medskip
    
    When dissonance becomes unbearable, the brain seeks relief.  
    Sometimes through rationalization.  
    Sometimes through denial.  
    And sometimes — like Emma — through dissociation.
    
    \medskip
    
    Emma’s ketamine drift isn’t rebellion.  
    It’s relief.

    \medskip
    
    
    A molecular version of Cypher’s steak.  
    A choice not to \textit{not know} — but to not \textbf{feel} the full weight of knowing.
    
    \medskip
    
    In clinical settings, this is known as \textbf{reality aversion coping}.  
    When traumatic or overwhelming environments can’t be escaped externally, the mind 
    escapes internally.  
    It’s not a defect. It’s a survival strategy.
    
    \medskip
    
    David recognizes this.  
    And like Morpheus choosing not to wake Neo prematurely,  
    he honors the unspoken pact:  
    
    \begin{quote}
    \textit{Let her float.  
    Don’t pull her from the matrix  
    if she doesn’t want to leave.}
    \end{quote}
    
    Because sometimes, survival isn't waking up.  
    Sometimes, it’s sleeping just long enough  
    to make it to the next day.
\end{PsychologicalSidebar}
