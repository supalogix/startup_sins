\subsection{The Morning After}

David never went to sleep.

He had stared at the screen until the slide blurred. The typeface swimming in his peripheral vision 
like noise underwater. By 5:42 a.m., he was editing bullet points more out of inertia than purpose. 
The house was still dark except for the glow of the monitor and the amber halo of the hallway nightlight.

Then he heard the soft patter of bare feet on the hardwood.

It was Oliver, the youngest. His Hair was tousled. He was clutching a stuffed octopus by the neck.

``Daddy?''

David turned in his chair. ``Hey, buddy.''

Oliver rubbed his eyes then asked the question with a seriousness that never failed to break 
David’s heart: ``Are you leaving again?''

David smiled, knelt down, and pulled him into a hug. ``Not yet.''

Ten minutes later, both kids were in the kitchen. David was still in yesterday’s shirt, last night’s mind, and was 
now rummaged through cabinets. He found pancake mix, a nonstick pan, and the chipped blue bowl that Emma never 
threw out because it reminded her of their first apartment. 
He stirred batter like he had muscle memory for it,  and flipped pancakes while refereeing an 
argument about syrup ratios.

By the time the second batch was browning, the noise must’ve reached upstairs. Emma walked into the kitchen 
wearing a loose sweater and a sleep-creased face while blinking at the brightness and the smell of butter and maple.

She paused.

``You’re... making breakfast?''

He looked over his shoulder. ``Emergency chef coverage. The regular guy called out.''

Emma chuckled softly and took a seat at the island, where the kids were already giggling over a lopsided 
pancake that looked vaguely like Pikachu.

They ate together, the four of them, at the kitchen counter. 
There was no rush. 
There were no schedules. 
There was just shared space. 
And shared syrup. 
And shared warmth.

And for a brief, flickering moment, it felt like something whole.

David kept sneaking glances at Emma. 
She smiled more in that one hour than he could remember in weeks. 
It was not the polite smile she wore at client dinners.
It was not the tight-lipped nod she gave when he said he was ``almost done.''
It was a real smile. Her smile was soft around the eyes. Her smile was present.

He tried to lock it into memory.

He couldn’t remember the last time she had actually enjoyed his company. Not tolerated it. Not supported it. 
Actually, enjoyed it.

Since the kids came, their connection had been rerouted. 
She had grown closer to them in ways that felt untouchable. 
And David had grown further from her, not out of malice, but out of momentum.

He didn’t blame her. 
She had every right to turn toward the people who needed her back.

And David? 
He told himself that he would make it up to her. 
He told himself that he would make it up to all of them.
He would make up for the late nights... and the missed recitals... and the silent gaps in the marriage.

He would make it all worth it.

Someday.

Because this wasn’t about escape. 
It was about building something they could all live inside.
It was about building something resilient. 
IT was about building something strong.

Even if it meant he had to stand outside of it for a while.

After breakfast, he kissed them all --- quick, like punctuation --- and grabbed his bag by the door.

His flight was in two hours.

But all he could think about was the warmth of syrup on her fingers.
And the way she smiled when she thought he wasn’t looking.


\subsection*{Editor Questions for ``The Morning After''}

This scene is quieter — more atmospheric than expository. The goal of these questions is to help surface what’s working on a subtle, emotional level and where it might land softer than intended. Focus on the resonance, intimacy, and implied stakes. You don’t have to answer everything — just respond where you felt something shift.

\subsubsection*{Narrative \& Structure}

\begin{itemize}
  \item Did the scene unfold at the right pace for its tone? Was anything rushed or overly drawn out?
  \item How well did the transition from solitude to domestic warmth land for you as a reader?
  \item Was there a moment that felt like the emotional or narrative pivot? Did it arrive at the right time?
\end{itemize}

\subsubsection*{Emotional Resonance}

\begin{itemize}
  \item What did this scene make you feel — and when did you feel it most?
  \item Did David’s emotional undercurrent (guilt, longing, resolve) come through clearly?
  \item Did the warmth of the scene feel earned, or did it risk sentimentality?
\end{itemize}

\subsubsection*{Character Insight}

\begin{itemize}
  \item Did David’s actions (making breakfast, watching Emma) feel honest to who he is?
  \item What do you learn about Emma, even though she says very little?
  \item What does this scene suggest about the emotional architecture of their marriage?
\end{itemize}

\subsubsection*{Scene Texture}

\begin{itemize}
  \item Did the domestic details (pancakes, syrup arguments, chipped bowl) enhance your immersion?
  \item Was there a moment that felt especially visual or sensory for you?
  \item Did the contrast between David’s professional world and this kitchen scene feel intentional — or like a temporary escape?
\end{itemize}

\subsubsection*{Theme \& Message}

\begin{itemize}
  \item What do you think this scene is ultimately about: redemption, guilt, sacrifice, or something else?
  \item Did the “someday” refrain (about making it up to them) feel hollow, hopeful, or heartbreaking?
  \item Did this scene add depth to your understanding of David’s internal conflict? If so, how?
\end{itemize}

\subsubsection*{Style \& Craft}

\begin{itemize}
  \item Was there a line or gesture that lingered with you after reading?
  \item Did the rhythm of the prose mirror the emotional tone?
  \item Did anything feel overwritten or unnecessary given the softness of the moment?
\end{itemize}

\subsubsection*{Deeper Testing}

\begin{itemize}
  \item What would change emotionally if this scene were cut from the story?
  \item If the scene ended just before the breakfast — or just after the flight — would it be stronger or weaker?
  \item If you were reading this as part of a longer work, what expectations would this scene set for what’s to come?
\end{itemize}










