
\subsection{The Locked Room}

The next morning, Emma took the kids to church.

She laid out their clothes the night before. It was not because they couldn’t choose for themselves, but 
because it gave her a small sense of order. Nora’s blue dress with the white sash. Oliver’s collared shirt, 
the one that always looked a little too grown-up until he smiled.

The morning moved with practiced efficiency: waffles from the freezer, hair brushed while shoes were 
hunted, cereal bowls soaking in the sink like artifacts from another life. 

David used to take Sundays off.

It was tradition. Not in the religious sense, though that was important, too. It was more a ritual of 
time: pancakes late, museum trips if the weather held, and afternoon walks through the park where they 
each took a kid’s hand and let the quiet do most of the talking.

Sunday was their day.

Until it wasn’t.

Now Sunday was calls to Tokyo. Risk reports queued for Monday morning. Expense reviews. Syncs. Slide decks. 
A workday in everything but name.

And not just for him anymore.

She’d told herself that the stress test of modern ambition was temporary. But slowly, imperceptibly, the 
stillness they used to guard had been spent.

So today, she dressed the kids. Buckled them in. Drove to the same church with the white steps and the 
smell of lemon polish in the foyer. Not for the sermon --- she could barely remember it --- but for the 
pause. The structure. The breath between weeks.

She didn’t expect peace.

But she wanted a moment where the clock stopped pretending it was in charge.

Oliver went to his Sunday school class. He liked the room. It had a rug shaped like a lion and posters 
about kindness in blocky letters. His teacher, Mrs. Grace, had a voice like an audiobook and let them 
sit on beanbags if they didn’t squirm too much.

Today’s lesson was about being the children of God.

She said it gently, like it was supposed to be comforting.

\begin{quote}
God is not far away. He’s not distracted. He’s not too busy. God hears you. God sees you. God wants to 
be with you.
\end{quote}

The other kids nodded or whispered their coloring choices to each other. But Oliver just sat still, 
staring at the picture of the Good Shepherd on the felt board.

He thought to himself: 
\medskip

\begin{quote}
We are God's children. So that means that God is like a father. But if God is like a father, 
and my dad is a father...  then why is God not like my dad? Is my dad doing it wrong?
\end{quote}

\medskip

It wasn’t an angry question. It was a quiet one. It was like standing at the edge of a swimming pool and 
wondering why the deep end doesn’t have a bottom.

Later in class, Mrs. Grace passed out blank paper and crayons.

``You can draw your house,'' she said. ``Where everyone 
goes. Where your favorite things are. Like a floorplan. Like you're the architect.''

Oliver liked the idea. He liked layouts. He liked putting things where they were supposed to go.

He started with rectangles.  
The living room with the soft gray couch.  
The kitchen with the round table and the cabinet that held the pancake mix.  
His sister’s room with stuffed animals.  
Mom’s room with the tall lamp she always turned on at night.  
The driveway, with the car that never quite parked straight.  
The front yard, with the swing set and the plastic planet he buried last summer under the tree.

And then — carefully, precisely — he drew a small rectangle near the back of the house. No windows. 
No labels. Just a thick, solid black box.

Mrs. Grace crouched beside him, peeking at the page.

``What’s this room here?'' she asked, pointing gently.

Oliver didn’t look up. He traced over the box one more time with his crayon, pressing hard enough to 
dull the tip.

``That’s my dad’s study.''

``Why is it all black?''

He paused. ``Because I don’t know what’s in there.''

``Why don’t you know?''

``It’s where my dad goes when he needs to focus,'' Oliver said. ``He says it’s his quiet place where 
he can think without being interrupted.''

She nodded.

``Does he let people in?''

Oliver shook his head.

``Not really. Sometimes he comes out and asks what day it is. He says he forgets when he’s working.''

``And you? Have you ever been inside?''

He looked down at the drawing.

``No. That's why I don't know what to draw.''

Mrs. Grace was quiet for a moment.  She didn’t press further.

Oliver finished the drawing. Every room had details: books, chairs, colors. But the black room stayed blank.  
No doors. No windows. No label. Just a box.  
Heavy with silence.

Later, when Emma picked them up from their classes and asked how it went, Oliver said,

``We learned God’s always with us.''

Emma smiled.

``That’s true.''

Oliver looked at her a little longer than usual.

``Do you think God has a locked room?''

Emma blinked --- just for a second --- as if the question had hit deeper than she expected.

Then she smiled gently, eyes still on the road.

``I think sometimes people need quiet places to think,'' she said. ``Even God. It doesn’t mean 
He isn’t close. He's just... thinking hard. Like Dad when he’s working.''

Oliver nodded, but didn’t say anything.

He turned to look out the window, and watched the sidewalk blur past.  
The silence between them stretched.

Emma didn’t push.  And Oliver didn’t ask again.


\medskip

\begin{PsychologicalSidebar}{Attachment and the God Image}

    In the 1990s, psychologist \textbf{Lee A. Kirkpatrick} extended \textbf{John Bowlby’s} attachment 
    theory into the domain of religious psychology.
    
    \medskip
    
    Kirkpatrick proposed that a person’s internal image of God is not formed in a vacuum. It’s often an 
    echo of their earliest attachment relationships, especially with parents.
    
    \medskip
    
    In Bowlby’s model, children develop \textbf{“internal working models”} based on how their caregivers 
    respond to their needs. If a child consistently experiences comfort, responsiveness, and safety, they 
    internalize the belief: \textit{``When I reach out, someone will be there.''}
    
    \medskip
    
    Kirkpatrick argued that these models often extend to the spiritual realm:

    \medskip
    
    \begin{itemize}
      \item Children with \textbf{secure attachments} tend to view God as loving, reliable, and present: 
      someone who listens, protects, and remains close even when others fail.
      \item Children with \textbf{insecure or avoidant attachments} are more likely to perceive God as 
      distant, emotionally cold, or inconsistent. Some may fear abandonment; others may reject belief in 
      God entirely as a defense against disappointment.
    \end{itemize}
    
    \medskip
    
    This doesn’t mean theology is reducible to psychology, but it does suggest that the \textbf{emotional 
    architecture} of faith is shaped early, and often unconsciously. A child doesn’t need to be taught 
    theology to develop a theology of presence. They simply ask:  
    ``When I needed love, who came?''
    
    \medskip
    
    In Oliver’s case, Sunday school didn’t just teach doctrine. It introduced a spiritual comparison. God 
    was described as attentive, near, and available. However, that didn’t line up with Oliver’s experience 
    of his earthly father. That gap becomes a quiet crisis:
    
    \begin{quote}
    \textit{If God is like a father, and my father feels far away, is God far away too?}  
    \textit{Or is my dad doing it wrong?}
    \end{quote}
    
    For a child, that’s not a theological abstraction. It’s a question of safety... and of love.
    
\end{PsychologicalSidebar}


\subsection*{Editor Questions for ``The Locked Room''}

This scene relies on parallelism — the structure of church, the structure of family, the structure of faith — to explore spiritual dissonance through a child’s eyes. These questions are meant to evaluate how that layering lands: emotionally, structurally, and symbolically. You don’t have to answer every question. Focus on what felt most (or least) effective to you as a reader.

\subsubsection*{Narrative \& Structure}

\begin{itemize}
  \item Did the Sunday morning setup flow naturally from the previous scenes? Or did it feel like a thematic detour?
  \item How did the intercutting between Emma’s experience and Oliver’s class work for you?
  \item Was the pacing effective — especially the transition from floorplan drawing to theological questioning?
\end{itemize}

\subsubsection*{Emotional Resonance}

\begin{itemize}
  \item How did you feel reading Oliver’s reflections about his father and God? Did it hit emotionally or feel too cerebral?
  \item Did Emma’s final answer land as comforting, heartbreaking, evasive — or something else?
  \item Was the moment of the black box in the drawing too on-the-nose, or did it feel earned?
\end{itemize}

\subsubsection*{Character Insight}

\begin{itemize}
  \item Did Oliver feel like a believable child — perceptive, but still developmentally appropriate?
  \item Did Emma’s silent processing of his question deepen your understanding of her character?
  \item How does this scene reframe or evolve your sense of David, even though he doesn’t appear?
\end{itemize}

\subsubsection*{Psychological Sidebar}

\begin{itemize}
  \item Did the sidebar on ``Attachment and the God Image'' deepen your understanding of the scene?
  \item Did the integration of Kirkpatrick and Bowlby feel illuminating, too academic, or just right?
  \item Would you prefer this content as narrative subtext, or does the sidebar format support the experience?
\end{itemize}

\subsubsection*{Theme \& Message}

\begin{itemize}
  \item What do you think this scene is ultimately about — faith, absence, inherited models of love?
  \item Did it raise any philosophical, theological, or personal reflections for you?
  \item Is the metaphor of the ``locked room'' clear? Overdone? Or quietly powerful?
\end{itemize}

\subsubsection*{Style \& Craft}

\begin{itemize}
  \item Did the child’s voice feel natural and emotionally layered?
  \item Was the tone of the Sunday school class believable and appropriately contrasted with Oliver’s interiority?
  \item Did any parts feel emotionally manipulative, overwritten, or unclear?
\end{itemize}

\subsubsection*{Deeper Testing}

\begin{itemize}
  \item If you had to describe this scene in one word, what would it be — and why?
  \item If the black box were removed from the drawing, what would be lost?
  \item What would change if this scene came earlier or later in the narrative?
\end{itemize}
