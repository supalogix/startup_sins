
\subsection{The Locked Room}

The next morning, Emma took the kids to church.

It wasn’t a holiday or a special service — just a normal Sunday, the kind that smelled like crayons and 
hand sanitizer and too many coffee cups stacked by the welcome table. The sanctuary was cool and hushed 
when they walked in, sunlight catching the stained-glass dust in the air. Emma kept her sunglasses on 
a little longer than usual before dropping them into her purse.

Oliver went to his Sunday school class. He liked the room. It had a rug shaped like a lion and posters 
about kindness in blocky letters. His teacher, Mrs. Grace, had a voice like an audiobook and let them 
sit on beanbags if they didn’t squirm too much.

Today’s lesson was about God being a Father.

She said it gently, like it was supposed to be comforting.

\begin{quote}
God is not far away. He’s not distracted. He’s not too busy. God hears you. God sees you. God wants to 
be with you.
\end{quote}

The other kids nodded or whispered their coloring choices to each other. But Oliver just sat still, 
staring at the picture of the Good Shepherd on the felt board.

\medskip

\begin{quote}
But if God is like a father, and my dad is a father...
then is God like my dad?
Or is my dad doing it wrong?
\end{quote}

\medskip

It wasn’t an angry question. Just a quiet one. Like standing at the edge of a swimming pool and 
wondering why the deep end doesn’t have a bottom.

Mrs. Grace passed out drawing paper. She said they could draw their family, or their house, or anything 
that made them feel safe.

Oliver drew a house. It had four windows and a red door and a little chimney like the ones in picture 
books. Then he added rooms — a kitchen with pancakes, a room with toys, his sister’s room, Mom’s room.

And then --- almost as an afterthought --- he drew another room in the back.

It had no window. Just a big brown door with a black line where the lock should be.

Mrs. Grace crouched beside him, her knees cracking softly as she sat on the rug.

``Who lives in that room?'' she asked.

Oliver didn’t look up. He tapped the door with his crayon.

``That’s where Dad keeps his work.''

She smiled gently.

``Is he in there a lot?''

He nodded.

``Sometimes I think he lives there.''

She paused.

``Does it feel like there’s a key?''

Oliver shrugged.

``Not for me.''

She didn’t push.

He finished the drawing without adding a doorknob.

Later, when Emma picked them up from their classes and asked how it went, Oliver said,

``We learned God’s not too busy.''

Emma smiled.

``That’s true.''

Oliver looked at her a little longer than usual.

``Do you think God has a locked room?''

Emma blinked.

``No, honey. I think He wants to be in all the rooms.''

He nodded.

Then said nothing the whole ride home.

\medskip

\begin{PsychologicalSidebar}{Attachment and the God Image}

    In the 1990s, psychologist \textbf{Lee A. Kirkpatrick} extended \textbf{John Bowlby’s} attachment 
    theory into the domain of religious psychology.
    
    \medskip
    
    Kirkpatrick proposed that a person’s internal image of God is not formed in a vacuum. It’s often an 
    echo of their earliest attachment relationships, especially with parents.
    
    \medskip
    
    In Bowlby’s model, children develop \textbf{“internal working models”} based on how their caregivers 
    respond to their needs. If a child consistently experiences comfort, responsiveness, and safety, they 
    internalize the belief: \textit{``When I reach out, someone will be there.''}
    
    \medskip
    
    Kirkpatrick argued that these models often extend to the spiritual realm:

    \medskip
    
    \begin{itemize}
      \item Children with \textbf{secure attachments} tend to view God as loving, reliable, and present: 
      someone who listens, protects, and remains close even when others fail.
      \item Children with \textbf{insecure or avoidant attachments} are more likely to perceive God as 
      distant, emotionally cold, or inconsistent. Some may fear abandonment; others may reject belief in 
      God entirely as a defense against disappointment.
    \end{itemize}
    
    \medskip
    
    This doesn’t mean theology is reducible to psychology, but it does suggest that the \textbf{emotional 
    architecture} of faith is shaped early, and often unconsciously. A child doesn’t need to be taught 
    theology to develop a theology of presence. They simply ask:  
    ``When I needed love, who came?''
    
    \medskip
    
    In Oliver’s case, Sunday school didn’t just teach doctrine. It introduced a spiritual comparison. God 
    was described as attentive, near, and available. However, that didn’t line up with Oliver’s experience 
    of his earthly father. That gap becomes a quiet crisis:
    
    \begin{quote}
    \textit{If God is like a father, and my father feels far away, is God far away too?}  
    \textit{Or is my dad doing it wrong?}
    \end{quote}
    
    For a child, that’s not a theological abstraction. It’s a question of safety... and of love.
    
\end{PsychologicalSidebar}