\subsection{The Daughter’s Voice}

After the playground, Emma took her daughter to the therapist’s office.

It wasn’t a crisis visit. 
It was something they’d started six months ago. 
It was every other week. 
It was mostly at the suggestion of the school counselor, who’d flagged some mild behavioral shifts.
She was more withdrawn in group work, more reactive to perceived slights, sometimes too quiet, and sometimes too loud. 
It was nothing dramatic. 
It was just a girl holding something she didn’t yet know how to name.

The office was small and sunlit. 
It had a textured rug and oversized cushions that looked like they belonged in a Pinterest nursery. 
It had a sand tray sat by the window. 
It had books about feelings lining one wall. 
Emma stayed in the waiting room. 
She always did.

Inside, Dr. Patel sat in a low chair, a notepad resting lightly on her lap. 
Her voice was warm, but never sweet. 
She didn’t talk down. 
She waited. 
She noticed.

``So,'' she said after a few minutes of quiet. ``Did anything feel different today?''

The girl shrugged. ``We went to the park. I saw my friend Lila. We made a game about the slide being a time machine.''

``That sounds fun,'' Dr. Patel said. ``Do you remember what year you time-traveled to?''

``3025,'' she said immediately, then added, ``There were no parents. Only kids. And snacks. And roller skates.''

Dr. Patel smiled. ``No parents?''

``Yeah. Just us. We made the rules.''

There was a pause.

``Do you ever wish there were days like that here?''

The girl didn’t answer right away. Then: ``Not really. I like my mom. She makes pancakes that look like animals.''

``What about your dad?''

Another pause. 
This one was longer. 
Her legs swung slowly back and forth.

``He’s... busy.''

Inside the softly lit office, Dr. Patel leaned forward. 
She rested a notepad on her lap with uncapped pen untouched.

``Can I ask you something kind of weird?'' she said gently.

The girl nodded.

``When I say the word \textit{Dad}, what picture pops into your head first?''

The girl tilted her head. 
``I don’t know. Not like... a picture-picture. Just... he’s upstairs. Or on a plane. 
Or in the office. Or sometimes at dinner, but kinda not all the way there.''

Dr. Patel nodded. ``What do you mean, ‘not all the way there’?''

``Like... I show him my drawing and he says, ‘That’s great, sweetie,’ but he doesn’t ask what it is. 
He just sticks it on the fridge and walks away.''

``How does that feel when that happens?''

The girl shrugged. ``It’s not bad. I forget what the picture was, too, sometimes. So then we both 
don’t know. And that’s just... what happens.''

There was a pause. Dr. Patel didn’t rush it.

``Have you ever thought about what other kids’ dads are like?''

The girl’s eyes lit a little. ``Lila’s dad swims with her.''

``Do you and your dad do that?''

She shook her head. ``He doesn’t like swimming. He says it's not relaxing.''

``Do you think work is relaxing for him?''

A quiet laugh. ``No. But he does a lot of it. So maybe he likes it more than games. Or maybe...'' 
she paused. ``Maybe he likes airports.''

Dr. Patel smiled softly. ``Airports?''

``Yeah. He’s always going to one. Or coming from one. Or talking about one. He says the Wi-Fi’s never 
good, but he still goes a lot.''

Dr. Patel adjusted her tone, making it just a little lighter. ``Do you think he misses you when he’s gone?''

The girl blinked. ``I think he wants to. But he’s busy. And when people are busy, they forget to feel stuff.''

Dr. Patel jotted something quickly, then said, ``What about your mom?''

The girl sat up straighter. ``She makes pancakes that look like animals.''

``That sounds special.''

``It is.'' She paused. ``She smiled a lot at the park today. But before that, she looked... sad. 
Not crying sad. Just tired-sad.''

``Do you ever ask her about that?''

``No. I think she wants me not to. So I don’t. I just be good. So she doesn’t have to work more.''  

She glanced down. ``Dad says I’m mature for my age.''

Dr. Patel tilted her head. ``What do you think he means by that?''

The girl thought for a long time. Then looked up and said:

``I think he means I don’t cry when I want to.''

The session ended with a sticker and a soft goodbye. 
Outside, Emma stood up as her daughter emerged from the office.
She re-fastend her smile. 
Emma asked if she wanted to get a smoothie. 
The girl said yes. 
She didn’t mention the time machine.

And Emma didn’t ask what year she had traveled to.

\medskip

\begin{PsychologicalSidebar}{Talk Therapy and the Search for Meaning}

  What Dr. Patel is doing in this scene might look simple. It was a gentle question, a nod, and 
  a space held open. But she’s practicing a form of therapy with deep roots in the modern 
  understanding of the mind.
  
  \medskip
  
  \textbf{Talk therapy}, or \textit{psychotherapy}, began in the late 19\textsuperscript{th} century 
  with \textbf{Sigmund Freud}, who believed that unspoken emotions and past experiences — especially 
  those repressed or misunderstood — could manifest as psychological symptoms 
  (Freud, 1905/1953). His ``talking cure,'' developed through work with patients like Anna O., marked 
  a radical shift: \textit{speech itself could be therapeutic.}
  
  \medskip
  
  Over time, Freud’s methods evolved through disagreement and refinement. His student-turned-rival 
  \textbf{Carl Jung} emphasized dreams, archetypes, and the collective unconscious (Jung, 1968). 
  Later, \textbf{Carl Rogers}, founder of humanistic therapy, rejected diagnosis and interpretation 
  altogether. Instead, he insisted that healing arises from \textbf{unconditional positive regard} 
  and \textbf{empathetic listening} (Rogers, 1951). His client-centered approach lives on in 
  therapists like Dr. Patel, who invite meaning without imposing it.
  
  \medskip
  
  For children, talk therapy isn’t just catharsis. It’s \textbf{cognitive scaffolding}.

  \medskip
  
  Psychologist \textbf{Lev Vygotsky} proposed that a child’s understanding of their own inner life 
  emerges through \textbf{social speech} (Vygotsky, 1978). They learn language from others, begin to 
  use it on themselves, and form internal dialogue. Therapy becomes a space where that dialogue is 
  first made visible.
  
  \medskip
  
  In 1995, developmental psychologist \textbf{Daniel Siegel} introduced the phrase \textbf{“narrative 
  integration”} to describe how children form coherent identities by telling stories about their 
  experiences (Siegel, 1999). Without a chance to process events — especially emotionally complex ones 
  like absence, disappointment, or unspoken conflict — a child may grow up with fragmented or distorted 
  self-understanding.
  
  \medskip
  
  In the session above, Dr. Patel isn’t just probing for answers. She’s helping the girl externalize 
  what she feels but doesn’t yet conceptualize.  
  She’s teaching her that:

  \medskip
  
  \begin{itemize}
    \item It’s okay to say what something felt like.
    \item You don’t need to fix a parent to name your experience of them.
    \item Not all stories have to be heroic to be valid.
  \end{itemize}
  
  \medskip
  
  Children like David’s daughter often carry \textbf{emotional ambiguity}. It is the sense that 
  something feels off, but no one’s naming it. Talk therapy gives shape to that fog. It doesn’t 
  force conclusions. It gives vocabulary to lived tension.

  \medskip
  
  And sometimes, the most important truth a child learns in therapy is the one they say without 
  realizing it — \textit{``I think he means I don’t cry when I want to.''}
  
\end{PsychologicalSidebar}

  
\subsection*{Editor Questions for ``The Daughter’s Voice''}

This scene is quiet by design, but that doesn't mean it's soft. These questions aim to help evaluate how well the emotional weight lands, how the child’s voice operates within the larger narrative, and whether the psychological layering supports or interrupts the experience. Focus on the prompts that resonate most with your reading.

\subsubsection*{Narrative \& Structure}

\begin{itemize}
  \item Did this scene feel like a natural extension of the playground sequence? Or too much of a tonal shift?
  \item Was the structure — alternating therapist questions and child responses — effective in sustaining engagement?
  \item Did the rhythm of the session feel realistic, or too scripted?
\end{itemize}

\subsubsection*{Emotional Resonance}

\begin{itemize}
  \item How did you emotionally respond to the child’s voice? Did she feel authentic or overly precocious?
  \item Were there moments where you felt a pang — a sentence that surprised or pierced?
  \item Did the closing line (“I think he means I don’t cry when I want to”) land with emotional force, or feel too crafted?
\end{itemize}

\subsubsection*{Character Insight}

\begin{itemize}
  \item What do you learn about David through the daughter’s language?
  \item How does Emma’s silence outside the office affect your sense of her role in this scene?
  \item Does Dr. Patel come across as a believable therapist figure — grounded, warm, appropriately restrained?
\end{itemize}

\subsubsection*{Psychological Sidebar}

\begin{itemize}
  \item Did the sidebar on ``Talk Therapy and the Search for Meaning'' enrich your understanding of the scene, or interrupt its mood?
  \item Was the integration of Vygotsky and narrative integration helpful context, or too academic?
  \item Would this content be more effective integrated into the main narrative, or does the standalone format work better?
\end{itemize}

\subsubsection*{Theme \& Message}

\begin{itemize}
  \item What do you think this scene is ultimately about — parenting, identity formation, absence?
  \item Did it raise any personal memories or associations for you?
  \item Is this a character development moment, or does it speak to something systemic (e.g., emotional neglect, generational patterns)?
\end{itemize}

\subsubsection*{Style \& Craft}

\begin{itemize}
  \item Did the child’s voice strike the right balance between age-appropriate simplicity and layered meaning?
  \item Were there any lines that felt emotionally manipulative or forced?
  \item How did the setting — sand tray, sunlight, sticker at the end — contribute to the emotional atmosphere?
\end{itemize}

\subsubsection*{Deeper Testing}

\begin{itemize}
  \item If this were the only scene you read, what assumptions would you make about David as a father?
  \item What changes — if any — would you suggest to make the daughter’s voice more impactful?
  \item If you had to cut 10–15\% of this scene, what would go without compromising its emotional weight?
\end{itemize}
