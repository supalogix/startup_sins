\section{ROI as a Feeling: Why Consultants Never Quantify Success (and You Shouldn’t Ask)}

\begin{quote}
If they ask for metrics, pivot to “brand impact.” If they insist, say it’s “too early to tell.”
\end{quote}

  When it comes to \textbf{ROI as a Feeling}, consultants have mastered the art of turning the absence of measurable success into a strategic advantage.
  
  \medskip
  
  \textbf{Law 22} from \textit{The 48 Laws of Power} explains this maneuver:
  \begin{quote}
  ``When you are weaker, never fight for honor's sake; instead, surrender. Surrender gives you time to recover, time to undermine, and time to wait for your opponent's power to wane.''
  \end{quote}
  
  \medskip
  
  Translation in consultant-speak: \\
  \textit{``We don’t have the numbers yet—but that’s because true impact takes time.''}
  
  \medskip
  
  By \textbf{surrendering} to the fact that ROI can't be shown today, they shift the narrative from failure to \textit{``long-term strategic value.''} \\
  The longer they delay concrete metrics, the harder it becomes for you to hold them accountable—because now you're ``invested in the journey.''
  
  \medskip
  
  If every request for data is met with phrases like \textit{``brand uplift''}, \textit{``market positioning''}, or \textit{``too early to quantify''}, you’re watching Law 22 in action.
  
  \medskip
  
  \textbf{Remember:} A real ROI isn’t afraid of being measured. \\
  If success is always just over the horizon, you’re being managed—not informed.
  


\ExecutiveChecklist{high}{When ROI Is Just a Feeling}{
  \item Insist on pre-defined ROI metrics—quantitative, not qualitative.
  \item Ask, “If this fails, how will we know?”
  \item Watch for evasive language like “brand uplift” or “strategic alignment.”
  \item Demand post-mortems on past client engagements.
}



\subsection{Case Study: The Pipeline That Knew Too Much (TitanBank, 2023)}

TitanBank had a problem—not with fraud detection, but with fraud detection that worked too well.

When the Risk Engineering team greenlit a multi-phase project titled \textit{Adaptive ML for Internal Anomaly Auditing}, the board saw it as a forward-thinking investment. On paper, it looked like a best-in-class pipeline for catching financial irregularities. In practice, it became a \$4.2 million ghost system—never deployed, never challenged, never explained.

At the center of the project were two people: a consultant from a boutique AI firm, and the bank’s own VP of Risk Engineering. Their relationship was professional in name only. The contract renewals weren’t tied to KPIs or deployment milestones—but to a more implicit agreement: the consultant’s invoices flowed steadily, padded and routed through a chain of shell companies quietly controlled by the VP himself. Every extension of the project wasn’t just another month of work—it was another layer in a laundering scheme hidden beneath the veneer of innovation.

\begin{itemize}
  \item The VP had chosen the boutique AI firm not for their technical skill, but for their vulnerability: small enough to be desperate for a big-name client, naive enough to mistake his interest for opportunity.
  \item From the start, the consultants were steered into a quiet arrangement: they were coerced into providing the VP with recordings of themselves in sexually compromising situations—footage that could be wielded to threaten their personal lives and reputations, ensuring their compliance as collateral.
  \item In exchange, the VP guaranteed continuous funding, shielded them from technical audits, and ensured no reviewer ever saw the full pipeline output.
  \item The pipeline itself was no accident: it was deliberately misconfigured—just enough buzzwords to impress the board, never enough rigor to detect the fraud buried deep in TitanBank’s books.
\end{itemize}

This wasn’t oversight—it was strategic surrender. The consultant had no power to enforce the pipeline’s adoption, so they leaned into Law 22: \textit{“When you are weaker, never fight for honor’s sake… surrender gives you time to undermine.”} In this case, surrender meant giving the VP what he wanted—not a solution, but a liability.

\begin{tcolorbox}[colback=blue!5!white, colframe=blue!50!black, breakable, title={Historical Sidebar: When ROI Becomes Loyalty --- China's Real Estate Blackmail Scandals}]

In the 2000s and early 2010s, China's booming real estate market created intense pressure on developers to win government contracts.  
For some, competitive bids and transparent negotiations were too slow—or too uncertain.

\medskip

Instead, a darker strategy emerged:  
\textbf{Shift the measure of value from public results to private loyalty}.

\medskip

\begin{itemize}
    \item Developers orchestrated sexual blackmail schemes against Communist Party officials.
    \item Bribery, favors, and secret relationships replaced competitive pricing or tangible outcomes.
    \item Officials granted favorable land deals not based on performance, but on personal compromise.
\end{itemize}

\medskip

The most famous example was Lei Zhengfu, a party secretary secretly filmed in a hotel by operatives working for a developer.  
The resulting scandal exposed dozens of officials and shattered public trust—but only after years of "success" built on hidden incentives.

\medskip

\begin{quote}
The implicit pitch: \textbf{You don't have to show public results if you privately secure loyalty}.
\end{quote}

\medskip

\textbf{The Lesson?} When "value" becomes something you can't measure—and aren't supposed to measure—real failure is already underway. It just hasn't surfaced yet.
\end{tcolorbox}

By removing \textbf{performance} as the metric, the project became something else entirely: \textbf{a loyalty engine}. When you decouple a project from measurable outcomes, it no longer functions as a tool for delivering results; it functions as a test of allegiance. Its persistence signals who is willing to support it \emph{despite} its inefficacy, who will sign off on its continuation, and who will absorb its costs without protest. The project’s existence becomes a litmus test for loyalty to leadership, institutional inertia, or an unspoken political order. Every renewal becomes less about the project’s goals and more about \textbf{who’s willing to publicly align themselves with it}.

\medskip

Each year it was renewed, it created a deeper trail of complicity. With every budget cycle, every reauthorization, every quarterly report rubber-stamped without challenge, more stakeholders became entwined in its survival. Each signature, each approval, each omission of critique added another thread to the web of complicity. Over time, the project didn’t just persist—it \textbf{accumulated co-ownership of its failure}. To challenge it now was to indict one’s past decisions. The cost of dissent rose, because to repudiate the project was to repudiate \emph{your own role} in its endurance.

\medskip

ROI couldn’t be quantified, because \textbf{quantification was the threat}. Measurement would imply accountability; it would invite uncomfortable questions: \emph{“Did it work?”} \emph{“Was it worth it?”} \emph{“Why are we still funding it?”} But these questions were dangerous precisely because the answers were already known—or worse, because they would expose a system that was \textbf{never designed to deliver results in the first place}. Metrics would not clarify; they would unravel the protective narrative. So measurement was avoided, deferred, obscured, or replaced with qualitative justifications: \emph{“The impact is intangible.”} \emph{“It’s about positioning, not performance.”} In this way, quantification became an existential threat not to the project’s goals, but to the political or institutional scaffolding holding it up.

\medskip

\noindent
\textbf{Qualitative summary:} \\
The project’s continued existence operated less like a functional enterprise and more like a ritualized act of loyalty signaling. It bound participants together not through shared outcomes, but through shared investment in the maintenance of the fiction. Over time, it became a form of \emph{mutually assured discretion}: no one could blow the whistle without implicating themselves.

\medskip

In effect, the project’s primary output was not the product or service it nominally provided—it was the slow but steady creation of a \textbf{political economy of silence and complicity}.

And so, a shadow economy bloomed inside the bank—founded not on results, but on mutual exposure.

\medskip

\textbf{The Takeaway:}  
This is why the checklist matters. It's not a formality—it’s your early warning system.  If a project can’t be evaluated, it can’t be trusted.  And if it can’t be trusted, you’d better ask who benefits from the ambiguity.

\begin{tcolorbox}[colback=blue!5!white, colframe=blue!50!black, breakable, title={Historical Sidebar: Kompromat --- When Blackmail Becomes the Business Model}]

  \textbf{Kompromat} (short for “compromising material”) is the Russian art—and strategic practice—of gathering scandalous, damaging, or embarrassing information to control, coerce, or neutralize an individual.  
  While associated with Soviet and post-Soviet intelligence operations, its roots trace back to Tsarist secret police tactics: \textbf{“control the person, control the outcome.”}
  
  \medskip
  
  In the corporate world, kompromat evolved as a tool of \textbf{corporate espionage}:  
  \begin{itemize}
      \item Competitors targeted executives during overseas conferences, planting hidden cameras in hotel rooms or arranging encounters engineered to look compromising.
      \item Such material was leveraged not always to fire or remove—but to \emph{influence decisions}, secure contracts, or ensure quiet compliance.
      \item The goal wasn’t public scandal—it was \textbf{private leverage}.
  \end{itemize}
  
  \medskip
  
  But like all coercive tools, kompromat carries a risk: what happens if the target doesn’t feel shame?
  
  \medskip
  
  \textbf{The Case of Sarahuto:}  
  Legend tells of \textbf{Kazuo Sarahuto}, a Japanese trade envoy in the 1970s, visiting Moscow for a series of negotiations.  
  After several days, KGB agents summoned him privately:
  
  \begin{quote}
  “We have something to show you,” they said. “Footage of you in an orgy—with Russian flight attendants.”
  \end{quote}
  
  Expecting panic, they instead watched him laugh.
  
  “Excellent!” Sarahuto replied. “Can I have a copy? I want to show my friends back home—I’ll never live this down if I don’t!”
  
  \medskip
  
  Faced with a target immune to embarrassment, the kompromat lost all value.  
  The KGB’s leverage dissolved the moment their threat became a gift.
  
  \medskip
  
  \textbf{The Lesson?}  
  Kompromat is only as powerful as its target’s willingness to protect their reputation.  
  When shame is absent—or reframed as pride—the blackmail engine stalls.
  
  \medskip
  
  \begin{quote}
  \textit{Control requires leverage. But leverage requires the other party to care.}
  \end{quote}
  
  In corporate settings, kompromat can backfire spectacularly if the intended target reframes the narrative—or if exposure turns the coercer into the exposed.
  
\end{tcolorbox}

