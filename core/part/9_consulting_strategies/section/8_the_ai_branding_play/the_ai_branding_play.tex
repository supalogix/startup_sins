\section{The AI\texttrademark{} Branding Play: If It Does Math, It’s Now Artificial Intelligence}

\begin{quote}
Linear regression? AI. Logistic regression? Also AI. CountIf in Excel? Close enough—call it “edge computing.”
\end{quote}

  In the age of \textbf{AI\texttrademark{} Branding}, it’s not about what the tool \textit{does}—it’s about what it \textit{looks like} it does.
  
  \medskip
  
  \textbf{Law 37} from \textit{The 48 Laws of Power} explains why every spreadsheet formula is now ``AI-powered'':
  \begin{quote}
  ``Striking imagery and grand symbolic gestures create the aura of power. People will believe what they see before they believe what is rational.''
  \end{quote}
  
  \medskip
  
  That’s why simple regression models are rebranded as \textit{``machine learning solutions''}, and automated Excel sheets become \textit{``AI-driven decision platforms.''}
  
  \medskip
  
  It’s not about algorithms—it’s about the \textbf{spectacle}: \\
  Glossy dashboards, futuristic buzzwords, and just enough mystery to make stakeholders nod in awe.
  
  \medskip
  
  If removing the word ``AI'' from the pitch makes it sound ordinary, that’s because it \textbf{is} ordinary. \\
  But ordinary doesn’t close deals—\textbf{spectacle} does.
  
  \medskip
  
  \textbf{Remember:} When consultants or vendors lean heavily on AI branding, they’re not selling intelligence. They’re selling \textbf{the illusion of innovation}.
  


\ExecutiveChecklist{medium}{AI Branding Detox}{
  \item Ask: “Would this still work if we removed the word ‘AI’?”
  \item Require a comparison to traditional methods (regression, heuristics, etc.).
  \item Don’t pay for branding—pay for results.
  \item If Excel could do it, don’t call it disruption.
}





\subsection{Case Study: The Accidental Spokesperson (Neuralium Labs, 2024)}

Neuralium Labs had a problem: they had an “AI platform” that was really just a handful of Python scripts wrapped in a flashy dashboard. What they needed wasn’t better tech—it was better credibility.

Enter Dr. Elias Moretti.

Moretti wasn’t a corporate veteran. He wasn’t an engineer. He wasn’t even familiar with production-grade machine learning.  But he had something better:  

\begin{itemize}
  \item A PhD from a prestigious European university.
  \item A trail of peer-reviewed publications in theoretical AI ethics.
  \item Invitations to keynote at academic conferences.
\end{itemize}

He wasn’t qualified to vet Neuralium’s technology—but he \textit{looked} qualified.

\begin{tcolorbox}[colback=blue!5!white, colframe=blue!50!black, breakable, title={Historical Sidebar: Theranos — When “Experts” Couldn’t Vet the Box}]

In the 2010s, \textbf{Theranos} promised to revolutionize medical testing with a sleek machine that could run hundreds of diagnostics on a single drop of blood.

\medskip 

To outsiders, it seemed airtight:  

\medskip 

\begin{itemize}
  \item A charismatic founder in Steve Jobs cosplay.
  \item A board stacked with generals, diplomats, and former cabinet members.
  \item Industry awards, magazine covers, and keynote stages.
\end{itemize}

\medskip 

But there was a catch.

\medskip 

Few of the “experts” in Theranos’s orbit had any background in biomedical engineering, laboratory science, or diagnostics.  The board was composed of political power brokers—not clinical scientists.  The advisory councils included legal and business luminaries—not technical validators.

\medskip

\textbf{The illusion:} A machine so advanced, only a visionary could understand it.

\medskip

\textbf{The reality:} A black box hiding manual lab work, faulty devices, and manipulated data.

\medskip

Theranos didn’t just sell a product—it sold an \textit{image of expertise}.  Each new “expert” endorsement reinforced the belief that scrutiny had already happened.  By the time skeptics started asking technical questions, the spectacle had already won.

\medskip

\begin{quote}
\textbf{The Lesson?} If no one on the board can open the black box, they’re not guarding innovation—they’re guarding a stage prop.
\end{quote}

\end{tcolorbox}

Neuralium didn’t even hire him directly. Instead, a network of “friends of the company”—investment partners, VC advisors, friendly adjacent firms—began inviting Moretti into their orbit:
\begin{itemize}
  \item A consulting retainer here.
  \item An “advisory board” title there.
  \item A guest appearance on an industry podcast.
  \item A panelist seat at a glitzy tech summit.
\end{itemize}

Every opportunity came with perks:  Wine tastings. Five-star hotels. Invitations to private club dinners. A vacation package “sponsored” by an angel investor’s family office.

And on those trips—always discreetly, never directly—other “perks” arrived:

\begin{quote}
Late at night, an unexpected knock at the hotel door.  
A woman standing there, smiling softly:  \textit{“I was told to bring you a little gift tonight.”}

A handwritten note slipped under the threshold a few minutes later, reading:  
\textit{“Hope you’re finding the delivery satisfactory. Let us know if you need... additional support.”}

And a coy remark the next morning over breakfast:  
\textit{“Heard you had some company last night. Funny how things like that always seem to happen around here.”}
\end{quote}


No direct quid pro quo.  No explicit instructions.  But Moretti got the message:  \textit{This is your world now. Be smart enough not to ask how it works.}

\begin{tcolorbox}[colback=blue!5!white, colframe=blue!50!black, breakable,
  title={Historical Sidebar: The SoftBank Vision Fund Blackmail Plot (2015)}]

In 2015, \textbf{Rajeev Misra}, the head of SoftBank’s Vision Fund, allegedly orchestrated a covert plot to undermine \textbf{Nikesh Arora}, SoftBank’s then-president and a rising star positioned as heir apparent to founder Masayoshi Son.

\medskip

According to multiple reports, Misra hired intermediaries to lure Arora into a compromising situation:  

\medskip

\begin{itemize}
  \item A Tokyo hotel room.
  \item Several women arranged to meet him.
  \item Hidden cameras set up to capture incriminating footage.
\end{itemize}

The plan?  Use the footage as blackmail—forcing Arora’s resignation and clearing Misra’s path to greater power within the company.

\medskip

But the scheme reportedly failed. Arora never took the bait. The plot came to light only later through internal investigations and media reports.

\medskip

\textbf{The illusion:} A boardroom rivalry won through corporate strategy.

\medskip

\textbf{The reality:} A backroom game of espionage, manipulation, and attempted entrapment.

\medskip

In high-stakes corporate settings, the power struggle isn’t always played out in quarterly reports or press releases. Sometimes it happens in whispered deals, shadowy setups, and schemes designed to destroy not just reputations—but futures.

\medskip

\begin{quote}
\textbf{The Lesson?} When the perks start arriving unasked, and the invitations seem too good to be true, it’s not networking—it’s grooming. And the next step might not be a promotion, but a trap.
\end{quote}

\end{tcolorbox}


When reporters and analysts called to inquire about Neuralium’s mysterious AI engine, Moretti—now proudly listed as an “AI Ethics Advisor” on the company’s website—was happy to respond:
\begin{quote}
It’s the most innovative system I’ve seen in the space. A real breakthrough in applied machine learning. I’m honored to advise such a forward-thinking team.
\end{quote}

Privately, he didn’t know what the system actually did. He’d never seen the source code. He wasn’t even sure the company had a product. But the perks were good, the title looked great on his CV, and nobody pushed him to verify anything.

\medskip

\textbf{Law 37} was alive and well:
\begin{quote}
“Striking imagery and grand symbolic gestures create the aura of power. People will believe what they see before they believe what is rational.”
\end{quote}

In this case, the striking image wasn’t a product demo—it was a credentialed academic sitting next to a neon “AI” sign at a cocktail reception, smiling politely while the CEO gestured at a black box no one dared open.

\medskip

\textbf{The Takeaway:}  
When credibility is built by proximity, not proof, you’re not witnessing validation—you’re witnessing stagecraft.  
And when the “expert” doesn’t check the black box, it’s not because they forgot. It’s because they’ve already been shown what happens when they ask.
