\section{When Startups Become Cartels: Power Consolidation in Plain Sight}

\vfill

\begin{figure}[H]
  \centering
  
  % === First row ===
  \begin{subfigure}[t]{0.45\textwidth}
  \centering
  \begin{tikzpicture}
    \comicpanel{0}{0}
      {Consultant}
      {Executive}
      {\small We’ve mapped a strategic synergy roadmap aligned with transformative KPIs.}
      {(-0.6,-0.6)}
  \end{tikzpicture}
  \caption*{The pitch: abstract nouns arranged in convincing order.}
  \end{subfigure}
  \hfill
  \begin{subfigure}[t]{0.45\textwidth}
  \centering
  \begin{tikzpicture}
    \comicpanel{0}{0}
      {Consultant}
      {Executive}
      {\small Fantastic. Can you explain it in plain English?}
      {(0.6,-0.6)}
  \end{tikzpicture}
  \caption*{The client is momentarily skeptical.}
  \end{subfigure}
  
  \vspace{1em}
  
  % === Second row ===
  \begin{subfigure}[t]{0.45\textwidth}
  \centering
  \begin{tikzpicture}
    \comicpanel{0}{0}
      {Consultant}
      {Executive}
      {\small Of course. It’s about leveraging holistic change to accelerate transformation synergies.}
      {(-0.6,-0.6)}
  \end{tikzpicture}
  \caption*{The consultant restates it using different buzzwords.}
  \end{subfigure}
  \hfill
  \begin{subfigure}[t]{0.45\textwidth}
  \centering
  \begin{tikzpicture}
    \comicpanel{0}{0}
      {Consultant}
      {Executive}
      {\small Brilliant. When can you start?}
      {(0.6,-0.6)}
  \end{tikzpicture}
  \caption*{The deal is sealed by sounding like you know what you’re doing.}
  \end{subfigure}
  
  \caption*{Consulting: the art of saying nothing so confidently that everyone hears something profound.}
  \end{figure}


  \subsection{The ``Technology Underbelly'': What Doesn’t Make the Pitch Deck}

  There’s a certain elegance in how the tech world operates.  
  Not elegance in the \textit{engineering} sense.
  No, this is the kind of elegance you find in stage illusions, casino tricks, or a con pulled off in broad daylight.
  
  The technology underbelly thrives at the intersection of \textbf{broken incentives}, \textbf{half-built systems}, and one enduring truth:  
  \textit{Nobody really knows how it works. They just hope it works long enough to cash out.}
  
  If you’ve ever read \textbf{\textit{The 48 Laws of Power}}, you’ll recognize the patterns:
  
  \begin{itemize}
    \item \textbf{Law 3: Conceal Your Intentions}
    \item \textbf{Law 6: Court Attention at All Costs}
    \item \textbf{Law 27: Play on People’s Need to Believe}
    \item \textbf{Law 45: Preach Change, But Never Reform Too Much at Once}
  \end{itemize}
  
  These aren’t just stray tactics—they’re baked into the fabric.  
  The investor decks. The product roadmaps. The “AI-powered” claims nobody checks too closely.
  
  \begin{itemize}
    \item Take a fragile prototype, cover it in buzzwords, and call it a platform.
    \item Build processes that only the founders understand, so no one can fire them.
    \item Redefine product-market fit as “whatever the last big customer said yes to.”
  \end{itemize}
  
  And when in doubt? Blame technical debt, praise the “move fast” culture, and remind everyone that  
  \textit{“in today’s fast-paced digital landscape, shipping is better than perfect.”}
  
  What the SEC doesn’t write about.  

  What the press releases won’t say.  

  What’s left out of the glossy product review.
  
  That’s the underbelly.  

  And sometimes, it’s the only real thing holding the whole thing together.
 
\medskip

\begin{HistoricalSidebar}{How Cynicism Became a Business Model}

  Robert Greene didn’t start out trying to write a guide to power.  He started out trying to survive it.

  \medskip
  
  In the 1990s, while working in Hollywood and media production, Greene saw up close how success actually operated.  It wasn’t about servant leadership. It wasn’t about humility.  It was about leverage, illusion, and the careful orchestration of appearances.

  \medskip
  
  One day, while working at a media lab in Italy, Greene voiced his jaded views about leadership to a Dutch publisher named Joost Elffers.  He argued — bluntly — that powerful people don't play by the rules they teach others.  They weaponize the rules.

  \medskip
  
  Elffers immediately saw the potential.  Here was a philosophy that cut through the polite fictions of business books and self-help seminars — raw, unsentimental, and disturbingly accurate.

  \medskip
  
  Elffers convinced Greene to turn his worldview into a book, funded its development, and helped bring it to life.

  \medskip
  
  The result was \textbf{\textit{The 48 Laws of Power}} (1998): a work so brutally honest about human nature that it became an underground classic in boardrooms, backrooms, and battlefields alike.

  \medskip
  
  Greene didn’t invent tech culture.  He just wrote down the rules everyone was already following, but no one wanted to admit.
  
\end{HistoricalSidebar}

\medskip

In this guide, I’m going to show you exactly how this game is played. We’ll dissect the strategy and tactics.
Not to admire them, but so you’ll recognize when you're buying 
\textbf{well-dressed ambiguity}.

Welcome to the backstage tour of the technology underbelly.


\subsection{Power Is Not Personal. It’s Institutional}

If you want to understand how the technology underbelly operates, you can’t just look at people.  
You have to look at structures.

Because power, in modern systems, is not wielded at the individual level.  
It’s wielded at the institutional level.

This is the heart of postmodernism.

Modernism — the philosophical engine behind Enlightenment thinking, rationalist politics, and early capitalism — was built 
on a hopeful idea:  
that humans could discover objective truth through reason, science, or lived experience.  
It was the intellectual core of secular humanism.  
And for a time, it worked. It built bridges, vaccines, and moral frameworks that are not based on religion.

But over time, that faith began to erode. However, it was not tools that failed. It was the institutions that failed.

By the late 20\textsuperscript{th} century, philosophers like Michel Foucault and Jacques Derrida began asking a more 
disturbing question:  
What if the “truths” we believe aren’t the product of reason or experience at all?  
What if they’re the product of power?

Foucault’s argument was simple, but radical:  
We don’t believe things because they’re true.  
We believe them because someone with power needs us to.

Schools, hospitals, prisons, media companies, and scientific institutions are not just part of the world.  
They produce the frameworks we use to understand it. They manufacture the categories --- sane/insane, normal/deviant, 
legal/illegal --- that shape our sense of what is “real.”

Power, in his view, wasn’t just coercion. It was invisible architecture.  
It didn’t shout. It whispered.

Derrida took a different but related approach.  
He saw language --- the very words we use to think --- as layered with assumptions that needed to be \textbf{deconstructed}.  
Thus language needed to be unpacked and examined.  
His work gave us tools to reveal how ideologies hide inside definitions, binaries, and ``common sense.''

Together, their project wasn’t nihilism.  
It was diagnosis.  
It was a way to see through the surface of claims (whether corporate, academic, religious, or political). 
And It was a way to understand the machinery behind them.

\medskip

\begin{HistoricalSidebar}{Nietzsche and the Misunderstanding of Nihilism}

  When Nietzsche wrote “God is dead” in \textit{The Gay Science} and again in \textit{Thus Spoke Zarathustra}, 
  he wasn’t being provocative for its own sake.  
  He wasn’t saying God had died in some literal or biological sense.  
  He was diagnosing something deeper: \textbf{we had killed God in our minds}.

  \medskip
  
  The Enlightenment had replaced theism with secular humanism: science, reason, and natural rights.  
  But it quietly kept the moral scaffolding of Christianity: the idea that human life had dignity, that truth mattered, 
  and that justice was real.  
  Nietzsche’s warning was simple: \textit{You cannot throw out God and keep everything God created.}

  \medskip
  
  The ``madman'' character who declares God's death isn’t celebrating. He’s horrified.  
  The ``madman'' saw what most of his contemporaries didn’t: that Western civilization still leaned on 
  moral claims inherited from a theological worldview, but without the metaphysical structure to support them.
  
  \medskip
  
  For example, democracy itself, Nietzsche understood, had theological roots.

  \medskip
  
  As John Locke argued in ``Second Treatise of Government'', all men are created equal because they are equally 
  responsible to God.  
  A king is not ontologically better than his subjects. He is only functionally different. It is like a husband 
  to a wife.  
  This was the philosophical spine of Jefferson’s Declaration of Independence:  
  If a king fails in his divinely appointed duties, his subjects — like a neglected wife — has a God-given right 
  to divorce him.

  \medskip
  
  But what happens when God doesn’t exist?

  \medskip
  
  Then the foundation of democratic equality becomes less self-evident.  
  Then rights are no longer inalienable. They are preferences that are up for negotiation or erasure.  
  Then power is no longer restrained by moral absolutes. It is only restrained by who holds the pen.
  
  \medskip
  
  Nietzsche was not a nihilist.  
  He feared nihilism. He feared the void left behind when the foundations inherited from Christianity collapse.  
  And he knew it was coming.

  \medskip
  
  His answer was the concept of the \textbf{Ubermensch} or the ``Superman''. 
  The Superman is not as a tyrant. 
  The Superman is a creature who could shoulder the burden of God after the death of God.

  \medskip
  
  The post-modernists picked up where Nietzsche left off.

  \medskip
  
  They didn’t deny the problem.  
  They tried to live in it. 
  They tried to make sense of meaning after the death of its author.

  \medskip
  
  That’s why post-modernism is often called \textbf{post-Enlightenment}.  
  It is not rebellion for rebellion’s sake.  
  It is what comes \textit{after} the gods are gone, the myths no longer work, and we still have to 
  continue living.
  
\end{HistoricalSidebar}

\medskip

This is where our current cultural flashpoints begin.

The word \textit{``woke''}, long before it became a political football, meant something very simple:  
To be \textbf{awake} enough to see what’s really happening behind the performance.

The phrase traces back to the 1930s, and to the African-American musician and activist Lead Belly.  
In one version of his protest songs titled ``Scottsboro Boys'',
he urged listeners to ``stay woke''. \footnote{In the lyrics, he warns Black audiences to “stay woke” and 
watch out for injustice, particularly from law enforcement and the courts. It became an early expression of 
political consciousness in the face of systemic racism, 
decades before the phrase was revived in modern discourse.}
He wanted everyone to stay alert to injustice that hid beneath the 
surface of legal proceedings.

\medskip

\begin{HistoricalSidebar}{The Scottsboro Boys}

  In 1931, nine Black teenagers were accused by two white women of rape in Scottsboro, Alabama.
  
  \medskip
  
  There was no evidence. One of the women, Ruby Bates, later recanted her testimony entirely.  
  But within days, all nine boys had been indicted by an all-white jury. Eight were sentenced to death.
  
  \medskip
  
  The case became a national and international scandal, exposing not just racial prejudice, but something more structural:  
  \textbf{Institutional Racism}.
  
  \medskip
  
  After the first trials, the U.S. Supreme Court intervened in \textit{Powell v. Alabama} (1932), ruling that the boys 
  had been denied their constitutional right to effective counsel.  
  The local courts responded by staging new trial with legal formalities now technically observed, but the 
  verdicts already preordained.
  
  \medskip
  
  When the defense produced exculpatory evidence and Bates testified for the defense, the jury convicted anyway.  
  The judge sentenced them to death... again.
  
  \medskip
  
  In 1935, the Court intervened a second time, in \textit{Norris v. Alabama}, finding that Black citizens had been 
  systematically excluded from jury service.  
  But even that decision didn’t end the trials. Alabama simply reshuffled the process, swapping judges and dragging 
  retrials across multiple counties.
  
  \medskip
  
  Some of the boys were held in prison for over a decade. Haywood Patterson escaped and was later convicted of 
  manslaughter in a separate incident.  
  Clarence Norris — the last surviving defendant — was finally pardoned in 1976.  
  The state of Alabama didn’t issue a collective posthumous pardon until 2013.
  
  \medskip
  
  Their trials were public. 
  The transcripts were official. 
  The injustice was documented.
  And that’s what makes it terrifying.
  
  \end{HistoricalSidebar}
  

\medskip

Here, the intellectual scaffolding of thinkers like \textbf{Michel Foucault} and \textbf{Jacques Derrida} becomes crucial.  
They didn’t invent the word, but they gave us the tools to understand what it was pointing at.

Foucault taught us that \textit{power isn’t just enforced through force}, but through norms, institutions, language, and classification — what he called \textbf{regimes of truth}.  
Derrida showed that \textit{meaning isn’t fixed}, and that every text — whether a legal code or a cultural script — contains absences, contradictions, and buried assumptions.

Together, they shifted the lens:  
Instead of asking ``What is this law or policy saying?'', we start asking:  
\textbf{Who gets to speak? Who gets heard? What is being left unsaid?}

To be \textit{woke}, in its original sense, is not to be partisan.  
It is to be suspicious of easy narratives.  
It is to suspect that what looks ``neutral'' or ``natural'' may actually be the polished mask of something inherited, 
constructed, and deeply uneven.

Later, in the Civil Rights era and beyond, ``stay woke'' evolved into a broader cultural shorthand:  
a reminder that what looks like ``progress'' might be something else entirely.

That’s what we’re doing here.

We are not criticizing the world. We are examining the structures that taught us what it means to live in the world, 
and who benefits when we do it without question.

This isn’t about cynicism.

It’s about waking up.


\subsection{Edutainment: When Storytelling Becomes Infrastructure}

If power hides in plain sight, so can pedagogy.

There’s a reason stories survive where syllabi don’t.
We evolved to tell them. Long before we built universities, we built campfires.
Long before we wrote whitepapers, we passed on cautionary tales, origin myths, and survival tricks wrapped in narrative.
Storytelling isn’t just how we entertain. It’s how we remember, how we relate, and how we learn.

That’s the real lesson behind the success of books like The Goal by Eliyahu Goldratt and The Phoenix Project by Gene Kim.
These weren’t textbooks. They didn’t start with definitions or frameworks or bulleted takeaways.
They told stories — full, human, and emotionally resonant stories — about factories and IT disasters and burned-out middle managers trying to make sense of chaos.

And in doing so, they pulled off something most academic work struggles to achieve.
They taught complex theories — like the Theory of Constraints and DevOps transformation — to people who didn’t know 
they were learning theory.

Their books became bestsellers. And it was not because they lowered the bar. 
It was because they disguised the bar as a plot point.

\medskip

\begin{HistoricalSidebar}{The Origins of Management Theory}

  Modern management theory was born on the factory floor.

  \medskip
  
  In the early 20\textsuperscript{th} century, thinkers like \textbf{Frederick Winslow Taylor} and \textbf{Henri Fayol} tried to systematize work the same way engineers systematized machines.  
  Taylor’s \textit{Scientific Management} reduced tasks into optimized, measurable motions.  
  Fayol laid out universal principles of planning, organizing, and controlling — the blueprints for the org chart.

  \medskip
  
  By mid-century, management had become a technocratic discipline.  
  MBA programs flourished. Strategic frameworks (SWOT, Porter’s Five Forces) promised analytical clarity.  
  PowerPoint replaced intuition. Flowcharts replaced experience.
  
  \medskip
  
  But something got lost.

  \medskip
  
  The human element — conflict, stress, error, improvisation — got pushed out of the frame.  
  Executives were taught how to structure work, but not how work actually feels.
  
  \medskip
  
  \textbf{Goldratt} and \textbf{Kim} kicked against this.

  \medskip
  
  Their books — \textit{The Goal} and \textit{The Phoenix Project} — didn’t read like textbooks.  
  They read like novels: stories of overwhelmed managers trying to rescue collapsing operations with limited time, 
  fragile egos, and unexpected allies.

  \medskip
  
  They taught theory not by explaining it, but by dramatizing it:  
  Bottlenecks. Constraints. Feedback loops. Cultural inertia.  
  All shown, not told.
  
  \medskip
  
  Where early management thinkers chased precision, Goldratt and Kim chased resonance.

  \medskip
  
  And in doing so, they proved something quietly radical:  
  That you could smuggle real operational insight into fiction, and that most people would learn more from the story than 
  they ever did from the syllabus.
  
\end{HistoricalSidebar}

\medskip

Academia largely ignored them. Management consultants dismissed their work as too simplistic, too anecdotal, and too populist.
But guess what?
Entire industries reorganized around their insights.
Operations managers, CTOs, and product leads started quoting lines from novels in board meetings.
Why? Because those stories stuck.

The truth is that expert knowledge isn’t inaccessible.
It’s just usually told badly.
What Goldratt and Kim proved is that pedagogy doesn’t have to sound like a textbook to be rigorous.
You don’t need to intimidate your reader to elevate them.

And that’s part of the structural irony.
The best way to teach someone something is to show them how someone like them struggles to learn it.

If you want to change a company then change the stories it tells itself.

If you want to educate at scale then don’t build a better curriculum. Build a better character arc.

Because sometimes, the difference between an unread policy binder and a cultural revolution is just a protagonist with a problem.