\section{When Startups Become Cartels: Power Consolidation in Plain Sight}

\vfill

\begin{figure}[H]
  \centering
  
  % === First row ===
  \begin{subfigure}[t]{0.45\textwidth}
  \centering
  \begin{tikzpicture}
    \comicpanel{0}{0}
      {Consultant}
      {Executive}
      {\small We’ve mapped a strategic synergy roadmap aligned with transformative KPIs.}
      {(-0.6,-0.6)}
  \end{tikzpicture}
  \caption*{The pitch: abstract nouns arranged in convincing order.}
  \end{subfigure}
  \hfill
  \begin{subfigure}[t]{0.45\textwidth}
  \centering
  \begin{tikzpicture}
    \comicpanel{0}{0}
      {Consultant}
      {Executive}
      {\small Fantastic. Can you explain it in plain English?}
      {(0.6,-0.6)}
  \end{tikzpicture}
  \caption*{The client is momentarily skeptical.}
  \end{subfigure}
  
  \vspace{1em}
  
  % === Second row ===
  \begin{subfigure}[t]{0.45\textwidth}
  \centering
  \begin{tikzpicture}
    \comicpanel{0}{0}
      {Consultant}
      {Executive}
      {\small Of course. It’s about leveraging holistic change to accelerate transformation synergies.}
      {(-0.6,-0.6)}
  \end{tikzpicture}
  \caption*{The consultant restates it using different buzzwords.}
  \end{subfigure}
  \hfill
  \begin{subfigure}[t]{0.45\textwidth}
  \centering
  \begin{tikzpicture}
    \comicpanel{0}{0}
      {Consultant}
      {Executive}
      {\small Brilliant. When can you start?}
      {(0.6,-0.6)}
  \end{tikzpicture}
  \caption*{The deal is sealed by sounding like you know what you’re doing.}
  \end{subfigure}
  
  \caption*{Consulting: the art of saying nothing so confidently that everyone hears something profound.}
  \end{figure}


  \subsection{The ``Technology Underbelly'': What Doesn’t Make the Pitch Deck}

  There’s a certain elegance in how the tech world operates.  
  Not elegance in the \textit{engineering} sense—where elegance means simplicity, efficiency, and robustness.  
  No, this is the kind of elegance you find in stage illusions, casino tricks, or a con pulled off in broad daylight.
  
  The technology underbelly thrives at the intersection of \textbf{broken incentives}, \textbf{half-built systems}, and one enduring truth:  
  \textit{Nobody really knows how it works. They just hope it works long enough to cash out.}
  
  If you’ve ever read \textbf{\textit{The 48 Laws of Power}}, you’ll recognize the patterns:
  
  \begin{itemize}
    \item \textbf{Law 3: Conceal Your Intentions}
    \item \textbf{Law 6: Court Attention at All Costs}
    \item \textbf{Law 27: Play on People’s Need to Believe}
    \item \textbf{Law 45: Preach Change, But Never Reform Too Much at Once}
  \end{itemize}
  
  These aren’t just stray tactics—they’re baked into the fabric.  
  The investor decks. The product roadmaps. The “AI-powered” claims nobody checks too closely.
  
  \begin{itemize}
    \item Take a fragile prototype, cover it in buzzwords, and call it a platform.
    \item Build processes that only the founders understand, so no one can fire them.
    \item Redefine product-market fit as “whatever the last big customer said yes to.”
  \end{itemize}
  
  And when in doubt? Blame technical debt, praise the “move fast” culture, and remind everyone that  
  \textit{“in today’s fast-paced digital landscape, shipping is better than perfect.”}
  
  What the SEC doesn’t write about.  
  What the press releases won’t say.  
  What’s left out of the glossy product review.
  
  That’s the underbelly.  
  And sometimes, it’s the only real thing holding the whole thing together.
 
\medskip

\begin{HistoricalSidebar}{How Cynicism Became a Business Model}

  Robert Greene didn’t start out trying to write a guide to power.  He started out trying to survive it.

  \medskip
  
  In the 1990s, while working in Hollywood and media production, Greene saw up close how success actually operated.  It wasn’t about servant leadership. It wasn’t about humility.  It was about leverage, illusion, and the careful orchestration of appearances.

  \medskip
  
  One day, while working at a media lab in Italy, Greene voiced his jaded views about leadership to a Dutch publisher named Joost Elffers.  He argued — bluntly — that powerful people don't play by the rules they teach others.  They weaponize the rules.

  \medskip
  
  Elffers immediately saw the potential.  Here was a philosophy that cut through the polite fictions of business books and self-help seminars — raw, unsentimental, and disturbingly accurate.

  \medskip
  
  Elffers convinced Greene to turn his worldview into a book, funded its development, and helped bring it to life.

  \medskip
  
  The result was \textbf{\textit{The 48 Laws of Power}} (1998): a work so brutally honest about human nature that it became an underground classic in boardrooms, backrooms, and battlefields alike.

  \medskip
  
  Greene didn’t invent tech culture.  He just wrote down the rules everyone was already following, but no one wanted to admit.
  
\end{HistoricalSidebar}

\medskip

In this guide, I’m going to show you exactly how this game is played. We’ll dissect the tactics—one buzzword, one dashboard, and one eternal proof of concept at a time. Not to admire them, but so you’ll recognize when you're buying \textbf{well-dressed ambiguity}.

Welcome to the backstage tour of the technology underbelly.

