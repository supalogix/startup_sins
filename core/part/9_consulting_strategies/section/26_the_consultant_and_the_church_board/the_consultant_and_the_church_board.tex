\section{The Consultant and the Church Board Club}

\subsection{Old Rules, New Startups}

Tyler Voss didn’t expect the offer.  
He had just wrapped a middling engagement in Atlanta when the email came:  
\textbf{``Looking for a discreet strategist to advise regional growth initiatives.''}

The sender was a senior partner at \textit{Holloway \& Reese}, a family office with a name that predated the internet and a portfolio that didn’t.  
No job title. No firm name.  
Just a reference to \textit{``a heritage-backed tech corridor forming in the Southeast.''}

The meeting wasn’t virtual.  
It was at a hunting lodge outside Montgomery, where the bourbon was local and the silence contractual.  
Dress code: collared shirt, slacks, no laptops on the table.

Three men waited inside. All from generational wealth.  
All managing directors of \textbf{Founders Glen}, a privately capitalized development trust turned “ecosystem accelerator.”

Officially, they were building a tech hub.  
Unofficially, they were fencing in the future.

Their kids had Ivy League MBAs.  
Their grandfathers had names on courthouses.

And while they envied Silicon Valley's velocity, they weren’t about to play by its rules.

They didn’t ask Tyler about term sheets or CAC.  
They asked if he could golf. If he owned a navy blazer. If he’d ever dated anyone from Nashville prep.

He had the sense to say yes.  
He also had the sense to lie.

\medskip

\begin{HistoricalSidebar}{Old Money vs New Money --- Different Gods, Same Mountain}

    \textbf{Old money plays for permanence.}  
    It values legacy, stability, and the preservation of structure over time.  
    It operates on inheritance, family names, and selective gatekeeping.

    \medskip
    
    \textbf{New money plays for dominance.}  
    It values meritocracy, velocity, and disruption.  
    It operates on founder mythology, personal branding, and tactical reinvention.

    \medskip
    
    The difference isn’t just in wealth.  
    It’s in \textit{philosophy}.

    \medskip
    
    \begin{itemize}
      \item Old money builds trusts.  
      \item New money builds apps.  
      \item Old money marries within the circuit.  
      \item New money date-hops across coasts and cap tables.  
    \end{itemize}

    \medskip
    
    Take \textbf{Bill Gates}.  
    He publicly stated his children would inherit only a fraction of his fortune.  
    Not because he doesn’t love them, but because he believes that status must be \textit{earned}, and not inherited.  
    None of his children ever worked at Microsoft.

    \medskip
    
    \textbf{Jeff Bezos} shares the same ethos.  
    His children have access... but not entitlement.  
    Because in new money culture, dynastic advantage is seen as moral hazard.
    
    \medskip
    
    \textbf{Alexander the Great} understood this better than most.  
    On his deathbed, when asked who should inherit his empire, he said only:
    
    \begin{quote}
        To the strongest.
    \end{quote}

    Old money sees this as dangerous.  
    A kingdom with no designated heir is one with inevitable succession wars.
    It wasn’t a blessing.  
    It was an invitation to chaos.
    
    \medskip
    
    And in today’s economy, that chaos is playing out across zip codes, boardrooms, and donor foundations — where  
    \textbf{old money builds legacy},  and \textbf{new money builds kingdoms it may never pass on.}

\end{HistoricalSidebar}



\subsection{The Families, the Pastor, and the Preserved Order}

The men who hired Tyler weren’t just investors.  
They were sons of the founding families — men whose names were stitched into the county seal and whose grandfathers still sat, embalmed in oil portraits, in the law library downtown.

Officially, they chaired boards: development councils, endowments, regional growth initiatives.  
Unofficially, they owned the zip code.

Nothing in the area moved without them.  
Not permits. Not newspaper editorials. Not church doctrine.

They didn’t just fund the ecosystem — they \textit{curated} it.  
The local newspaper ran op-eds they ghostwrote.  
The university hired who they recommended.  
The megachurch preached what they tolerated.

\textbf{Kingdom Harvest Fellowship}, the largest church in the region, wasn’t just a place of worship — it was their soft-power instrument.  
A pulpit wrapped in patriotism, prosperity gospel, and plausible deniability.

On paper, the church followed a \textit{congregationalist polity} —  
a structure where power ostensibly resided with the members.  
Budgets were approved by majority vote.  
Elders were elected by ballot.  
Pastors were “called” by a democratic process that echoed New England town meetings and Reformation ideals.

In theory, it was ecclesial democracy.  
In practice, it was obedience theater.

Because in this town, the major employers tithed in six figures.  
The founding families paid the building mortgage.  
And every deacon, music director, and Sunday school teacher knew that voting against the families’ preferred candidate  
was a fast way to lose your job — or your social standing.

Democracy doesn’t work when the ballot box is underwritten.

Everyone voted.  
But they voted as expected.  
Because maintaining consensus wasn’t about belief —  
it was about \textbf{belonging}.

Their pastor, \textbf{Reverend Cole}, had been hand-picked.

Not for his theology.  
But for his obedience.

The search committee had interviewed over a dozen candidates.  
There were seminary scholars, firebrand preachers, even a former military chaplain.

But Cole stood out.

Not because he was devout.  
Because he was \textit{complicit}.

He knew how to speak about righteousness without ever confronting power.  
He knew when to quote scripture and when to quote shareholders.  
And he understood that congregationalism in Kingdom Harvest wasn’t a doctrine.  
It was a ritual of controlled consent —  
a democracy that only worked when everyone remembered who built the building.


He and his wife, Miriam, had a history — tastefully contained, quietly documented, and never spoken aloud.  
The kind of history that made them manageable.

Not a threat.  
Not a liability.

Before Kingdom Harvest, they’d served at a mid-sized Southern Baptist church in the Piedmont.  
On Sundays, Cole preached family values.  
On weeknights, he and Miriam blurred them.

There were rumors, of course.  
Late-night counseling sessions that turned into long drives.  
Prayer retreats that required hotel rooms.  
A youth intern who left abruptly, her exit labeled “God’s redirection.”

Eventually, the whispers became too loud to ignore — not for the congregation, but for the men upstream.

The \textbf{Southern Baptist Convention} prefers to handle such things internally.  
Officially, there was no scandal.  
No statement.  
Just a letter of “release,” a recommendation that emphasized Cole’s “passion for growth” and “sensitivity to adult ministry.”

Unofficially, every leader in the SBC’s orbit knew what it meant:  
A quiet exile.  
A discreet offloading of risk.

They didn’t want blood.  
They wanted distance.

And Cole knew how to walk away clean — as long as he didn’t ask questions and didn’t make noise.

By the time the families at Kingdom Harvest interviewed him, Cole was already a known quantity.  
Not unknown.  
Just well-understood.

His file had circulated.  
Not through HR, but through prayer breakfasts, golf retreats, and whisper-networks of denominational influence.  
What mattered wasn’t what he did.  
It was that he didn’t deny it.  
And that he could be trusted never to do so publicly.

That’s what made him perfect.

He wasn’t holy.  
He was \textit{containable}.

\medskip

\begin{HistoricalSidebar}{The Southern Baptist Shuffle: How Institutions Prefer Containment Over Discipline}

    In theory, the Southern Baptist Convention (SBC) operates under a decentralized structure.  
    Each church is autonomous.  
    Each pastor answers to their congregation, not to a bishop or denominational board.

    \medskip
    
    In practice, the SBC functions through informal power —  
    a tight weave of seminaries, donor networks, missions boards, and legacy families that manage reputation with more urgency than righteousness.
    
    \medskip
    
    When a pastor crosses a line — morally, sexually, financially — there is no trial, no public reckoning.  
    Instead, there is the \textbf{shuffle}:

    \medskip
    
    \begin{itemize}
      \item A quiet resignation framed as “seeking new opportunities.”  
      \item A letter of recommendation that omits the real reason for departure.  
      \item A lateral move to a smaller church, often in a different state, where the backstory becomes “God refining his servant.”  
    \end{itemize}

    \medskip
    
    The goal isn’t justice.  
    It’s containment.
    
    \medskip
    
    This system protects more than reputations.  
    It protects \textbf{denominational stability}.  
    Every scandal is a potential fracture — a headline that could disrupt giving, or weaken claims to moral authority.

    \medskip
    
    So the SBC leadership rarely confronts misconduct head-on.  
    It redirects.  
    It obscures.  
    It prays for “restoration” in place of investigation.
    
    \medskip
    
    Over time, this creates a caste of \textbf{known but tolerated men} —  
    pastors with records, not rap sheets; histories, not charges.  
    They cycle through pulpits like flawed assets in a clerical hedge fund — too exposed to lead nationally, too useful to discard entirely.

    \medskip
    
    \textit{They are not rehabilitated.  
    They are recycled.}
    
\end{HistoricalSidebar}




\medskip

\begin{HistoricalSidebar}{When Dependency Gets Personal --- The Oracle Consultant Allegations}
  
  In the early 2010s, multiple lawsuits surfaced from female sales consultants contracting with \textbf{Oracle}.  
  
  \medskip
  
  The allegations were striking: Oracle managers allegedly created a toxic sales culture where inappropriate behavior blurred into business pressure.  If consultants wanted to \textit{keep} lucrative licensing deals—or \textit{win} new ones—they were expected to ``play along'' with advances and tolerate harassment.
  
  \medskip
  
  \begin{quote}
  The implied contract for female sales consultants was clear: \textbf{Access to deals required access to you}.
  \end{quote}
  
  \medskip
  
  Most cases were quietly settled, but the underlying dynamic became a cautionary tale in broader tech industry reports.  The idea that vendor and supplier relationships could be tainted by \textbf{quid pro quo} misconduct sharpened scrutiny of corporate sales environments (especially those fueled by rapid revenue growth at all costs).
  
  \medskip
  
  For anyone surprised by these revelations, Oracle's culture was hardly a secret.  As chronicled in Mike Wilson's book, \textit{``The Difference Between God and Larry Ellison: God Doesn't Think He's Larry Ellison''}, Oracle’s founder cultivated a mythos of power, control, and strategic aggression.  The sales floor, unsurprissingly, had simply followed his lead.
  
  \medskip
  
  \begin{quote}
  \textbf{The Lesson?} In some vendor relationships, the real lock-in isn't technical. It's personal, and it costs more than money to maintain.
  \end{quote}
  
\end{HistoricalSidebar}

\medskip

\begin{HistoricalSidebar}{The Velvet Cross — When Charisma Becomes a Gate to Controlled Transgression}

In the modern South, the megachurch is a hybrid institution: part temple, part television studio, part corporate lobbying arm.  

Its board is often indistinguishable from the local chamber of commerce. Its budget rivals a small municipality.  
And its pastor, if carefully chosen, becomes a spiritual emissary with a curated past and a board-approved future.

\begin{itemize}
  \item The pulpit preaches fidelity.  
  \item The green room negotiates indulgence.  
  \item The lifestyle is managed like a hedge fund — reputational risk offset by institutional loyalty.
\end{itemize}

The key isn’t secrecy. It’s \textbf{containment}.  
And the best pastor isn’t the one who’s pure.  
It’s the one who can sin… and not talk.

\end{HistoricalSidebar}

\subsection{The Church Visit That Wasn’t About Faith}

Tyler didn’t go to Kingdom Harvest on his own.

One of the executives — Collin, the one with the turquoise cufflinks and four generations buried in the same cemetery — had suggested it offhandedly during lunch.

\textit{“You’ll want to meet Cole sooner rather than later. He likes to get a feel for new faces.”}

So Tyler put on a blazer, borrowed a truck, and showed up for the 9:30am service.

The church was massive — LED screens, polished stone, a welcome team that felt suspiciously like a brand activation squad.  
The worship band opened with a country-tinged arrangement of a Hillsong anthem. The lyrics were universal. The style was not.

What stood out wasn’t the lighting.  
It was the composition.

\textbf{The congregation was almost entirely white.}  
Not incidentally, but unmistakably.

There were a handful of Black men, all his age, all strikingly attractive, dressed in coordinated earth tones like they were on rotation for a catalog shoot.  
Not a single Black family.  
No grandparents. No toddlers. No aunts or uncles chasing children with purse mints.  
Just the right kind of diversity — visible, curated, deniable.

Tyler wasn’t shocked. But he was watching.

\medskip

After the sermon — forty minutes of prosperity theology wrapped in Scripture and Texas charm — Tyler was ushered into the pastor’s lounge.

\textbf{Reverend Cole} met him with a smile that didn’t quite reach the eyes.  
He wore a trim navy jacket and spoke with the polished ease of someone who’d done donor dinners in five states.

“Glad you came,” he said, extending a hand.  
“Always good to meet new energy in the region.”

Tyler introduced himself as a strategist.  
Cole nodded. “Founders Glen told me you were the quiet kind. We like quiet kinds.”

They sat in leather chairs beneath a mounted cross and a taxidermied buck.

Sweet tea was poured. No alcohol. Not yet.

Cole’s tone softened.

\textit{“So, Tyler… you good at keeping confidences?”}

Tyler didn’t flinch.  
\textit{“I don’t gossip.”}

Cole smiled, satisfied.

\textit{“That’s what I like about consultants. You’re paid to forget.”}


\medskip

\begin{HistoricalSidebar}{Ravi Zacharias --- The Fall of a Trusted Voice}

    Ravi Zacharias was once celebrated as a leading Christian apologist, revered for his eloquence and intellectual defense of the faith. However, posthumous investigations unveiled a disturbing pattern of sexual misconduct and spiritual abuse that starkly contrasted his public persona.
    
    \medskip
    
    A comprehensive report by Miller \& Martin, commissioned by Ravi Zacharias International Ministries (RZIM), revealed that Zacharias exploited his spiritual authority to manipulate and abuse numerous women. One victim recounted how Zacharias coerced her into sexual acts, framing them as divine blessings and warning that any disclosure would jeopardize countless souls by tarnishing his ministry's reputation.
    
    \medskip
    
    Another victim, Vicki Blue (a business partner), disclosed that he threatened to ``ruin'' her and her daughter if they spoke out about their sexual abuse. Blue stated, “Ravi kept talking about an anonymous donor that he could get limitless money from. He said, ‘I can keep it going and make your life miserable until you die.’” \cite{blue2021}
    
    \medskip
    
    The Zacharias case serves as a sobering reminder that charisma and public acclaim can mask profound moral failings.
    
    \begin{thebibliography}{9}
    \bibitem{blue2021}
    Julie Roys, \textit{Spa Co-Owner \& Victim of Ravi Zacharias Speaks: He Threatened to “Ruin” Me}, The Roys Report, April 13, 2021. \url{https://julieroys.com/victim-of-ravi-zacharias-speaks-he-threatened-to-ruin-me/}
    \end{thebibliography}
\end{HistoricalSidebar}
    
\medskip


\subsection{The Bull by Design}

The engagement launched quietly.  
No kickoff call. No roadmap deck.  
Just a six-month retainer and a house on the lake, compliments of the Bellwether leadership team.

By week two, Tyler had been introduced to three wives.  
By week four, he had been “prayed over” by Miriam with her hand on his thigh.  
By week six, he’d been invited to a Wednesday night “study group” that began with worship... and ended in bedrooms.

The men called it \textbf{``stewardship of desire.''} The wives just called it \textbf{``fellowship.''}

\begin{PsychologicalSidebar}{Bull Theology — Sexual Roles and Power Displacement in the Christian Lifestyle Scene}

In underground Southern polyamorous communities, especially among public Christians, the “bull” serves a paradoxical role:

\medskip

\begin{itemize}
  \item He is \textbf{masculine}, but \textbf{disposable}.  
  \item He is \textbf{sexually dominant}, but \textbf{socially submissive}.  
  \item He brings energy to the marriage… but no authority to the church.
\end{itemize}

\medskip

In this inversion, the bull functions as \textit{both scapegoat and sacrament}:  
He carries the temptation.  
He absorbs the transgression.  
He lets the couple keep their vows in public — while breaking them in curated, deniable private.

\medskip

He is not an outsider.  
He is a \textbf{mechanism}.

\end{PsychologicalSidebar}

\subsection{The Sunday Illusion}

On Sundays, the couples dressed the part.

Bellwether execs in pressed suits. Their wives in modest dresses and subtle diamonds. Tyler sat in the back, unintroduced. Invisible.

The sermon was about \textit{``discipline in times of testing.''}  
Reverend Cole spoke of Abraham. Of obedience. Of sacrifice.

And Tyler thought:  
\textit{You’re not preaching against sin. You’re narrating it.}  
And everyone in the front row already knows the ending.

\begin{HistoricalSidebar}{The Pool Boy and the Prince of Purity}

    Jerry Falwell Jr. was evangelical royalty.

    \medskip
    
    The son of \textbf{Jerry Falwell Sr.}, founder of the Moral Majority and spiritual architect of the modern Christian Right, he inherited not just a name — but an empire.

    \medskip
    
    At the height of his influence, Falwell Jr. controlled \textbf{Liberty University}, one of the largest and most powerful Christian universities in the United States.  
    With over 100,000 students, a billion-dollar endowment, and deep ties to Republican politics, Liberty wasn’t just a school — it was a factory of conservative leadership, churning out lawyers, pastors, and culture warriors under the banner of \textit{“training champions for Christ.”}

    \medskip
    
    Publicly, Falwell enforced a rigid code of sexual conduct:

    \medskip
    
    \begin{itemize}
      \item Strict abstinence policies.  
      \item Dress codes for women.  
      \item Expulsion for same-sex relationships.  
    \end{itemize}

    \medskip
    
    But in private, his theology was… flexible.
    
    \medskip
    
    In 2020, Falwell was exposed in a long-standing sexual relationship — not directly, but as an \textbf{enthusiastic voyeur}.  
    His wife, Becki Falwell, had carried on an affair for years with a younger man named \textbf{Giancarlo Granda}, a former pool attendant at a Miami resort.

    \medskip
    
    According to Granda, Falwell often watched.

    \medskip
    
    The scandal broke publicly when Granda came forward with photos, text messages, and details that painted a vivid picture of a marriage structured less around monogamy and more around managed transgression.
    
    \medskip
    
    The reaction was swift but not surprising:

    \medskip
    
    \begin{itemize}
      \item Falwell initially denied, then admitted parts of the story.  
      \item Liberty's board placed him on indefinite leave.  
      \item He resigned with a severance package rumored to exceed \$10 million.  
    \end{itemize}

    \medskip
    
    \textbf{What made it scandalous wasn't the sex.}  
    It was the hypocrisy.

    \medskip
    
    Falwell wasn't just a Christian leader.  
    He was the guardian of moral discipline — punishing students for behavior that mirrored his own household rituals.
    
    \medskip
    
    \textit{In evangelical politics, sin is survivable.  
    But contradiction?  
    That's fatal.}

\end{HistoricalSidebar}


\begin{PsychologicalSidebar}{Polyamory, Prestige, and the Optics of Controlled Transgression}

    In elite circles, the appearance of control matters more than actual boundaries.
    Polyamory becomes not just a relationship structure but a performance of trust, fluidity, and modernity — as long as it’s kept off the front page.
    
    The presence of a “bull” — a charismatic, unattached outsider brought in for sexual novelty — can be simultaneously taboo and status-enhancing.
    
    \begin{itemize}
    \item Taboo because it breaks monogamous norms.
    \item Status-enhancing because it signals control, mutual consent, and discretionary access.
    \end{itemize}
    
    But here’s the twist:
    
    \textbf{The bull is not just a guest.}
    He’s a release valve.
    He’s the insurance policy against stagnation, against scandal, against real emotional entanglement between the partners themselves.
    
    He’s valuable precisely because he can be replaced.
    
\end{PsychologicalSidebar}

\subsection{The Breakdown in Private Rooms}

Eventually, things shifted.

Not because of guilt.  
Because of politics.

Miriam began sending shorter texts.  
Sabine made a passive-aggressive comment about “being treated like rotation.”  
Laurel’s husband — an elder with a temper — called Tyler “arrogant” during a church offsite.

The next day, Tyler’s retainer wasn’t renewed.

No reason given.

Just a voicemail from HR with a bless-your-heart tone and a recommendation letter filled with euphemisms:

\textit{“Tyler was highly effective in high-touch, intimate engagements requiring discretion and relational flexibility.”}

\begin{HistoricalSidebar}{Consulting as Cover: The Dual Function of External Advisors}

    In elite institutions, “consultant” is a job title, but it’s also a decoy.
    A label that grants access without accountability.
    
    External advisors are brought in not just for expertise — but for their disposability.
    They serve as:
    
    \begin{itemize}
    \item Intellectual fig leaves (“We brought in outside experts.”)
    \item Strategic mirrors (“Here’s what you already believe, but with slides.”)
    \item Social lubrication (“He’s not staff. He’s flexible.”)
    \end{itemize}
    
    In environments where hierarchy is rigid but boundaries are porous, the consultant becomes the perfect container for transgressive intimacy — especially if they don’t have HR protections, voting rights, or subpoena relevance.
    
    Because when the questions come later, the answer is always the same:
    
    \textit{“He was just a contractor.”}
    
\end{HistoricalSidebar}

\subsection{The Final Prayer Circle}

The last time Tyler saw the group was at a silent prayer circle during revival week.

The pastor laid hands on the congregation.

The wives sang in soft harmony.

And Tyler — watching from the back row, in pressed khakis and an untucked shirt — realized the most dangerous men weren’t the ones who sinned.  
They were the ones who choreographed it.

\begin{HistoricalSidebar}{The Southern Mask — Moral Theater as Power Structure}

In the post-evangelical South, reputation is currency.  
Church is not just a place of worship.  
It is the public stage upon which deals are struck, affairs are sanitized, and social capital is managed.

\begin{itemize}
  \item The pulpit becomes a firewall.  
  \item The congregation becomes a blind.  
  \item The “lifestyle” becomes a doctrine of exception, practiced only by those with enough influence to make sin look like providence.
\end{itemize}

The bull, the pastor, the wives — they are not outliers.  
They are \textit{instrumental}.  
And the greatest heresy is not desire.  
It is breaking the illusion that keeps it sacred.

\end{HistoricalSidebar}

\begin{quote}
Because in the Bible Belt,  
what damns you isn’t the sin.  
It’s the lack of discretion.
\end{quote}
