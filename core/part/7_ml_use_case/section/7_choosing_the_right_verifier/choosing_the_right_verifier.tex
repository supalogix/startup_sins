\section{Choosing the Right Verifier (When Risk Has a Shape)}

Verification isn’t a one-size-fits-all task. The right tool depends on what kind of error you're guarding against, how fast you need an answer, and what complexity you're willing to entertain.

In this section, we’ll compare the major verification tools used for neural networks — each with its own philosophy, trade-offs, and domain of relevance.

\subsection{Reluplex — Precision in the Piecewise}

\textbf{Reluplex} was one of the first solvers designed specifically for ReLU-activated networks. Built on top of the classical simplex algorithm, it adds support for piecewise-linear constraints induced by ReLU activations.

\textbf{Strengths}:
\begin{itemize}
  \item Precise handling of linear and piecewise-linear logic.
  \item Sound and complete (for the networks it can handle).
\end{itemize}

\textbf{Limitations}:
\begin{itemize}
  \item Poor scalability — exponential blow-up with network size.
  \item Not well-suited for deep or convolutional networks.
\end{itemize}

Use Reluplex when:
\begin{quote}
You need exact verification on small, fully connected feedforward networks — especially in safety-critical systems where false negatives are unacceptable.
\end{quote}

\subsection{Marabou — Modular, Scalable, SMT Hybrid}

\textbf{Marabou} evolves Reluplex into a more modular SMT-style solver. It separates constraint solving from network architecture and adds better heuristics for case splitting.

\textbf{Strengths}:
\begin{itemize}
  \item More scalable than Reluplex.
  \item Modular design supports new activation functions and optimizations.
\end{itemize}

\textbf{Limitations}:
\begin{itemize}
  \item Still exponential in worst-case scenarios.
  \item Less precise in floating-point domains.
\end{itemize}

Use Marabou when:
\begin{quote}
You want a more flexible, extensible platform for verifying mid-sized networks — or when experimenting with different input domains and perturbation models.
\end{quote}

\subsection{AI\textsuperscript{2} and Planet — Overapproximation at Scale}

\textbf{AI\textsuperscript{2}} (Abstract Interpretation for AI) and \textbf{Planet} take a different approach: they use abstract domains or convex relaxations to overapproximate the behavior of ReLU networks.

\textbf{Strengths}:
\begin{itemize}
  \item Fast — suitable for deeper networks and industrial applications.
  \item Scales to larger architectures (e.g., CNNs).
\end{itemize}

\textbf{Limitations}:
\begin{itemize}
  \item Incomplete — may miss real counterexamples.
  \item Overapproximation can lead to false alarms.
\end{itemize}

Use AI\textsuperscript{2} or Planet when:
\begin{quote}
You care more about high-throughput screening of possible vulnerabilities than formal guarantees — or when you're running many checks in a continuous integration pipeline.
\end{quote}

\subsection{ERAN and DeepPoly — Hybrid Abstraction with Refinement}

\textbf{ERAN} uses zonotope and DeepPoly abstract domains to combine speed and relative precision. It is one of the most practical tools for verifying modern deep networks trained in PyTorch or TensorFlow.

\textbf{Strengths}:
\begin{itemize}
  \item Supports common network architectures.
  \item Can refine over-approximations dynamically.
\end{itemize}

\textbf{Limitations}:
\begin{itemize}
  \item Still limited on very deep or highly recurrent networks.
  \item Requires tuning of abstraction parameters.
\end{itemize}

Use ERAN when:
\begin{quote}
You want fast, approximate verification with feedback on uncertainty — particularly for evaluating robustness and safety in production-trained models.
\end{quote}

\subsection{So, Which One Do You Choose?}

Let the risk shape your verifier.

\begin{itemize}
  \item \textbf{Safety-critical system with small network?} Use \textbf{Reluplex} or \textbf{Marabou}.
  \item \textbf{Large image classifier?} Start with \textbf{AI\textsuperscript{2}}, \textbf{Planet}, or \textbf{ERAN}.
  \item \textbf{Do you care more about formal proof or high coverage?} That’s your tradeoff between soundness and scalability.
  \item \textbf{Need to run nightly regressions for edge-case bugs?} Use abstract verifiers as filters, and apply precise solvers only when needed.
\end{itemize}

\begin{quote}
\textit{There is no universal verifier — only verifiers appropriate to your universe.}
\end{quote}

\vspace{1em}
\noindent
In the next section, we’ll zoom in on how verification interacts with adversarial robustness — and how to design networks with verification in mind.
