\section{De Moivre and the Emergence of Mathematical Probability}

\subsection{From Counting to Distributions}

Building on the insights of Pascal, Fermat, and Bernoulli, Abraham de Moivre (1667--1754) carried probability theory into a new phase: systematic, quantitative, and deeply mathematical.

In his work \textit{The Doctrine of Chances} (1718), de Moivre developed methods to calculate the probabilities of various outcomes by using combinatorics — the art of counting arrangements and possibilities.

\subsection{The Binomial Distribution}

De Moivre recognized that repeated independent trials — such as flipping a coin \( n \) times — could be modeled using powers of binomials. He formalized the use of binomial coefficients to determine the likelihood of achieving a particular number of successes.

He established the now-fundamental formula:

\[
P(\text{k successes}) = \binom{n}{k} p^k (1-p)^{n-k}
\]

where \( p \) is the probability of success on a single trial.

This structure, which became known as the \textbf{binomial distribution}, allowed for the systematic computation of probabilities over sequences of random events — turning probability into a calculable science.

\subsection{The First Glimpse of the Bell Curve}

De Moivre's vision extended even further. He noticed that as the number of trials \( n \) increased, the shape of the binomial distribution began to resemble the smooth, symmetric curve of what we now call the normal distribution.

This early form of the \textbf{Central Limit Theorem} — the idea that aggregated randomness tends toward a stable, Gaussian-like distribution — anticipated one of the most important concepts in all of statistics.

\subsection{Toward Modern Statistics}

Through de Moivre's work, probability theory evolved from a collection of methods for gambling problems into a rigorous mathematical framework. His contributions introduced a new era:

\begin{itemize}
    \item Systematic use of combinatorics to model uncertainty.
    \item Recognition that large numbers of trials smooth out randomness into predictable patterns.
    \item Early formulation of convergence concepts that underpin modern statistical inference.
\end{itemize}

By connecting combinatorics, probability, and the geometry of distributions, de Moivre opened a path that would be followed by Laplace, Gauss, and eventually the creators of modern statistics.

\begin{quote}
    \textit{``It is a question of counting, but it is also a question of seeing order arise from chance.''}
\end{quote}
