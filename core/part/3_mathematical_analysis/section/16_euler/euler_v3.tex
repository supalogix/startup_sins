\section{Euler: The Exponential Bridges Everything (1740s)}

If Newton found calculus in series, and Leibniz turned it into a grammar, \textbf{Leonhard Euler} revealed its secret melody: the exponential function.

For Euler, the constant \( e \) wasn’t just a number—it was the gateway to something deeper. He saw that \( e \) was simply the value of the exponential function at \( x = 1 \):

\[
e^x = \sum_{n=0}^{\infty} \frac{x^n}{n!}
\]

The real hero wasn’t \( e \) itself, but the entire exponential function \( e^x \). Why? Because it let you solve differential equations—the equations that describe change, growth, decay, oscillation, and diffusion. The exponential function wasn’t just a formula—it was a universal key.

\vspace{1em}

\subsection{A Function That Solves Itself}

Euler noticed something remarkable: the exponential function is its own derivative:

\[
\frac{d}{dx} e^x = e^x
\]

This seemingly simple fact made it the natural solution to countless differential equations. Every time a process changed proportionally to its current value—radioactive decay, compound interest, population growth—the exponential function was quietly at work.

\vspace{1em}

\subsection{The Bridge to Trigonometry}

But Euler’s most astonishing discovery wasn’t just about solving equations—it was about connecting worlds. He found that the exponential function could describe rotation, by letting \( x \) be imaginary:

\[
e^{ix} = \cos x + i \sin x
\]

This equation—now known as \textbf{Euler’s formula}—wove together algebra, geometry, trigonometry, and analysis in a single, elegant thread. Through it, waves, rotations, and oscillations became different faces of the same exponential function.

It also unlocked new ways of thinking about Fourier series: whether as sums of sines and cosines, or as sums of complex exponentials. Music, signal processing, quantum mechanics—everything vibrating could now be analyzed through Euler’s lens.

\vspace{1em}

\subsection{From Heat to Randomness}

The exponential function didn’t stop at circles and waves. It also appeared in diffusion, through solutions to the \textbf{heat equation}. The heat kernel—describing how heat spreads from a point—has the shape of a normal distribution whose variance grows over time. And as heat spreads, probability follows: the same exponential underpins the \textbf{central limit theorem}, explaining why randomness converges to a bell curve.

\vspace{1em}

\subsection{A Few Curious Appearances}

Even outside deep mathematics, \( e \) sneaks into life’s puzzles:

\begin{itemize}
    \item At a chaotic coat check, if everyone randomly grabs a coat, the chance that no one gets their own is about \( \frac{1}{e} \).
    \item In love, if you expect to meet \( N \) potential partners, the optimal stopping rule is to reject the first \( \frac{N}{e} \) and choose the next one who surpasses all before.
\end{itemize}

\vspace{1em}

\begin{center}
\textit{If Newton saw calculus in series, and Leibniz in symbols, Euler heard it in the exponential’s song—a harmony uniting change, rotation, heat, and chance.}
\end{center}