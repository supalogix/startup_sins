\section{Wallis: Arithmetic of Powers and Infinitesimals (1656)}

John Wallis extended Cavalieri's indivisible approach using infinite series and arithmetic summation of powers.

He wrote expressions such as:

$$
1^n + 2^n + 3^n + \ldots + (x - 1)^n
$$

to describe the accumulation of infinitesimal rectangles. Wallis formalized this as a method for calculating the total magnitude formed by summing terms of the form:

\begin{quote}
If a quantity be divided into an infinite number of parts, then the sum of the parts raised to the \( n \)-th power may be taken as proportional to a higher power of the whole.
\end{quote}

He generalized this across integer and fractional powers, deriving tabulated values for various cases. Though Wallis lacked a symbolic integral sign, he used sequences and infinite arithmetic processes to estimate areas.

\vspace{1em}

\begin{table}[H]
    \centering
    \renewcommand{\arraystretch}{1.4}
    \setlength{\tabcolsep}{12pt}
    \begin{tabular}{|c|c|c|}
    \hline
    \textbf{Exponentis Valor} & \textbf{Aream Figuris Subtus} & \textbf{Valor Expressus} \\
    \hline
    $n = 0$ & Figura Recta & $1$ \\
    $n = 1$ & Triangulum & $\dfrac{1}{2}$ \\
    $n = 2$ & Parabolica Secundi Gradus & $\dfrac{1}{3}$ \\
    $n = 3$ & Cubica Figura & $\dfrac{1}{4}$ \\
    $n = 4$ & Quartica Figura & $\dfrac{1}{5}$ \\
    $n = 5$ & Quintica Figura & $\dfrac{1}{6}$ \\
    $n = \frac{1}{2}$ & Semiparabolica & $\dfrac{2}{3}$ \\
    $n = -1$ & Figura Logarithmica & $\infty$ \\
    \hline
    \end{tabular}
    \caption{\textbf{Tabula Arearum Figurarum Exponentium} — Table of Areas Bounded by Powers}
\end{table}


\begin{tcolorbox}[colback=gray!5!white, colframe=black, title=\textbf{Historical Sidenote: Wallis and the Ghost of Integration}, fonttitle=\bfseries, arc=1.5mm, boxrule=0.4pt]

    John Wallis, writing in an era just before the formal birth of calculus, never used the integral symbol (\( \int \))—because it hadn’t been invented yet. But in his \textit{Arithmetica Infinitorum} (1656), he constructed a systematic table of “area-like” results by summing powers across an infinite number of subdivisions.
    
    He referred to these results in Latin, using phrases such as \textit{Aream Figuris Subtus} (“area of the figures below”) to describe the region under curves defined by powers of the horizontal axis. His terminology included period-specific descriptors like \textit{Semiparabolica} (for roots such as \( x^{1/2} \)) and \textit{Figura Logarithmica} (for the inverse power \( x^{-1} \)).
    
    Wallis's value for the case \( n = -1 \) was marked as infinite—implicitly anticipating the divergent nature of the area under \( \frac{1}{x} \), and unknowingly brushing up against the threshold of the logarithmic integral.
    
    \vspace{0.3em}
    \textit{In short: Wallis didn’t integrate—but he came tantalizingly close.}
\end{tcolorbox}


