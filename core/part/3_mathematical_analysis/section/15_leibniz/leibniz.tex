\section{Leibniz: Calculus by Rules, Not Just Results (1675–1700)}

While Newton built calculus through series and geometry, \textbf{Gottfried Wilhelm Leibniz} turned it into a formal language.  
He didn’t just want to solve problems—he wanted to mechanize the solving itself.

Leibniz introduced the notation we still use today: \( \frac{dy}{dx} \), \( \int \), and the differential \( dx \).  
But more than that, he developed a systematic method for computing derivatives by applying symbolic rules.

\subsection{Differentiation Becomes a Calculus}

Where Newton worked through fluents and fluxions tied to time, Leibniz treated variables algebraically and operations procedurally.  
He discovered—and codified—general rules for how to differentiate powers, products, quotients, and even early transcendental functions like logarithms and exponentials:

\[
\frac{d}{dx} x^n = n x^{n-1} \qquad \text{(for any positive integer \( n \))}
\]

\[
\frac{d}{dx} \log x = \frac{1}{x}, \qquad \frac{d}{dx} (e^x) = e^x
\]

\[
\frac{d}{dx}(uv) = u'v + uv' \qquad \text{(Product Rule)}
\quad\quad
\frac{d}{dx} \left( \frac{u}{v} \right) = \frac{u'v - uv'}{v^2} \qquad \text{(Quotient Rule)}
\]

Leibniz’s true genius was in how he treated these operations as \textit{syntactic}—a grammar of change.  
You didn’t need to visualize motion or sum areas. You could manipulate symbols according to rules, like playing algebra in fast-forward.

\subsection{From Curves to Calculators}

This perspective was radical. It laid the foundation for treating calculus like a machine—an idea that would echo through later thinkers like Laplace, Lagrange, and eventually into the logic of modern computers.

\vspace{1em}

\begin{center}
\textit{If Newton saw calculus in the curves of nature, Leibniz heard it in the grammar of reason.}
\end{center}

