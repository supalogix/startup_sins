\subsection{Why This Mattered — Especially After Weierstrass}

This wasn’t an idle metaphysical game. Analysis in the 1800s had uncovered functions so wild they seemed to defy everything calculus had promised.

\textbf{Weierstrass} had created functions that were continuous everywhere but differentiable nowhere — wild, erratic, mathematically legal monsters. But to even define such functions, you needed a solid concept of continuity. That meant you needed limits. And to define limits, you needed a complete number system.

Dedekind gave Weierstrass the ground to stand on.

By showing how to construct the reals from within the rationals — using nothing more than logical partitioning — Dedekind made it possible to treat even the strangest functions with mathematical precision.

\begin{tcolorbox}[colback=gray!5!white, colframe=black!80!white, title={Historical Sidenote: Filling the Gaps with Logic}]
When Dedekind introduced his cuts, some mathematicians resisted. The idea that a number could be defined by a gap — a hole between two sets — felt too abstract. But Dedekind insisted: it’s not the digits that matter, it’s the position.

As he wrote:
\begin{quote}
\textit{"The essence of number lies entirely in its being a part of a simply ordered system... determined solely by its position in that system."}
\end{quote}

With that philosophy, the real number line became not a collection of decimal strings, but a complete and ordered structure — no gaps, no guessing, no infinity without grounding.
\end{tcolorbox}


\begin{HistoricalSidebar}{Positivism and the Suspicion of Abstraction}

    In the 19th century, European intellectual life was steeped in \textbf{positivism}---a philosophy championed by thinkers like Auguste Comte and Ernst Mach. Positivism insisted that knowledge must be grounded in empirical observation and verifiable facts. Anything metaphysical, speculative, or abstract was to be avoided or dismissed as meaningless.

    \medskip
    
    This attitude didn’t stay confined to philosophy. It shaped science, and indirectly, mathematics. While positivism itself wasn’t a dominant philosophy among mathematicians, its suspicion of abstraction resonated with key figures like \textbf{Leopold Kronecker}, who led the conservative Berlin school of mathematics.
    
    \medskip
    
    Kronecker’s famous dictum, \emph{“God made the integers; all else is the work of man,”} reflected a deep distrust of non-constructive entities. Infinite sets, irrational numbers defined by Dedekind cuts, or numbers existing merely “in thought”---these were uncomfortable notions for a mathematical culture shaped by positivist leanings toward the concrete and the explicitly constructed.

    \medskip
    
    Against this backdrop, \textbf{Richard Dedekind’s} work landed like an alien text. His definitions of real numbers via Dedekind cuts, his structural view of arithmetic, and his willingness to embrace completed infinities were seen by some as dangerously abstract---even metaphysical.

    \medskip
    
    Though Dedekind wasn’t a philosopher of metaphysics, his mathematical formalism implicitly challenged the positivist epistemology seeping through German intellectual life. He wasn’t just defining numbers; he was redefining what it meant for a number to \emph{be}.

    \medskip
    
    In a culture wary of abstraction for abstraction’s sake, Dedekind’s project wasn’t merely mathematically radical---it was philosophically provocative, whether he intended it or not.
    
    \begin{quote}
    \textit{In an age suspicious of invisible structures, Dedekind quietly insisted that mathematics was built from them.}
    \end{quote}
    
\end{HistoricalSidebar}
