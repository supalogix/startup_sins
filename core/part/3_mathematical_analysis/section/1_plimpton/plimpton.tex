\section{Plimpton 322: The Tablet of Triangles (c. 1800 BCE)}

Long before mathematics had philosophers, it had scribes.

\textbf{Plimpton 322} is a clay tablet from ancient Mesopotamia, dated to around 1800 BCE. On its surface: fifteen rows of cuneiform numbers written in base-60.  
To a modern mathematician, these numbers describe a series of \textbf{right triangles}—and not just any triangles, but ones with perfectly rational sides.

\[
a^2 + b^2 = c^2
\]

Each row contains a triple of numbers that satisfy this relationship. Today, we’d call them \textbf{Pythagorean triples}.  
But there’s no mysticism here. This wasn’t philosophy. It was paperwork.

\subsection{Algorithmic Geometry}

The structure of the tablet reveals something more profound:  
the numbers aren’t random—they’re generated systematically.

Historians believe the tablet may have been built from a now-lost algorithm using whole number ratios, perhaps even a version of:

\[
a = p^2 - q^2, \quad b = 2pq, \quad c = p^2 + q^2
\]

This means Babylonian mathematicians weren’t just aware of these relationships—they could \textit{produce} them at will.

Not prove. Not derive.  
\textbf{Generate.}

\subsection{Mathematics Without Proof, but With Power}

Plimpton 322 is a window into a world where mathematics was:

\begin{itemize}
    \item Base-60
    \item Tabulated
    \item Repeatable
    \item Applied
\end{itemize}

It may have been used for surveying, construction, teaching, or even astronomical prediction.  
Whatever its use, it wasn't speculative—it was functional.

The Babylonians weren’t theorizing. They were computing.  
They didn’t need Euclidean axioms. They had procedures.  
And when a procedure works, you use it—again and again.

\begin{tcolorbox}[title=Historical Sidebar: Scribes and Systems, colback=gray!5, colframe=black, fonttitle=\bfseries]

  Babylonian mathematics wasn’t abstract. It was institutional.

  Clay tablets like Plimpton 322 weren’t private discoveries—they were tools, exercises, perhaps even standardized reference sheets for students in scribal schools.

  These schools trained bureaucrats to manage land, grain, calendars, and construction.  
  To do that well, you needed numbers. Not just to count, but to calculate. To predict. To model.

  And so they built a mathematical system that emphasized rules, tables, and operational control.  
  What we now call “algorithms,” they called “procedures.”

\end{tcolorbox}

\vspace{1em}

\begin{center}
\textit{Plimpton 322 is not a relic of abstract thought.  
It’s a record of applied intelligence.}
\end{center}
