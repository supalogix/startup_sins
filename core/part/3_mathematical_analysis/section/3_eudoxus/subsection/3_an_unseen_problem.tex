\subsection{An Unseen Problem}
Eudoxus’ system was brilliant; but like all brilliant fixes, it only solved the problem people knew they had.

There was still an assumption hiding beneath all of this: that numbers (whatever they were) and magnitudes (whatever they were measuring) behaved in a smooth, predictable way. That they sat neatly in order, filling space without gaps, without strange jumps or missing pieces.

But what if that assumption wasn’t quite right?

The Greeks weren’t asking that question. They were too busy celebrating that proportions still worked, even in the face of irrational numbers. Their system was airtight --- until someone, centuries later, started poking at its foundations.

Because hidden inside the number line was something nobody had really thought to check.

\begin{figure}[H]
\centering
\begin{tikzpicture}[every node/.style={font=\footnotesize}]

% Panel 1 — Eudoxus happy
\comicpanel{0}{4}
  {Pythagorean}
  {Eudoxus}
  {\textbf{Eudoxus:} See? Ratios still work, even with irrational magnitudes! Crisis averted!}
  {(0,-0.5)}

% Panel 2 — Pythagorean celebrating
\comicpanel{6.5}{4}
  {Pythagorean}
  {Eudoxus}
  {\textbf{Pythagorean:} Finally, math makes sense again.  
Everything lines up, smooth and perfect.}
  {(0,-0.5)}

% Panel 3 — Eudoxus starting to worry
\comicpanel{0}{0}
  {Pythagorean}
  {Eudoxus}
  {\textbf{Eudoxus:} …Right?  
There aren’t any weird gaps hiding between the numbers?}
  {(0,0.8)}

% Panel 4 — Pythagorean unsure
\comicpanel{6.5}{0}
  {Pythagorean}
  {Eudoxus}
  {\textbf{Pythagorean:} Wait—  
are we supposed to be checking for those?}
  {(0,0.8)}

\end{tikzpicture}
\caption{Sometimes the most dangerous problems are the ones you don't know to look for.}
\end{figure}