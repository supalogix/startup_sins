\section{Epistemology and the Foundations of Mathematical Analysis: How We Learned to Trust Numbers (Sort Of)}

\subsection{What Is Epistemology?: Or, Why Mathematicians Had a Panic Attack About Truth}

Let’s get real for a second. “Epistemology” sounds like something you’d catch from a bad philosophy lecture—but it’s actually a fundamental part of how math got its act together.

Epistemology is the study of \textbf{knowledge}. How do we know what we know? How do we know it’s \emph{true}? And can we prove it—or are we all just really good at pretending?

Now, ancient math had a certain swagger. Eudoxus, Euclid, Archimedes—they built towering logical systems on top of geometric intuition and clever reasoning. And for a while, that was enough.

But then things got messy.

We started inventing new numbers. We tried slicing infinities into bite-sized pieces. We started saying things like “the area under the curve” and “the limit as epsilon approaches zero.” And eventually, someone asked the inevitable:

\textbf{“Wait... how do we know this is actually valid?”}

\subsection{From Diagrams to Definitions: Or, When Math Had to Grow Up and Get Rigor}

What followed was centuries of epistemological growing pains.

\textbf{Simon Stevin} said: “Let’s stop messing around and treat decimals like real things.”  
\textbf{Napier} and \textbf{Cavalieri} pushed logarithms and infinitesimals, while \textbf{Wallis} and \textbf{Barrow} tiptoed toward calculus without ever quite defining what a limit was.

And then came the hammer: \textbf{Cauchy}, armed with epsilon-delta precision and the vibe of a man who had seen too many math students get away with murder.

He wasn’t alone. Over time, the mathematical world decided that vibes and clever pictures weren’t enough—we needed rock-solid definitions. \textbf{Riemann} built integrals. \textbf{Weierstrass} demolished intuition. \textbf{Dedekind} and \textbf{Cantor} built number systems and infinite sets from scratch.

And just when you thought the anxiety couldn’t get any worse, along came \textbf{Borel}, \textbf{Lebesgue}, and \textbf{Kolmogorov}, who said, “Cool theory, bro. Now let’s make it measurable, complete, and probabilistically sound.”

In short: math became self-aware. And like any self-aware system, it immediately started doubting itself.

\subsection{From Proof to Prediction: Or, Why This Still Matters in Machine Learning}

So why does any of this matter? Why should we care about 19th-century French analysts having existential crises?

Because machine learning is facing the same questions—just with bigger data and worse debugging tools.

Every time you deploy a model, you’re relying on math born from centuries of epistemological struggle. You’re trusting convergence, optimization, measurability, and a whole mess of assumptions about continuity and generalization. And somewhere under all that, you’re silently asking:

\textbf{“How do I know this is real?”}

Understanding the history of mathematical analysis isn’t just a nerd flex—it’s a survival skill. Because when your model fails in production and your confidence interval evaporates, you’ll want to know which parts of the math were built on granite… and which were built on vibes.

So we’re going to tell that story.  
Not just who proved what—but how they convinced themselves (and everyone else) that it was actually \emph{true}.

Because before we had machine learning, we had math.  
And before we had math, we had a question: \textbf{“Can I trust this?”}
