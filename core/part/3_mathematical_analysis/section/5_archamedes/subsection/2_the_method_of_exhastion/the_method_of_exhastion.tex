\subsection{The Method of Exhaustion: Approximating Infinity}  

Archimedes didn’t have calculus, but he had something that looked a lot like it: the method of exhaustion.  

And he didn’t invent it from scratch—he built on the earlier work of **Eudoxus**, who had already developed a way to rigorously compare magnitudes without measuring them directly. Eudoxus’s idea was simple but revolutionary: \textbf{trap an unknown quantity between known ones}, and keep tightening the trap.

Archimedes took this logic and supercharged it.  

\begin{itemize}
    \item Instead of trying to calculate an area or volume directly, he surrounded it with shapes whose areas he \textbf{could} calculate.
    \item By making those shapes smaller and more numerous, he squeezed the unknown more and more tightly.
    \item If the difference between the approximation and the true value became smaller than any quantity you could name—then, boom: he had "exhausted" the gap.
\end{itemize}

For example, suppose you want to find the area of a circle. Archimedes didn’t know the formula \(\pi r^2\), but he knew how to calculate the area of polygons. So he did this:

\begin{enumerate}
    \item Start with a hexagon inscribed in the circle.
    \item Double the number of sides to form a dodecagon.
    \item Keep going—24, 48, 96 sides—until the polygon is hugging the circle so tightly, there's nowhere left for the true area to hide.
\end{enumerate}

At each step, he used **Eudoxus’s logic of comparison** to argue that the circle’s area must lie between the area of the inscribed polygon and that of a circumscribed one.

Eventually, this gave him one of the first rigorous estimates of \(\pi\), accurate to within less than a hundredth.  

This wasn’t just a clever trick: it was the first serious attempt in history to handle the idea of taking a process to the limit. It’s not quite calculus—but it’s definitely calculus-adjacent.

\subimport*{figure}{exhaustion.tex}

\begin{figure}[H]
\centering
\begin{tikzpicture}[every node/.style={font=\footnotesize}, scale=1]

% Panel 1 — Archimedes with a circle and hexagon
\comicpanel{0}{4}
  {Archimedes}
  {Student}
  {Eudoxus showed me how to trap irrational numbers. I just do the same—with polygons.}
  {(-0.3,-0.6)}

% Panel 2 — Student squints skeptically
\comicpanel{6.5}{4}
  {Archimedes}
  {Student}
  {I started with 6 sides. Now I’m up to 96.}
  {(-0.2,-0.6)}

% Panel 3 — Student getting concerned
\comicpanel{0}{0}
  {Archimedes}
  {Student}
  {Eventually, the difference between my polygon and the circle becomes smaller than anything you can name.}
  {(0.2,-0.6)}

% Panel 4 — Student muttering
\comicpanel{6.5}{0}
  {Archimedes}
  {Student}
  {So you invented calculus without inventing calculus. Cool cool cool.}
  {(0.5,-0.5)}

\end{tikzpicture}
\caption{Eudoxus gave him the logic. Archimedes gave it teeth.}
\end{figure}
