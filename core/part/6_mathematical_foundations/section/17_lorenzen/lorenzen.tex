\section{Lorenzen in the Loop: Foundations Without Metaphysics}

If Lakatos taught us that science evolves through patches and repairs, then \textbf{Paul Lorenzen} wondered whether we could rebuild mathematics itself—not from abstract truths, but from human operations.

Where Dedekind abstracted numbers into infinite cuts, and Hilbert axiomatized geometry with formal systems, Lorenzen asked a different question:

\begin{quote}
    “What if mathematics wasn’t a discovery of ideal objects, but a construction of things we can actually do?”
\end{quote}

His answer was \textbf{constructivist logic} and \textbf{operational foundations}—a radical attempt to ground mathematics not in set theory, platonism, or metaphysics, but in the concrete operations a mathematician could perform: drawing a line, marking a point, counting an object, proving a statement.

In Lorenzen’s view, a proof wasn’t a static artifact. It was an \emph{operation}. Logic wasn’t an abstract calculus of truth—it was a sequence of moves in a game of construction.

\vspace{1em}

\subsection{Proof Theory as the Engine of Construction}

This operational perspective fed directly into \textbf{proof theory}: the study of proofs as mathematical objects in their own right.

Instead of asking whether a theorem was “true” in some external sense, proof theory asks:

\begin{quote}
    “How was this theorem constructed? What steps are allowed? What transformations preserve validity?”
\end{quote}

Every proof becomes a traceable, inspectable sequence of operations—a blueprint of its own making.

Where Dedekind’s cuts and Hilbert’s axioms reached skyward into abstraction, Lorenzen’s constructivism dug back down into the soil of doing. It wasn’t about what exists “out there.” It was about what could be carried out, constructed, verified by an agent embedded in a world.

\begin{center}
    \textit{Not being, but building.}
\end{center}

\vspace{1em}

\subsection{From Operational Logic to Verified Learning}

At first glance, Lorenzen’s project might seem distant from modern machine learning. But proof theory—his intellectual descendant—has become a surprising ally in one of ML’s growing challenges:

\textbf{How do we know the model is right?}

As machine learning systems enter safety-critical domains—medical diagnostics, autonomous driving, legal decision-making—the old black-box approach starts to look risky. We don’t just want predictions. We want guarantees.

This is where \textbf{formal verification} enters the chat: an effort to prove, mathematically, that a system satisfies certain properties under all possible inputs.

And behind formal verification? \textbf{Proof theory.}

Whether verifying a neural network’s robustness, proving that a model cannot exceed a certain error under adversarial attack, or certifying fairness properties, these methods translate learned systems into logical structures amenable to proof.

In effect, we’re bringing the operational rigor Lorenzen envisioned—proofs as sequences of verifiable operations—back into the messy world of empirical, stochastic models.

\vspace{1em}

\begin{tcolorbox}[colback=gray!5!white, colframe=black, title=\textbf{Sidebar: Why Proof Theory Matters for ML}, fonttitle=\bfseries, arc=1.5mm, boxrule=0.4pt]
\textbf{Proof theory:} The study of proofs as formal objects—how they’re constructed, transformed, and verified.

\textbf{In machine learning:}  
\begin{itemize}
  \item Certifying that a neural network is robust to small perturbations.
  \item Proving that a model satisfies fairness constraints.
  \item Verifying that a learned controller will never enter unsafe states.
\end{itemize}

It’s not just about accuracy—it’s about \emph{provable guarantees}.
\end{tcolorbox}

\vspace{1em}

\subsection{A Constructivist Echo in the Age of AI}

In an ironic twist, the operational philosophy that sought to ground mathematics in human activity now finds itself grounding the verification of non-human learners.

Lorenzen’s project wasn’t just an intellectual curiosity. It was an early recognition that meaning, validity, and reliability are tied to what can be built, checked, and operationalized—not just what can be posited in theory.

And as machine learning continues to scale beyond human interpretability, the need for operational guarantees grows louder.

\begin{quote}
In the end, even the wildest AI models must answer the old constructivist question:  
\emph{“What can you prove?”}
\end{quote}
