\section{Hume and the Empirical Turn: When Foundations Begged for Evidence}

If Spinoza placed mathematics at the heart of the cosmos, David Hume (1711–1776) pulled the rug out from under it.

Where rationalists like Spinoza and Descartes believed that knowledge could be deduced from first principles—clear, necessary truths derived by logic—Hume offered a far more unsettling proposition:

\begin{quote}
\textit{What if reason isn't the foundation of knowledge, but a habit of the mind?}
\end{quote}

Hume was a radical empiricist. For him, the only real source of knowledge was **experience**—what we see, hear, touch, or remember. Everything else was a mental construction.

That included causality.  
That included mathematical certainty.  
And that, disturbingly, included the very idea of laws of nature.

\subsection*{The Problem of Induction}

Hume’s most devastating critique came in the form of a simple question:

\begin{quote}
\textbf{Why do we believe that the future will resemble the past?}
\end{quote}

Every scientific law, every mathematical model applied to the real world, relies on the assumption that the universe is consistent—that what worked yesterday will work tomorrow.

But Hume showed that this assumption isn’t rationally justified. We believe in induction because it has worked before... which is itself a circular argument. There’s no logical necessity in causal connections—only psychological expectation.

\textbf{Cause and effect}, he argued, is not observed. All we ever see is one thing following another. The connection is something the mind adds after the fact.

\subsection*{Mathematics in a World Without Necessity}

What made Hume's critique particularly troubling for mathematicians and philosophers was this:  

\begin{quote}
\textit{If we can't rationally justify induction, how can we ground any claim about the real world—including those based on mathematics?}
\end{quote}

Pure mathematics, Hume conceded, is certain—but only because it's tautological. It's true by definition, not by reality.

- \textbf{“2 + 2 = 4”} is undeniable—but only because we defined the symbols that way.
- The moment we try to apply mathematics to experience—to bridges, to stars, to probabilities—we’re back in the uncertain world of induction, expectation, and habit.

\subsection*{The Crisis Hume Created}

Hume didn’t destroy mathematics. But he **fractured** its foundation:

\begin{itemize}
    \item Rationalists had treated math as the key to the universe.
    \item Hume treated math as the key to a well-organized filing cabinet—internally consistent, but disconnected from causality or necessity in nature.
    \item The result was a deep epistemological unease: how can we be sure that our most elegant equations mean anything real?
\end{itemize}

\begin{tcolorbox}[colback=gray!5!white, colframe=black!75!white, title={Hume’s Fork: The Two Kinds of Knowledge}]
Hume divided all knowable statements into two categories:

\begin{itemize}
    \item \textbf{Relations of ideas} — logical truths (like math), necessarily true but disconnected from the physical world.
    \item \textbf{Matters of fact} — based on observation, but never guaranteed.
\end{itemize}

Mathematics belongs to the first.  
Science tries to describe the second.  
But no bridge between them is ever truly certain.
\end{tcolorbox}

\subsection*{After Hume: The Search for Solid Ground}

Hume didn’t kill rationalism, but he **forced it to justify itself**.

- Kant would famously say that Hume “awoke him from his dogmatic slumber,” prompting a new synthesis between reason and experience.
- Mathematicians would increasingly turn inward, asking whether the certainty of mathematics could survive once it let go of its divine or metaphysical moorings.
- Probability theory, statistics, and later Bayesian inference would all emerge—consciously or not—in response to the deep challenge Hume posed.

\begin{quote}
\textbf{Spinoza gave us a world built from logic.  
Hume reminded us that the world doesn't owe us certainty.}
\end{quote}

And so, as the Enlightenment unfolded, the foundation of mathematics shifted once again—from the perfection of divine reason to the unpredictability of human perception.

It would be decades before anyone could say, with confidence, what mathematics was truly built on.

