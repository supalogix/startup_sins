\subsection{The Shadows of Numbers (or: Zeno Ruins Everything)}

Plato saw the danger and chose the metaphysical route. Motion, change, imperfection—these weren’t fundamental features of reality. They were symptoms of corruption. Flickering shadows on the cave wall. The true world, the one worth reasoning about, was still, eternal, and perfect: the realm of Forms.

\textbf{Enter Zeno.}

Zeno’s paradoxes weren’t just clever puzzles; they were existential threats to the very notion of continuity. If motion was real, Zeno argued, then it required completing an infinite number of steps. But how could that be possible? How could anything ever move?

\textit{(An arrow can’t fly. Achilles can’t outrun a tortoise. Reality can’t actually… happen.)}

\vspace{0.5em} To the Greeks, motion felt obvious. But Zeno showed it wasn’t logically obvious. Not if you assumed space and time were smooth, uninterrupted continua.
Not if you assumed the number line had no gaps.

\textbf{And yet—everyone kept assuming.}

Rather than confront the paradox head-on, Plato reframed the problem: the contradictions of motion weren’t flaws in logic—they were flaws in perception. The senses deceived; the world of becoming was broken. Only the changeless realm of mathematical Forms was real.

But this was a philosophical maneuver, not a solution. The question lingered:

\textit{What does it even mean for space and time to be continuous?}

Zeno wasn’t just trolling. He was revealing a conceptual fault line that ancient philosophy evaded—and modern mathematics would spend centuries trying to repair.

\begin{figure}[H]
\centering
\begin{tikzpicture}[every node/.style={font=\footnotesize}]

% Panel 1 — Zeno explaining the paradox
\comicpanel{0}{4}
  {Student}
  {Zeno}
  {\textbf{Zeno:} To move from A to B, you must first go halfway.  
Then half of that. Then half of that...  
Infinite steps. So, you never arrive.}
  {(0,-0.5)}

% Panel 2 — Student thinking
\comicpanel{6.5}{4}
  {Student}
  {Zeno}
  {\textbf{Student:} Wait—are you saying walking is logically impossible?}
  {(0,-0.5)}

% Panel 3 — Plato enters
\comicpanel{0}{0}
  {Zeno}
  {Plato}
  {\textbf{Plato:} Don’t worry. Movement only happens in the imperfect world.  
The true Forms remain still.}
  {(0,0.8)}

% Panel 4 — Student, more confused
\comicpanel{6.5}{0}
  {Student}
  {Plato}
  {\textbf{Student:} So I can’t walk, but it’s okay,  
because the perfect version of me isn’t moving anyway?}
  {(0,0.8)}

\end{tikzpicture}
\caption{Zeno exposed the paradox of motion. Plato responded with metaphysics. The rest of the world just kept walking.}
\end{figure}
