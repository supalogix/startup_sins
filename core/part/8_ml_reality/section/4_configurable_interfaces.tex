\subsection{Configurable Interfaces: Because We Had to Deploy at 2AM to Change a Single Parameter}

\textbf{Act I: The Eve of Glory}

It’s the night before a major product launch. The team is buzzing. Everything’s green.

Then someone notices: the fraud detection model’s confidence threshold is a little too strict — it’s blocking legitimate users.

``No problem,'' says the ML engineer. ``We’ll just change it real quick.''

\vspace{1em}
\textbf{Act II: The Hardcode Curse}

Turns out, the threshold is:
\begin{itemize}
    \item Hardcoded in the model training script
    \item Wrapped in a Docker container
    \item Version-locked in a Jenkins pipeline
    \item Operated by exactly one person
\end{itemize}

And that one person? On vacation. In a different time zone. With no cell reception.

\vspace{1em}
\textbf{Act III: The 2AM Ritual}

Two engineers, now wide-eyed and wired on cold brew, SSH into a staging box at 2AM.\\
They find the training script. They find the threshold. They manually edit a Python file.

Then they restart the container and pray the YAML deployment doesn’t explode.

It’s a fix. Technically.

\vspace{1em}
\textbf{Act IV: The Hidden Cost}

The parameter change works. But it cost:
\begin{itemize}
    \item Two engineers’ sleep cycles
    \item Several years off a DevOps lead’s life
    \item A potential rollback plan no one tested
\end{itemize}

All for a single float value.

\vspace{1em}
\textbf{Epilogue: Why Configurable Interfaces Matter}

If the threshold had been passed via a config file, a CLI flag, or an environment variable --- you know, like a normal system --- this would’ve been a two-minute fix.

Instead, it became a fire drill wrapped in technical debt wrapped in sleep deprivation.

Interfaces should be designed for change. Especially the ones you’ll need the night before a launch.