\subsection{The DSM-ML: Diagnosing the Everyday Madness of Machine Learning Teams}

If the field of machine learning had the self-awareness of clinical psychology, it would publish its own \textbf{DSM-ML}: \textit{the Diagnostic and Statistical Manual of Machine Learning Disorders}. Like its psychiatric counterpart, it wouldn't describe rare pathologies—it would catalog the standard dysfunctions that have become so pervasive, so predictable, that they’re practically diagnostic criteria for a ``mature'' data team.

Because let’s be honest: the myth of the clean, modular, versioned ML pipeline is just that: a myth. In reality, the modern pipeline looks like a burnout diagram: hazy goals, shifting data definitions, surprise schema changes, ghosted owners, ``experimental'' branches with god-level access, metrics with no units, and dashboards that haven't loaded since Q2. You don’t build the system you want... you inherit the one that accidentally evolved after three interns, two restructurings, and a rushed product launch.

This section is an attempt at classification. Not to judge, but to \textbf{diagnose}. Because just like the DSM doesn't blame the patient for the system they’re in, this isn’t about blaming the engineers—it’s about naming the dysfunction so you can see it clearly. There’s the \textbf{``Frankenstack Syndrome''} (where every part of the system was built by someone with a different job title and none of them talk to each other). The \textbf{``Config Whisperer Delusion''} (when one person holds the entire tuning process in their head and refuses to write it down). The \textbf{``Observability Anxiety Spiral''} (where the only metric being monitored is whether the monitoring system is up). And, of course, the \textbf{``Prod-as-Staging Disorder''}, so common it might as well be airborne.

We explore these archetypes not just to laugh (though yes, please do), but because dysfunction left undiagnosed becomes dysfunction institutionalized. And once it’s institutionalized, it becomes unfixable—not because it’s hard to fix, but because no one remembers that it *wasn’t always like this*.

This is your mirror. Your audit. Your group therapy session for codebases. You are not alone. And you are not hallucinating. The dysfunction is real. But it is also—diagnosable.

\begin{tcolorbox}[
    title=Sidebar: The DSM — A Taxonomy of Dysfunction,
    colback=gray!5,
    colframe=black,
    fonttitle=\bfseries,
    sharp corners=south,
    boxrule=0.5pt,
    enhanced,
    breakable
  ]
  The \textbf{Diagnostic and Statistical Manual of Mental Disorders (DSM)} was first published in 1952 by the American Psychiatric Association. It was an attempt to do something radical: give names and structure to chaos. At a time when mental illness was either misunderstood or mythologized, the DSM offered a cold, clinical lens: dysfunction not as moral failure or divine punishment, but as a \textit{pattern}.
  
  Its evolution reflects more than medicine: it maps how a society learns to catalog its own dysfunction. DSM-I was 130 pages. DSM-5 is over 900. Turns out, once you start looking, dysfunction is everywhere.
  
  The DSM didn’t cure anyone overnight. But it gave people a name for their suffering, and a starting point for fixing it. Because step one isn’t fixing the problem. It’s recognizing it.
\end{tcolorbox}


